\section{Experimental results} \label{sec-experiments}

In this section, we describe a proof-of-concept implementation of our
proposed algorithm as an extension of the model checker ARMC 
\cite{RybalPodelskiPADL07}.


\paragraph{Tool description}
The verifier we built takes as input a number of functions (written in
the C language) representing threads that should execute concurrently.
The input file also contains the description of an initial state and
a number of assertions to be proven correct.
Our tool uses a frontend based on the CIL infrastructure
\cite{NeculaCC02} to translate a C program to its corresponding
multi-threaded transition system that is formalized in
Section~\ref{sec-prelim}.
The main component of our tool is an implementation of our algorithm
done using SICStus Prolog \cite{sicstus}.

An important design decision in our implementation concerns the 
treatment of control-location and data variables.
Even if both control-location variables and data variables can be
handled uniformly by our algorithm, we found that different
abstraction domains and refinement for the two domains can lead to
significant improvement.
In our implementation, the \algRefine procedure first splits the
constraints into data variable constraints and control-location
constraints. 
The splitting procedure preserves satisfiability/unsatisfiability of
the original constraint since there is no atomic formula in the
program transitions that relates both control variables and data
variables.
If the data constraints are satisfiable, the algorithm proceeds as in
Figure~\ref{fig-alg-refine}.
If the data constraints are unsatisfiable, our implementation relies
on a specialized refinement procedure (described in
\cite{GuptaATVA10}) that takes advantage of the simpler form of
control counterexamples.
For these counterexamples, control variables range over a finite
domain and no atomic formula from the program transitions involves
different control variables.

\begin{table*}[t]
  \begin{center}
  \begin{tabular}{||l|c|c|cc|cc||}
    \hline
    \multicolumn{3}{||c|}{Benchmark programs} &\multicolumn{4}{|c||}{Our algorithm}\\\hline
    Name                                  &LOC&Has a modular proof?&\cc{With bias} & \ccbars{No bias} \\\hline 
    \hline
    Fig2-cex\cite{LuASPLOS08}             &  33 & No & \ipNo     & 0.2s     & \ipNo         & 0.2s \\\hline
    Fig2-fixed\cite{LuASPLOS08}           &  38 & Yes& \ipYes-Modular& 0.8s & \ipYes-Modular& 0.7s \\\hline
    Fig4-cex\cite{LuASPLOS08}             & 175 & No & \ipNo     & 4.5s     & \ipNo         & 3.7s \\\hline
    Fig4-fixed\cite{LuASPLOS08}           & 168 & Yes& \ipYes-Modular& 1.5s & \ipYes        & 11.1s\\\hline
    Bluetooth2\cite{QadeerPLDI04}         &  90 & No & \ipYes    & 29.1s    & \ipYes        & 11.2s\\\hline
    Bluetooth2-fixed                      &  90 & No & \ipYes    & 3.7s     & \ipYes        & 0.4s \\\hline
    Bluetooth3-fixed                      &  90 & No & \ipYes    & 135s     & \ipYes        & 9.7s \\\hline
    Scull\cite{CorbetBook05}              & 451 & Yes& \ipYes-Modular&128.5s& \ccbars{\ipTO}   \\\hline
    \hline
    Dekker\cite{BarDavidDISC03}           &  39 & No & \ipYes        &11.1s & \ipYes        & 6s   \\\hline
    Peterson\cite{BarDavidDISC03}         &  26 & No & \ipYes        & 4.7s & \ipYes        & 3.9s \\\hline
    Readers-writer-lock\cite{CalvinESOP}  &  22 & Yes& \ipYes-Modular& 0.2s & \ipYes        & 0.4s \\\hline
    Time varying mutex\cite{CalvinESOP}  &  29 & No & \ipYes         &11.8s & \ipYes        & 3.1s \\\hline
    Szymanski\cite{SzymanskiICS88}        &  43 & No & \ipYes        & 32s  & \ipYes        & 8.8s \\\hline
    Na\"{i}veBakery\cite{MannSafetyBook95}&  22 & Yes& \ipYes-Modular& 2.5s & \ipYes        & 3s   \\\hline
    Bakery\cite{LamportCACM74}            &  37 & No & \ipYes        &105.4s& \ipYes        & 101s \\\hline
    Lamport\cite{LamportTCS87}            &  62 & No & \ipYes        &120.8s& \ipYes        &  97s \\\hline
    QRCU\cite{McKenneyLWN07}              & 120 & Yes& \ipYes-Modular& 34.5s& \ccbars{\ipTO}   \\\hline
  \end{tabular}
  \end{center}
\nocaptionrule
\caption{``Has a modular proof?'' indicates whether the program has a
  modular proof of correctness.
  ``\ipYes'' and ``\ipNo''
  indicate whether the program is proven safe or a 
  counterexample is returned, while ``\ipTO'' stands for time out
  after 15 minutes.
  \ipYes-Modular indicates that a modular proof is found by our
  tool.}
  \label{table-vs-no-bias}
\end{table*}


%%% Local Variables: 
%%% mode: latex
%%% TeX-master: "main"
%%% End: 


\paragraph{Benchmark programs} 

We tested our prototype implementation using a collection of programs
that have intricate correctness proofs for their safety assertions.
The first four programs shown in Table~\ref{table-vs-no-bias} are
derived from two buggy examples highlighted as figures in
\cite{LuASPLOS08}, together with their fixes from the \Mozilla CVS
repository.
The property to verify is that two operations performed by different
threads are executed in the correct order.
The next three examples model the stopping procedure of a Windows NT
Bluetooth driver \cite{QadeerPLDI04}.
\BluetoothTwo contains two threads, one worker thread and another thread
to model the stopping procedure of the driver.
%\BluetoothThree includes an additional stopper thread and illustrates
%a bug in the driver model. 
%(The bug cannot be triggered for the two-thread model.)
\BluetoothTwoFixed and \BluetoothThreeFixed are the fixed versions of
the model with two and respectively three threads.
\Scull\cite{CorbetBook05} is a Linux character device driver that
implements access to a global memory area.
The property to verify is that read and write operations are performed
in critical section.

We also include some examples which are not particularly favorable to
a modular reasoning approach.
These examples are algorithms that establish mutual exclusion and mainly
deal with global variables (no local computation is included in the
critical region).
The mutual exclusion property of the na\"ive version of the Bakery
algorithm \cite{MannSafetyBook95} holds only when assuming
assignments are performed atomically. 
(Our verifier was able to confirm the bug present in the code without
such atomicity assumption.)
\Bakery\cite{LamportCACM74} is the complete version of the Bakery
algorithm, while \Lamport\cite{LamportTCS87} is an algorithm with an
optimized path in the absence of memory contention.
\QRCU\cite{McKenneyLWN07} is an algorithm implementing the
Read-copy-update synchronization algorithm.
It is an alternative to a readers-writer lock having wait-free read
operations.


\paragraph{Performance of our tool} 

To explain our experimental results, we first articulate a working
hypothesis.
This hypothesis suggests that, when verifying a program that does not
have a modular proof, the algorithm with preference for modular
solutions (denoted as \emph{verification with bias}) is expected to
pay a penalty by insisting to search for modular solutions that do
not exist.
On the other hand, for cases where a modular proof does exist, the
non-biased verification could fail to find a modular proof and instead
return a more detailed non-modular proof.
Therefore, the hypothesis suggests that in these cases the biased
verification is expected to succeed faster compared to the non-biased
verification.

We report statistical data for each of the programs in
Table~\ref{table-vs-no-bias}.
We show the number of lines of code (LOC) and whether a modular proof
exists for a program (see Column 3).
Our implementation has two modes. 
Column 4 shows the verification results, when using our algorithm with
a preference for modular solutions.
The last column of the table shows the verification results for the 
non-biased implementation of our algorithm.
The results demonstrate that our approach to verification of
multi-threaded programs is feasible and that the constraint solving
procedure with bias is able to produce modular proofs more
often than the non-biased verification.
Furthermore, without the bias, the verification procedure times-out
for Scull and QRCU examples showing the benefits of modular proofs.

As another experiment, we tested some of our smaller examples
using two state-of-the-art model checkers for sequential C programs,
Blast \cite{BlastLazy02} and ARMC\cite{RybalchenkoARMC}.
For each of the tested programs (Fig2-fixed, Fig4-fixed, Dekker,
Peterson, and Lamport), we instrumented the program counter as
explicit program variables ($\pcone$ and $\pctwo$) and obtained a
sequential model of the multi-threaded examples.
Both Blast and ARMC eagerly consider all interleavings and obtained
timeouts after 30 minutes for both Fig4-fixed and Lamport.
Comparatively, our tool exploits the thread structure of these
programs and obtains conclusive verification results fast.


%%% Local Variables: 
%%% mode: latex
%%% TeX-master: "main"
%%% End: 
