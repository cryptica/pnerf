\section{Thread reachability and environment transitions}
\label{sec-algo-safety}

In this section, we present our rely-guarantee based verification
algorithm for proving safety properties of multi-threaded programs.
The algorithm is based on Theorem~\ref{thm-abst-fix} and consists of
three  main steps.
The first step computes for each thread a tree that is decorated by
abstract states and environment transitions, so-called \aret, and
analyses the discovered abstract states.
If an intersection with the error states of the program is found, then
the second step generates a set of corresponding Horn clauses, see
Section~\ref{sec-refinement}.
At the third step, we solve the constraint defined by the conjunction
of the generated Horn clauses and use the solutions to the refine the
abstraction functions used for the \aret computation, see
Section~\ref{sec-horn-solving}. 

\linespread{1.1}
\begin{figure}[t]
\hrule
  \centering
  \begin{minipage}[t]{.04\columnwidth}
   \mbox{} \\ \\ \\ \\ \\ \\ \\ \\ \\ \\ \\ \\ \\ \\ \\ \\[\jot] 
   1 \\ 2 \\ 3 \\ 4 \\ 5 \\ 6 \\ 7 \\ 8 \\ 9 \\ 10 \\
   11 \\ 12 \\ 13 \\ 14 \\ 15 \\ 16 \\ 17 \\ 18 \\ 19 % \\ 20 \\
%   21 \\ 22 \\ 23 \\ 24 \\ 25 \\ 26 \\ 27 \\ 28 \\ 29 % \\ 30 \\ 
%   31 \\ 32 \\ 33 \\ 34 \\ 35 \\ 36
  \end{minipage}
  \begin{minipage}[t]{.94\columnwidth}
    \mbox{}\\
    \algFunction \algMain\\
    \algInput\\
    \tabT $\program$ - program with $\Num$ threads\\
    \algVars\\
    \tabT $\artPreds_i, \abst_i$ - 
    \begin{array}[t]{@{}l}
      \text{predicates for thread } i \text{ and}\\ 
      \text{corresponding state abstraction function} 
    \end{array}\\
    \tabT $\artTPredsOf{i}{j}, \abstTrans_{\envFromTo{i}{j}}$ - 
    \begin{array}[t]{@{}l@{}}
      \text{transition predicates for pair of threads } i, j \text{ and}\\ 
      \text{corresponding transition abstraction function}
    \end{array}\\
    \tabT $\artReachOf{i}$ - abstract states of thread $i$\\
    \tabT $\EnvTransOf{i}$ - abstract environment transitions of thread $i$\\
    \tabT $\artParent$ - 
    \begin{array}[t]{@{}l}
      \text{parent function for abstract states and}\\
      \text{environment transitions}
    \end{array}\\
    \tabT $\artTId$ - 
    \begin{array}[t]{@{}l}
      \text{parent thread function for abstract states and}\\
      \text{environment transitions}
    \end{array}\\
    \algBegin\\
    \tabT \algForeach $i \neq j \in \ThreadIds$ \algDo \\
    \tabTT $\artPreds_i$ \algAssgn $\artTPredsOf{i}{j}$ \algAssgn
    $\emptyset$\\
    \tabT \algRepeat\\
    \tabTT \algForeach $i \neq j \in \ThreadIds$ \algDo \\
    \tabTTT $\abst_i$ \algAssgn $\lambda S. \Land \set{\dot{p}\in\artPreds_i\mid 
      \forall \Vars : S \limplies \dot{p}}$ \\
    \tabTTT $\abstTrans_{\envFromTo{i}{j}}$ \algAssgn $\lambda T.\Land\set{\ddot{p}
    \in\artTPredsOf{i}{j}\mid \forall \Vars, \Vars' : T\limplies
    \ddot{p}}$ \\
    \tabTT $\algAbstReachEnvSolveOf{()}$ \\
    \tabTT \algIf 
    \begin{array}[t]{@{}l}
      \text{exists } S_1\in\artReachOf{1}, \dots, S_{\Num} \in
      \artReachOf{\Num} \text{ such that}\\
      \quad \exists \Vars : S_1 \land \dots \land S_\Num \land
      \SymbError 
    \end{array}
    \\
    \tabTT \algThen\\
    \tabTTT \algTry \\
    \tabTTTT $\algRefine(S_1,\dots,S_\Num)$\\
    \tabTTT \algWith \algUnsolvable\\
    \tabTTTT $\StateDst$ \algAssgn some $S_i$ from $\set{S_1, \dots, S_\Num}$\\
    \tabTTTT \algReturn ``counterexample $\algCexPathOf{\StateDst}$'' \\
    \tabTT \algElse \\
    \tabTTT \algReturn ``% ''
    \begin{array}[t]{@{}l}
      \text{program }\program\text{ is safe with the proof}\\
      \Lor\artReachOf{1},\dots,\Lor\artReachOf{\Num}, \Lor\EnvTransOf{1},\dots,\Lor\EnvTransOf{\Num}\text{''}
    \end{array}
    \\
    \tabT \algUntil $\ltrue$ \\
    \algEnd.
  \end{minipage}
  \vspace{1ex}
  \linespread{1}
  \caption{Function \algMain for verifying safety of the
    multi-threaded program~$\program$.
  }
  \label{fig-alg-main}
\end{figure}

%%% Local Variables: 
%%% mode: latex
%%% TeX-master: "main"
%%% End: 


\paragraph{Function \algMain}

The main function of our algorithm \algMain is shown in
Figure~\ref{fig-alg-main}.
\algMain takes as input the multi-threaded program $\program\ $.
The \algRepeat loop iterates through the three main steps of the
algorithm.
First, we construct the abstraction functions $\abst_i$ and
$\abstTrans_{\envFromTo{i}{j}}$ at lines~4--6 from a given set of
(transition) predicates, which is empty initially.
Next, the \aret computation is performed in line~7 using these
abstraction functions.
In lines~8--9, the abstract states in the computed \aret's are
analyzed wrt.\ the safety property.
In case of a positive outcome of this check, \algMain constructs and
returns a safety proof in lines~17--18.
If the safety check fails, then \algRefine is executed on the
violating abstract states.
If \algRefine terminates normally, and hence succeeds in eliminating
the violation by refining the abstraction functions, then \algMain
continues with the next iteration of the \algRepeat loop.
In case an \algUnsolvable exception is thrown, \algCexPath from
Figure~\ref{fig-alg-cex-path} constructs a counterexample path that we
report to the user.

\paragraph{Procedure \algAbstReachEnvSolve}
\linespread{1.2}
\begin{figure}[t]
\hrule
  \centering
  \begin{minipage}[t]{.04\columnwidth}
   \mbox{} \\ \\ \\
   1 \\ 2 \\ 3 \\ 4 \\ 5 \\ 6 \\ 7 \\ 8 \\ 9 \\ 10 \\
   11 \\ 12 \\ 13 \\ 14 \\ 15 \\ 16 \\ 17 \\ 18 \\ 19 \\ 20 \\
   21 \\ 22 \\ 23 \\ 24 \\ 25 \\ 26 \\ 27 \\ 28 %\\ 29 \\ 30 \\
%   31 \\ 32 \\ 33 \\ 34 \\ 35 \\ 36
  \end{minipage}
  \begin{minipage}[t]{.94\columnwidth}
    \mbox{}\\
    \algProcedure \algAbstReachEnvSolve\\
    \algBegin\\
    \tabT $\artParent$ \algAssgn $\artTId$ \algAssgn $\bot$ \hfill
    \algComment{the empty function}\\
    \tabT \algForeach $i \in \ThreadIds$ \algDo\\
    \tabTT $\artReachOf{i}$ \algAssgn $\set{\abst_i(\SymbInit)}$\\
    \tabTT $\EnvTransOf{i}$ \algAssgn $\emptyset$\\
    \tabT \algRepeat\\
    \tabTT $\doneFlag$ \algAssgn $\ltrue$\\
    \tabTT \algForeach $i \in \ThreadIds$ and $S \in \artReachOf{i}$
    \algDo \\
    \tabTTT \algComment{states}\\
    \tabTTT \algForeach $\rel \in \Trans_i \cup \EnvTransOf{i}$ \algDo \\
%    \tabTTTT $S'$ \algAssgn $\abst_i(\post(\rel, S))$\\
    \tabTTTT $S'$ \algAssgn 
    \begin{array}[t]{@{}l}
      \algIf\  \rel \in \Trans_i \ \algThen\  \abst_i(\post(\rel, S))\\
      \algElse\ \abst_i(\post(\rel\land \relLocEqOf{i}, S))\\
    \end{array}\\
    \tabTTTT \algIf $\neg(\exists S''\in \artReachOf{i}\; \forall \Vars : S' \limplies
    S'')$ \algThen\\
    \tabTTTTT $\artReachOf{i}$ \algAssgn $\set{S'} \cup \artReachOf{i}$\\
    \tabTTTTT $\artParent(S')$ \algAssgn $(S, \rel)$\\
    \tabTTTTT $\artTIdOf{S'}$ \algAssgn $i$\\
    \tabTTTTT $\doneFlag$ \algAssgn $\lfalse$\\
    \tabTTT \algDone\\
    \tabTTT \algComment{environment transitions}\\
    \tabTTT \algForeach $\rel \in \Trans_i$ and $j \in \ThreadIds \setminus 
    \set{i}$ \algDo \\
    \tabTTTT $\rel'$ \algAssgn $\abstTrans_{\envFromTo{i}{j}}(S \land
    \rel) $\\
    \tabTTTT \algIf $\neg(\exists \rel''\in \Trans_j\cup\EnvTransOf{j} \;
    \forall \Vars, \Vars' : \rel' \limplies \rel'')$ \algThen\\
    \tabTTTTT $\EnvTransOf{j}$ \algAssgn $\set{\rel'} \cup \EnvTransOf{j}$\\
    \tabTTTTT $\artParent(\rel')$ \algAssgn $(S,\rel)$\\
    \tabTTTTT $\artTIdOf{\rel'}$ \algAssgn $i$\\
    \tabTTTTT $\doneFlag$ \algAssgn $\lfalse$\\
    \tabTTT \algDone\\
    \tabTT \algDone\\
    \tabT \algUntil $\doneFlag$\\
    \algEnd
  \end{minipage}
  \vspace{1ex}
  \linespread{1}
  \caption{Procedure \algAbstReachEnvSolve implements \aret
    computation.
We assume that the iterator statements in lines 7 and 9 make an
immutable snapshot of their domains~$\artReachOf{i}$ and
$\EnvTransOf{i}$, respectively. 
For example, this implies that each addition of $S'$ in line 12 is
unnoticed in line 7 until the next iteration of the \algRepeat
loop. 
}
  \label{fig-alg-abst-reach}
\end{figure}
%%% Local Variables: 
%%% mode: latex
%%% TeX-master: "main"
%%% End: 


See Figure~\ref{fig-alg-abst-reach} for the procedure
\algAbstReachEnvSolve that implements \aret computation using the
abstraction functions $\abst_i$ and $\abstTrans_{\envFromTo{i}{j}}\ $.
We use $\artParent$ and $\artTId$ to maintain information about the
constructed trees, and initialize them with the empty function $\bot$
in line~1.
$\artReachOf{i}$ and $\EnvTransOf{i}$ keep track of abstract states
and environment transitions for a thread $i\in\ThreadIds\ $. 
$\EnvTransOf{i}$ is initialized to an empty set in line~4, while
$\artReachOf{i}$ contains the abstraction of the initial program
states computed for the thread~$i\ $.
The \aret computation is performed iteratively in the \algRepeat loop,
see lines~5--27.

% =========== compute states

The first part of the loop (see lines~7--17) implements a standard, 
least fixpoint computation over reachable states.
At line~7, the algorithm picks an already reachable states
$S\in\artReachOf{i}$ in order to compute its abstract successors. 
After computing at lines~10--11 one successor of $S\ $, line~12 implements
a fixpoint check, which succeeds if $S'$ contains program states that
have not been reached yet.
The new states reachable in thread $i$ are stored in
$\artReachOf{i}\ $.
At line~14, the function $\artParent$ is updated to keep track of the
child-parent relation between abstract states, while $\artTId$ maps
the new reachable state to its parent thread.

% ======= compute environment transitions

The second part of the loop (see lines~$19$--$26$) performs a least
fixpoint computation over environment transitions. 
Each time a transition from a thread $i$ is picked at line~7, the
abstraction of its effect computed in line~20 is propagated to each
other thread~$j\ $.
Note however, that the propagation only happens for environment
transitions that are not subsumed by the existing ones, which is
checked in line~21.
Additional environment transitions are recorded in line~22.
Environment transitions are taken into consideration when computing
abstract state reachability, see line~9.

Upon termination, which is guaranteed by the finiteness of our
abstract domains, the function \algAbstReachEnvSolve computes sets of
abstract states $\artReachOf{1}, \dots, \artReachOf{\Num}$ and sets of
environment transitions $\EnvTransOf{1}, \dots, \EnvTransOf{\Num}\ $.

\linespread{1.1}
\begin{figure}[t]
\hrule
  \centering
  \begin{minipage}[t]{.04\columnwidth}
   \mbox{} \\ \\ \\ \\ \\ 
   1 \\ 2 \\ 3 \\ 4 \\ 5 \\ 6 \\ 7 % \\ 8 \\ 9  \\ 10 %\\ 11 \\ 12 
% \\ 13 \\ 14 \\ 15 \\ 16 \\ 17 \\ 18 %\\ 19 \\ 20 \\
%   21 \\ 22 \\ 23 \\ 24 \\ 25 \\ 26 \\ 27 \\ 28 \\ 29 % \\ 30 \\
%      31 \\ 32 \\ 33 \\ 34 \\ 35 \\ 36
  \end{minipage}
  \begin{minipage}[t]{.94\columnwidth}
    \mbox{}\\
    \algFunction \algCexPath\\
    \algInput \\
    \tabT $\StateDst$ - abstract state\\
    \algBegin\\
    \tabT \algMatch $\artParent(\StateDst)$ \algWith\\
    \tabT \algCase{$(\StateSrc, \rel)$} \\
    \tabTT $\path$ \algAssgn $\algCexPath(\StateSrc)$\\
    \tabTT \algMatch $\artParent(\rel)$ \algWith\\
    \tabTT \algCase{$(\StateOrig, \OrigRel)$} \algReturn $\path \cdot \OrigRel$\\
    \tabTT \algCase{$\bot$} \algReturn $\path\cdot\rel$\\
    \tabT \algCase{$\bot$} \algReturn $\emptypath$ \hfill 
    \algComment{the empty sequence}\\
    \algEnd
  \end{minipage}
  \vspace{1ex}
  \linespread{1}
  \caption{Function \algCexPath takes as input an abstract state
    $\StateDst$ and returns a sequence of transitions that lead to
    $\StateDst$.
  }
  \label{fig-alg-cex-path}
\end{figure}
%%% Local Variables: 
%%% mode: latex
%%% TeX-master: "main"
%%% End: 



%%% Local Variables: 
%%% mode: latex
%%% TeX-master: "main"
%%% End: 
