\section{Solving Horn clauses over linear inequalities} \label{sec-horn-solving}

As presented in the previous section, \algRefine calls the function
\algSolveHornClauses.
In this section we present a function \algSolveLinearHornClauses that
can be used as an implementation of \algSolveHornClauses takes as
input a set of clauses $\HornClauses$ over linear inequalities that is
recursion-free.

To simplify the presentation of the algorithm, we make two additional
assumptions on~$\HornClauses$.
First, we assume that for each pair of clauses $(\dots\!\,) \limplies
\HornBody(\FreeVars)$ and $(\dots\!\,) \limplies
\HornBody'(\FreeVars')$ from $\HornClauses$ we have~$\HornBody \neq
\HornBody'$ and~$\HornBody \neq (\leq)\neq \HornBody'$.
Second, we assume that $\HornClauses$ contains a clause $(\dots\!\,)
\limplies \lfalse\ $. 

The additional assumptions are satisfied by the clauses generated in
Section~\ref{sec-refinement}.
In case \algSolveLinearHornClauses is applied on a set of
recursion-free Horn clauses over linear arithmetic that violates the
two assumptions above, we can apply a certain renaming of relation
symbols and introduction of additional clauses to meet the
assumptions.



%-----------------------------------------------------------------------------------

\begin{figure*}[p]
  \small
  \centering
  \hrule
%-----------------------------------------------------------------------------
\begin{minipage}[t]{0.13\linewidth} 
  \vspace{12pt}
  \centering
  \begin{tikzpicture}[->,>=stealth',shorten >=1pt,auto, node distance=1.1cm,
                    semithick]

  \node[stateBox,node distance=4cm] (m1) {$m_1$};
  \node[stateBox, below of=m1] (m2) {$m_2$};
  \path (m1) edge node[left]{$\rel_{0}$} (m2);
  \node[stateBox, below of=m2] (m3) {$m_3$};
  \path (m2) edge node[left]{$\rel_1$} (m3);
  \node[stateBox, errorState, below of=m3] (m4) {$m_4$};
  \path (m3) edge node[left]{$e_2$} (m4);


  \node[stateBox,right of=m2, node distance=1.2cm] (n1) {$n_1$};
  \node[stateBox, below of=n1] (n2) {$n_2$};
  \path (n1) edge node[right]{$\rel_2$} (n2);
  \node[stateBox,errorState, below of=n2] (n3) {$n_3$};
  \path (n2) edge node[right]{$e_1$} (n3);
  \path (2mm, -16mm) edge[envTransition] (10mm,-29mm);
  \path (10mm,-16mm) edge[envTransition]( 2mm,-29mm);
  \end{tikzpicture}~\vspace{12pt}\\
\textbf{(a)}
\end{minipage}
%---------------------------------------------------------------------------
\begin{minipage}[t]{0.38\linewidth}
\centering
\begin{equation*}
  \begin{array}[t]{ll}
  \SymbInit \limplies \unkOf{m_1}(\Vars)           \\
  \unkOf{m_1}(\Vars) \land  \rel_{0} \limplies \unkOf{m_2}(\Vars')  \\
  \unkOf{m_2}(\Vars) \land  \rel_1 \limplies \unkOf{m_3}(\Vars')  \\
  \unkOf{m_2}(\Vars) \land  \rel_1 \limplies
  \unkOf{e_1}(\Vars,\Vars')  \\
  \unkOf{m_3}(\Vars) \land \unkOf{e_2}(\Vars,\Vars') \land \relLocEqOf{1} \limplies\unkOf{m_4}(\Vars')
  \vspace{10pt}\\
  \SymbInit \limplies \unkOf{n_1}(\Vars) \\
  \unkOf{n_1}(\Vars) \land \rel_{2} \limplies \unkOf{n_2}(\Vars') \\
  \unkOf{n_1}(\Vars) \land \rel_{2} \limplies \unkOf{e_2}(\Vars,\Vars') \\
  \unkOf{n_2}(\Vars) \land \unkOf{e_1}(\Vars,\Vars')  \land  \relLocEqOf{2} \limplies \unkOf{n_3}(\Vars')
  \vspace{10pt}\\
  \unkOf{m_4}(\Vars) \land \unkOf{n_3}(\Vars) \land  \SymbError \limplies \lfalse 
\end{array}
\end{equation*}
\textbf{(b)}
\end{minipage}
\hfill
%----------------------------------------------------------------------------
\begin{minipage}[t]{0.46\linewidth}
\centering
\begin{equation*}
  \begin{array}[t]{ll}
  \SymbInit \limplies \unkOf{m_1}(\GlobVars,\LocVarsOf{1})           \\
  \unkOf{m_1}(\GlobVars,\LocVarsOf{1}) \land  \rel_{0} \limplies \unkOf{m_2}(\GlobVars',\LocVarsOf{1}')  \\
  \unkOf{m_2}(\GlobVars,\LocVarsOf{1}) \land  \rel_1 \limplies \unkOf{m_3}(\GlobVars',\LocVarsOf{1}')  \\
  \unkOf{m_2}(\GlobVars,\LocVarsOf{1}) \land  \rel_1 \limplies
  \unkOf{e_1}(\GlobVars,\GlobVars')  \\
  \unkOf{m_3}(\GlobVars,\LocVarsOf{1}) \land \unkOf{e_2}(\GlobVars,\GlobVars')
  \land \relLocEqOf{1} \limplies \unkOf{m_4}(\GlobVars',\LocVarsOf{1}') 
  \vspace{10pt}\\
  \SymbInit \limplies \unkOf{n_1}(\GlobVars,\LocVarsOf{2}) \\
  \unkOf{n_1}(\GlobVars,\LocVarsOf{2}) \land \rel_{2} \limplies \unkOf{n_2}(\GlobVars',\LocVarsOf{2}') \\
  \unkOf{n_1}(\GlobVars,\LocVarsOf{2}) \land \rel_{2} \limplies \unkOf{e_2}(\GlobVars,\GlobVars') \\
  \unkOf{n_2}(\GlobVars,\LocVarsOf{2}) \land \unkOf{e_1}(\GlobVars,\GlobVars')  \land\relLocEqOf{2} \limplies 
  \unkOf{n_3}(\GlobVars',\LocVarsOf{2}')\vspace{10pt}\\

  \unkOf{m_4}(\GlobVars,\LocVarsOf{1})\land\unkOf{n_3}(\GlobVars,\LocVarsOf{2})\land\SymbError\limplies\lfalse 
\end{array}
\end{equation*}
\textbf{(c)}
\end{minipage}
\hfill 
\vspace{1ex}
%----------------------------------------------------------------------------------------------
\hrule
\begin{minipage}[t]{\linewidth} 
\centering
\vspace{1pt}
  \begin{tikzpicture}[<-,>=stealth',shorten >=1pt,auto, node distance=9mm,
                    semithick]
 \node[stateBox] (err) {$1:\lfalse$};

%----------------------------
 \node [stateBox,xshift=-83mm,yshift=2mm,above left of=err] (m47) {$2:\unkOf{m_4}(\Vars^1)$};\path (m47) edge (err);
 \node [stateBox,xshift=-34mm,yshift=3mm, above left of=m47] (m36) {$3:\unkOf{m_3}(\Vars^2)$};\path (m36) edge (m47);
 \node [stateBox, above of=m36 ] (m25) {$4:\unkOf{m_2}(\Vars^3)$}; \path (m25) edge (m36);
 \node [stateBox, above of=m25 ] (m14) {$5:\unkOf{m_1}(\Vars^4)$}; \path (m14) edge (m25);

 \node [draw,xshift=13.5mm,yshift=3mm,above right of=m36] (t1) {$8:\rel_1(\Vars^3,\Vars^2)$};\path (t1) edge (m36);
 \node [draw,xshift=13.5mm,yshift=3mm,above right of=m25] (te1) {$7:\rel_{0}(\Vars^4,\Vars^3)$};\path (te1)edge(m25);
 \node [draw, above of=m14]  (i1)  {$6:\SymbInit(\Vars^4)$};    \path (i1) edge (m14);

 \node [stateBox,above of=m47] (e2) {$9:\unkOf{e_2}(\Vars^2,\Vars^1)$}; \path (e2) edge (m47);
 \node [stateBox, above of=e2 ] (n16) {$10:\unkOf{n_1}(\Vars^2)$}; \path (n16) edge (e2);

 \node [draw,xshift=17mm,yshift=3mm,above right of=e2] (et2) {$12:\rel_2(\Vars^2,\Vars^1)$}; \path (et2) edge (e2);
 \node [draw, above of=n16] (ei2){$11:\SymbInit(\Vars^2)$};  \path (ei2) edge (n16);

 \node [draw,xshift=17mm,yshift=3mm, above right of=m47,text centered] (eq2) 
 {$13:\relLocEqOf{1}(\Vars^2,\Vars^1)$}; \path (eq2) edge (m47);

%---------------------------

 \node [stateBox,above of=err] (n37) {$14:\unkOf{n_3}(\Vars^1)$};\path (n37) edge (err);
 \node [stateBox,xshift=-36mm,yshift=3mm,above left of=n37 ] (n23) {$15:\unkOf{n_2}(\Vars^5)$};\path (n23)edge(n37);
 \node [stateBox,     above of=n23 ] (n12) {$16:\unkOf{n_1}(\Vars^6)$}; \path (n12) edge (n23);
 
 \node [draw,xshift=14.5mm,yshift=2.5mm,above right of=n23](t2){$18:\rel_2(\Vars^6,\Vars^5)$}; \path (t2) edge (n23);
 \node [draw, above of=n12] (i2)  {$17:\SymbInit(\Vars^6)$}; \path (i2) edge (n12);
 
 \node [stateBox,above of=n37] (e1)  {$19:\unkOf{e_1}(\Vars^5,\Vars^1)$};\path (e1) edge (n37);
 \node [stateBox, above of=e1 ] (m23) {$20:\unkOf{m_2}(\Vars^5)$}; \path (m23) edge (e1);
 \node [stateBox, above of=m23] (m11) {$21:\unkOf{m_1}(\Vars^7)$}; \path (m11) edge (m23);
 
 \node[draw,xshift=17mm,yshift=3mm,above right of=e1] (et1) {$24:\rel_1(\Vars^5,\Vars^1)$};\path (et1) edge (e1);
 \node[draw,xshift=17mm,yshift=3mm,above right of=m23] (ete1) {$23:\rel_0(\Vars^7,\Vars^5)$};\path (ete1) edge (m23);
 \node[draw, above of=m11]  (ei1)  {$22:\SymbInit(\Vars^7)$};\path (ei1) edge (m11);
 
 \node [draw,xshift=17mm,yshift=3mm,above right of=n37,text centered ] (eq1) 
 {$25:\relLocEqOf{2}(\Vars^5,\Vars^1)$}; \path (eq1) edge (n37);

%---------------------------
  \node [draw, xshift=22mm, yshift=-1mm, above of=err] (errn) {$26:\SymbError(\Vars^1)$};\path (errn) edge (err);
  \end{tikzpicture}\vspace{-10pt}\\
\textbf{(d)}
\end{minipage}
\hrule
%-------------------------------------------------------------------------------
% Pred map
\begin{minipage}[t]{\linewidth} 
\centering
\vspace{1pt}
  \begin{tikzpicture}[<-,>=stealth',shorten >=1pt,auto, node distance=9mm,
                    semithick]
 \node[stateBox] (err) {$1:\linfalse$};

%----------------------------
 \node [stateBox,xshift=-83mm,yshift=2mm,above left of=err] (m47) {$2:\linfalse$};\path (m47) edge (err);
 \node [stateBox,xshift=-34mm,yshift=3mm, above left of=m47] (m36) {$3:1\leq\local^2$};\path (m36) edge (m47);
 \node [stateBox, above of=m36 ] (m25) {$4:1\leq\local^3$}; \path (m25) edge (m36);
 \node [stateBox, above of=m25 ] (m14) {$5:\lintrue$}; \path (m14) edge (m25);

 \node [draw,xshift=13.5mm,yshift=3mm,above right of=m36] (t1) {$8:$$\local^3\leq\local^2$};\path (t1) edge (m36);
 \node [draw,xshift=13.5mm,yshift=3mm,above right of=m25] (te1) {$7:1\leq\local^3$};\path (te1)edge(m25);
 \node [draw, above of=m14]  (i1)  {$6:\lintrue$};    \path (i1) edge (m14);

 \node [stateBox,above of=m47] (e2) {$9:\local^2\leq0$}; \path (e2) edge (m47);
 \node [stateBox, above of=e2 ] (n16) {$10:\local^2\leq0$}; \path (n16) edge (e2);

 \node [draw,xshift=16mm,yshift=3mm,above right of=e2] (et2) {$12:\lintrue$}; \path (et2) edge (e2);
 \node [draw, above of=n16] (ei2){$11:\local^2\leq0$};  \path (ei2) edge (n16);

 \node [draw, xshift=16mm,yshift=2mm, above right of=m47,text centered] (eq2) {$13:\lintrue$}; \path (eq2) edge (m47);

%---------------------------
 \node [stateBox,above of=err] (n37) {$14:\lintrue$};\path (n37) edge (err);
 \node [stateBox,xshift=-36mm,yshift=3mm,above left of=n37 ] (n23) {$15:\lintrue$};\path (n23)edge(n37);
 \node [stateBox,     above of=n23 ] (n12) {$16:\lintrue$}; \path (n12) edge (n23);
 
 \node [draw,xshift=14.5mm,yshift=2.5mm,above right of=n23](t2){$18:\lintrue$}; \path (t2) edge (n23);
 \node [draw, above of=n12] (i2)  {$17:\lintrue$}; \path (i2) edge (n12);
 
 \node [stateBox,above of=n37] (e1)  {$19:\lintrue$};\path (e1) edge (n37);
 \node [stateBox, above of=e1 ] (m23) {$20:\lintrue$}; \path (m23) edge (e1);
 \node [stateBox, above of=m23] (m11) {$21:\lintrue$}; \path (m11) edge (m23);
 
 \node[draw,xshift=18mm,yshift=3mm,above right of=e1] (et1) {$24:\lintrue$};\path (et1) edge (e1);
 \node[draw,xshift=18mm,yshift=3mm,above right of=m23] (ete1) {$23:\lintrue$};\path (ete1) edge (m23);
 \node[draw, above of=m11]  (ei1)  {$22:\lintrue$};\path (ei1) edge (m11);
 
 \node [draw,xshift=19mm,yshift=2.5mm,above right of=n37,text centered ] (eq1) 
 {$25:\lintrue$}; \path (eq1) edge (n37);

%---------------------------
  \node [draw, xshift=22mm, yshift=-1mm, above of=err] (errn) {$26:\lintrue$};\path (errn) edge (err);
  \end{tikzpicture}\vspace{-10pt}\\
\textbf{(e)}
\end{minipage}
%-------------------------------------------------------------------------------
% 
\hrule
\begin{minipage}[t]{\linewidth} 
\centering
\vspace{1pt}
  \begin{tikzpicture}[<-,>=stealth',shorten >=1pt,auto, node distance=9mm,
                    semithick]
 \node [stateBox] (m47) {$2:\unkOf{m_4}(\GlobVars^1,\LocVarsOf{1}^1)$};
 \node [stateBox,xshift=-70mm,yshift=3mm, above left of=m47] (m36) {$3:\unkOf{m_3}(\GlobVars^2,\LocVarsOf{1}^2)$};
 \path (m36) edge (m47);
 \node [stateBox, above of=m36 ] (m25) {$4:\unkOf{m_2}(\GlobVars^3,\LocVarsOf{1}^3)$}; \path (m25) edge (m36);
 \node [stateBox, above of=m25 ] (m14) {$5:\unkOf{m_1}(\GlobVars^4,\LocVarsOf{1}^4)$}; \path (m14) edge (m25);

 \node [draw,xshift=29mm,yshift=2mm,above right of=m36] (t1)
 {$8:\rel_1(\GlobVars^3,\LocVarsOf{1}^3,\LocVarsOf{2}^{\textsc{IV}},
            \GlobVars^2,\LocVarsOf{1}^2,\LocVarsOf{2}^{\textsc{V}})$};
 \path (t1) edge (m36);
 \node [draw,xshift=29mm,yshift=2mm,above right of=m25] (te1) 
 {$7:\rel_{0}(\GlobVars^4,\LocVarsOf{1}^4,\LocVarsOf{2}^{\textsc{II}},
              \GlobVars^3,\LocVarsOf{1}^3,\LocVarsOf{2}^{\textsc{III}})$};
 \path (te1) edge (m25);
 \node [draw, above of=m14] (i1)
 {$6:\SymbInit(\GlobVars^4,\LocVarsOf{1}^4,\LocVarsOf{2}^{\textsc{I}})$};
 \path (i1) edge (m14);

 \node [stateBox,above of=m47] (e2) {$9:\unkOf{e_2}(\GlobVars^2,\GlobVars^1)$}; \path (e2) edge (m47);
 \node [stateBox, above of=e2 ] (n16) {$10:\unkOf{n_1}(\GlobVars^2,\LocVarsOf{2}^2)$}; \path (n16) edge (e2);

 \node [draw,xshift=30mm,yshift=2mm,above right of=e2] (et2) 
 {$12:\rel_2(\GlobVars^2,\LocVarsOf{1}^{\textsc{VI}},\LocVarsOf{2}^2,
             \GlobVars^2,\LocVarsOf{1}^{\textsc{VII}},\LocVarsOf{2}^1)$};
 \path (et2) edge (e2);
 \node [draw, above of=n16] (ei2)
 {$11:\SymbInit(\GlobVars^2,\LocVarsOf{1}^{\textsc{VIII}},\LocVarsOf{2}^2)$};
 \path (ei2) edge (n16);

 \node [draw, xshift=18mm,yshift=2mm, above right of=m47,text centered] (eq2) 
 {$13:\dots$}; \path (eq2) edge (m47);
%---------------
  \node [xshift=20mm, yshift=5mm, below right of=m47] (errt) {};\path (m47) edge (errt);
  \node [xshift=20mm, yshift=5mm, below right of=errt] (err) {};\path (errt) edge[-,dashed] (err);
  \end{tikzpicture}\vspace{-7pt}\\
\textbf{(f)}
\end{minipage}
\caption{
(a) Reachability trees constructed by \aret computation.
(b) Corresponding Horn clauses generated using \algMkHornClauses.
(c) Horn clauses generated with preference for modular solutions.
(d) A tree representation of the clauses from (b) as generated by \algMkHornTree.
Each node shows its $\cFormula$ attribute.
The superscript of the node name identifies the set of
variables appearing in the attribute of the node.
(e) The $\cInterPol$ map generated by \algSolveLinearHornClauses from the clauses in (b).
(f) The $\cFormula$ map generated by \algMkHornClauses from the clauses in (c).
}
\label{fig-ex-cex}
\end{figure*}

%%% Local Variables: 
%%% mode: latex
%%% TeX-master: "main"
%%% End: 


\paragraph{Function \algMkHornTree} 

The function \algMkHornTree generates a tree representation for a set
of Horn clauses and is shown in Figure~\ref{fig-alg-mkTree}.
For every relation appearing in the Horn clauses, the algorithm
generates a corresponding tree node.
The children of a node are maintained in a function $\cChildren$ as follows.
Nodes that correspond to linear arithmetic relations have no children,
see lines~11--12.
A node that corresponds to an unknown relation with a relation symbol
$b$ has as children those nodes that represent relation symbols that
depend on~$b\ $.
The $\cFormula$ attribute of the tree nodes is initialized to a linear
arithmetic constraint for leaves of the tree in line~11, and to an
unknown relation for internal tree nodes in line~13.

\paragraph{Function \algSolveLinearHornClauses}
See Figure~\ref{fig-alg-solveHorn} for the pseudocode of the procedure
$\algSolveLinearHornClauses\ $.
This procedure creates a tree representation for $\HornClauses$ at line~2.
At line~3, we build a set containing all the $\cFormula$ attributes of
leaf nodes and store this set in $\cLeaves\ $ .
The input set of Horn clauses is satisfiable if and only if
$\Land \cLeaves$ is unsatisfiable.
If $\Land \cLeaves$ is unsatisfiable, the test at lines~4--5 succeeds
and returns a proof of unsatisfiability in the form of weights for
each linear inequality. 
This test can be implemented using some linear arithmetic constraint
solver.
If the constraint solver fails to find a proof, an exception
$\algUnsolvable$ is thrown at line~12.

At line~7, $\algSolveLinearHornClauses$ calls the procedure
$\algBuildPred\ $, which is presented in Figure~\ref{fig-alg-annotPred}.
This procedure recursively traverses the input tree in postorder.
If this procedure is invoked for a leaf node $\cnode\ $, it directly
computes the value of $\cInterPolOf{\cnode}$ as a linear combination
of atomic formulas with weights given by the $\proofVec$ function (see
line~2).
If this procedure is invoked for an internal node $\cnode\ $, the
attribute $\cInterPol$ of $\cnode$'s children is computed using a
recursive call at line~5.
After completing the recursive call, $\cInterPolOf{\cnode}$ is
calculated by adding the values of the $\cInterPol$ attributes of
$\cnode$'s children.

Since there may be multiple nodes in the tree corresponding to the
same unknown relation, the algorithm has to account for the
$\cInterPol$ attributes of all these nodes.
Therefore, at lines~8--9 we compute solutions for each
$\HornBody(\FreeVars)$ in $\Unknowns$ by taking conjunction of
$\cInterPol$ of each node of the tree that is labeled with
$\HornBody(u)$ for some~$u\ $.

%The algorithm $\algSolveLinearHornClauses$ is closely related to the
%interpolation generation method described in \cite{McMillanTACAS04}.

%----------------------------------------------------------------------------------- 

\begin{theorem}
\algSolveLinearHornClauses computes a solution for a set of Horn
clauses  $\HornClauses$ if and only if the conjunction of the clauses
in $\HornClauses$ is satisfiable.
\end{theorem}
\iffalse
\todo{proof sketch}
\includeProof{
\begin{proof}
  Given that each relational variable appears only once as the head of
  a Horn clause in $\HornClauses\ $, we denote by $body(b(w))$ the left
  hand side of this particular clause.
  Given the recursion-free structure of the set of clauses
  $\HornClauses\ $, we can replace each relational variable $b_i(w_i)$
  that appears in a body of a clause with $body(b_i(w_i))$ (with
  proper renamings).
  This replacement preserves the solutions of $\HornClauses\ $.
  After replacement, the set of Horn clauses is reduced to one Horn
  clause such that all the terms that appear in the body of this
  clause are known predicates.

  From the hypothesis that $\HornClauses$ has a solution, we can
  deduce that the conjunction of the terms in the body of the clause
  is unsatisfiable.
  From the fact that the solution of $\HornClauses$ is in the
  conjunctive linear inequality domain, we can guarantee that a proof
  of the unsatisfiability is found using a linear arithmetic solver.
  Therefore the test at lines~4--5 from \algSolveLinearHornClauses
  succeeds and returns a function $\proofVec\ $.
  From the $\proofVec$ function, \algBuildPred computes solutions for
  each relational variable which satisfy the input set of Horn
  clauses.
\end{proof}
}
\fi

\paragraph{Example}
We illustrate the solving procedure using the same example from the
previous section. 
Given the Horn clauses from Figure~\ref{fig-ex-cex}(b), \algMkHornTree
constructs a tree that is shown in Figure~\ref{fig-ex-cex}(d).
This tree contains nodes which we label for convenience with
identifiers from $1$ to $26\ $.
In Figure~\ref{fig-ex-cex}(d) we show the $\cFormula$ map of the tree.
A witness of the unsatisfiability of $\Land \cLeaves$ is given by the
following atomic formulas:
\begin{equation*}  
  \begin{array}[t]{l}
    \begin{array}[t]{@{}c@{}}
      (\local^{3} \geq 1 )\\
      {\downin}\\
      {\cFormulaOf{7}}    
    \end{array}
    \land 
    \begin{array}[t]{@{}c@{}}
      (\local^{3} = \local^{2})\\
      {\downin} \\ 
      {\cFormulaOf{8}}
    \end{array}
    \land 
    \begin{array}[t]{@{}c@{}}
      (\local^{2}=0)\\
      {\downin}\\ 
      {\cFormulaOf{11}}
    \end{array}
\end{array}\end{equation*}
Our solver treats each linear equality as a conjunction of two linear
inequalities.
The equality $\local^{2}=0$ is split in two inequalities
$\local^{2}\leq0\land -\local^{2}\leq0\ $.
The proof of unsatisfiability is:
%
\begin{equation*}  \begin{array}[t]{l}
(1 \leq \local^{3} )+
(\local^{3} \leq \local^{2} ) +
( \local^{2} \leq 0) = (\linfalse).
\end{array}\end{equation*}
%
This is encoded in the $\proofVec$ map with values of $1$ at locations
corresponding to the three atomic formulas above and values of $0$ for
all the other atomic formula.
Next, we show in Figure~\ref{fig-ex-cex}(e) the values for the
$\cInterPol$ map as computed by $\algBuildPred\ $.
The final solution of the Horn clauses is built by a conjunction of
the $\cInterPol$ attributes for nodes with the same unknown label. 
The resulting solution $\Solution$ is shown below.
%
\begin{equation*} 
  \begin{array}[t]{l@{\;\;=\;\;}l@{\;\;=\;\;}l}
    \SolutionOf{\unkOf{m_1}(\Vars)} & \SolutionOf{\unkOf{n_2}(\Vars)} &
    (\lintrue)\\[\jot]
    \SolutionOf{\unkOf{e_1}(\Vars,\Vars')}& & (\lintrue) \\[\jot]
    \SolutionOf{\unkOf{m_2}(\Vars)} & \SolutionOf{\unkOf{m_3}(\Vars)} &
    (1 \leq \local) \\[\jot]
    \SolutionOf{\unkOf{n_1}(\Vars)} & \SolutionOf{\unkOf{e_2}(\Vars,\Vars')} & 
    (\local \leq 0) 
  \end{array}
\end{equation*}

\linespread{1.2}
\begin{figure}[t]
%\small
\hrule
  \centering
  \begin{minipage}[t]{.04\columnwidth}
    \mbox{} \\ \\ \\ \\ \\ \\ \\ \\ \\ \\ \\
    1 \\ 2 \\ 3 \\ 4 \\ 5 \\ 6 \\ 7 \\ 8 \\ 9 \\ 10 \\ 11 \\ 12
  \end{minipage}
  \begin{minipage}[t]{.94\columnwidth}
    \mbox{}\\
    \algFunction \algSolveLinearHornClauses\\
    \algInput \\
    \tabT $\HornClauses$ - recursion-free Horn clauses over linear inequalities \\
    \tabT $\Unknowns$ - unknown relations \\
    \algVars\\
    \tabT $\cFormula$ - map from node to attribute \\
    \tabT $\cChildren$ - map from node to a set of nodes \\
    \tabT $\cInterPol$ - map from node to an atomic predicate \\
    \tabT $\proofVec$ - weight function for inequalities \\
    \algBegin\\
    \tabT $\cFormula$\algAssgn $\cChildren$ \algAssgn $\bot$ \hfill
    \algComment{the empty function}\\
    \tabT $\rcnode$ \algAssgn $\algMkHornTree(\lfalse)$ \\
    \tabT $\cLeaves$ \algAssgn $\bigcup\set{\cFormulaOf{\cnode} \mid \cChildren(\cnode) = \emptyset}$\\
    \tabT \algIf
    \begin{array}[t]{@{}l}
      \text{exists } \proofVec: \cLeaves \rightarrow \Rationals_{\geq
        0} \text{ such that}\\
      \sum \set{\proofVec(\HornBody(\FixVars)) \cdot \HornBody(\FixVars)
        \mid \HornBody(\FixVars) \in \cLeaves} = (\linfalse)
    \end{array}\\
    \tabT \algThen\\
    \tabTT $ \algBuildPred(\rcnode)$ \\
    \tabTT \algForeach $\HornHead(\FreeVars)\in\Unknowns$ \algDo\\
    \tabTTT $\SolutionOf{\HornHead(\FreeVars)}:=\Land\{\cInterPolOf{n}[\FreeVars/\FixVars]| \cFormulaOf{\cnode}=\HornHead(\FixVars)\}$\\ 
    \tabTT \algReturn $\Solution$ \\
    \tabT \algElse \\
    \tabTT \algThrow \algUnsolvable\\
    \algEnd
  \end{minipage}
  \vspace{1ex}
  \linespread{1}
  \caption{Function \algSolveLinearHornClauses returns a solution for
    a set of recursion-free Horn clauses over linear arithmetic.}
  \label{fig-alg-solveHorn}
\end{figure}
%%% Local Variables: 
%%% mode: latex
%%% TeX-master: "main"
%%% End: 

\linespread{1.2}
\begin{figure}[t]
%\small
\hrule
  \centering
  \begin{minipage}[t]{.04\columnwidth}
   \mbox{}
   \\ \\ \\ \\ \\ 
   1 \\ 2 \\ 3 \\ 4 \\ 5 \\ 6 \\ 7 \\ 8 \\ 9 \\ 10 \\ 11 \\ 12 \\ 13
   \\ 14 \\ 15 \\ 16
  \end{minipage}
  \begin{minipage}[t]{.94\columnwidth}
    \mbox{}\\
    \algFunction \algMkHornTree\\
    \algInput\\
    \tabT $g$ - relation, either $\HornHead(\FixVars)$ or $\lfalse$\\
%    \tabT $\HornClauses$ - set of Horn clauses \\
    \algBegin\\
    \tabT $p, q$ \algAssgn new nodes \\
    \tabT \algMatch $g$ \algWith \\
    \tabT \algCase{$\lfalse$}\\
    \tabTTT $\set{\HornBody_1(\FreeVars_1)\land \dots\land \HornBody_n(\FreeVars_n)
      \rightarrow \lfalse, \; \dots}$ \algAssgn $\HornClauses$\\
    \tabTTT $\FreshVars_1, \dots, \FreshVars_n$ \algAssgn fresh copies
    of $\FreeVars_1, \dots, \FreeVars_n$\\
    \tabTTT $\headSubst$ \algAssgn $\substXtoY{\FreshVars_1}{\FreeVars_1}\cdots\substXtoY{\FreshVars_n}{\FreeVars_n}$\\
    \tabT \algCase{$\HornHead(\FixVars)$}\\
    \tabTTT $\set{\HornBody_1(\FreeVars_1)\land \dots\land \HornBody_n(\FreeVars_n)
      \rightarrow \HornBody(\FreeVars), \; \dots}$ \algAssgn
    $\HornClauses$\\
    \tabTTT $\FreshVars_1, \dots, \FreshVars_n, \FreshVars$ \algAssgn fresh copies
    of $\FreeVars_1, \dots, \FreeVars_n, \FreeVars$\\
    \tabTTT $\headSubst$ \algAssgn $\substXtoY{\FreshVars_1}{\FreeVars_1}\cdots\substXtoY{\FreshVars_n}{\FreeVars_n}\substXtoY{\FixVars}{\FreshVars}$\\
    \tabT $\cFormulaOf{p}$ \algAssgn 
    $\set{
      \HornBody_i(\FreeVars_i)\headSubst
      \mid i\in\betweenOf{1}{n}\land\HornBody_i = (\leq)
    }
    $\\
    \tabT $\cChildren(p)$ \algAssgn $\emptyset$\\
    \tabT $\cFormulaOf{q}$ \algAssgn $g$ \\ 
    \tabT $\cChildren(q)$ \algAssgn \\
    \mbox{} \hfill $
    \begin{array}[t]{@{}l}
      \set{p}\mathrel{\cup} 
      \bigcup
      \set{
        \algMkHornTree(
        \HornBody_i(\FreeVars_i)\headSubst
        ) 
        \mid
        \begin{array}[t]{@{}l}
          i \in \betweenOf{1}{n}\mathrel{\land} 
          \HornBody_i \neq (\leq)}
        \end{array}
      \end{array}
    $ \\
    \tabT \algReturn $q$ \\
    \algEnd
  \end{minipage}
  \vspace{1ex}
  \linespread{1}
  \caption{Function \algMkHornTree.
  Fresh copies are created consistent, e.g., fresh copies of $\set{v_1,
    v_2}$, $\set{v_3, v_1}$ returns $\set{f_1, f_2}$, $\set{f_3 f_1}$, where
  $f_1$, $f_2$, $f_3$ are fresh variables that do not appear anywhere
  else. }
  \label{fig-alg-mkTree}
\end{figure}
%%% Local Variables: 
%%% mode: latex
%%% TeX-master: "main"
%%% End: 

\linespread{1.2}
\begin{figure}[t]
%\small
\hrule
  \centering
  \begin{minipage}[t]{.04\columnwidth}
   \mbox{}
   \\ \\ \\ \\ \\ 
   1 \\ 2 \\ 3 \\ 4 \\ 5 \\ 6 % \\ 7 \\ 8 
  \end{minipage}
  \begin{minipage}[t]{.94\columnwidth}
    \mbox{}\\
    \algProcedure \algBuildPred\\
    \algInput \\
    \tabT $\cnode$ - node of Horn tree\\
    \algBegin \\
    \tabT \algIf $\cChildren(\cnode) = \emptyset$ \algThen \\
    \tabTT $\cInterPolOf{\cnode}$ \algAssgn 
    $
    \sum\set{\proofVecOf{\HornHead(\FixVars)}\cdot\HornHead(\FixVars) \mid
      \HornHead(\FixVars)\in\cFormulaOf{\cnode}}
    $ \\
    \tabT \algElse \\
    \tabTT  \algForeach $\cnode'\in\cChildren(\cnode)$ \algDo \\
    \tabTTT $\algBuildPred(\cnode')$\\
    \tabTT $\cInterPolOf{\cnode} \algAssgn \sum\{\cInterPolOf{\cnode'}|\cnode'\in\cChildren(\cnode)\}$ \\
    \algEnd
  \end{minipage}
  \vspace{1ex}
  \linespread{1}
  \caption{Procedure \algBuildPred.}
  \label{fig-alg-annotPred}
\end{figure}
%%% Local Variables: 
%%% mode: latex
%%% TeX-master: "main"
%%% End: 


\paragraph{Solving the clauses shown in Figure~\ref{fig-ex-cex}(c)}
Given the Horn clauses shown in this figure, \algMkHornTree returns a
tree representation with a similar $\cChildren$ map structure but with
different $\cFormula$ attributes.
The part of the tree that contributes to the proof of unsatisfiability
is shown in Figure~\ref{fig-ex-cex}(f).
The variable $\local^2$ does not appear in the subtree of the node $9$
since $\cFormulaOf{9}=\unkOf{e_2}(\GlobVars^2,\GlobVars^1)\ $. 
Part of this subtree is the node 11.
Let us name the variable at this node as $\local^{\textsc{VIII}}\ $.
The proof of unsatisfiability shown above does no longer hold, since
the following formula is satisfiable:
\[
 (1\leq\local^{3} ) \land (\local^{3} = \local^{2}) \land (\local^{\textsc{VIII}}=0)
\]
However, the conjunction of the elements from the $\cLeaves$ set is
still unsatisfiable, indicating that a modular solution exists.
We find that the following atoms contribute to a proof of unsatisfiability:
\begin{equation*}
  \begin{array}[t]{@{}c@{}}
      \lock^2=1\\
      \downin\\
      {\cFormulaOf{8}}    
    \end{array}
  \land 
  \begin{array}[t]{@{}c@{}}
      \lock^2=0\\
      \downin\\
      {\cFormulaOf{12}}
  \end{array}
\end{equation*}
After splitting the equalities in equivalent inequalities, our
algorithm computes the following solution:
\begin{equation*}  
  \begin{array}[t]{r@{\;}lr@{\;}l}
    \cInterPolOf{8} &= (1\leq\lock^{2} ) & \cInterPolOf{12} &= (\lock^{2}\leq 0 ) \\
    \cInterPolOf{3} &= (1\leq\lock^{2} ) & \cInterPolOf{9} &= (\lock^{2} \leq 0 ) \\
    \cInterPolOf{2} &= (\linfalse)  
  \end{array} 
\end{equation*}
%
From this $\cInterPol$ map, our algorithm derives a solution
$\Solution$ in lines 8--9 and succeeds in computing modular
predicates.

% \todo{ PC vs. Data.
% Each constraint originating from the initial states, the transition
% relations and the error states can be split in to two parts.
% The first constraint is over program counter variables and the other
% constraint relates the data variables. 
% The splitting procedure preserves unsatisfiability of the original
% constraint since there in no atomic formula that relates both program
% counter variables and data variables.
% The original constraint system is unsatisfiable if and only if either
% of the two parts is unsatisfiable.
% For solving the program counter constraints, \algRefinePC at line 13
% implements an algorithm proposed in our earlier work
% \cite{GuptaATVA10}.
% }


%%% Local Variables: 
%%% mode: latex
%%% TeX-master: "main"
%%% End: 
