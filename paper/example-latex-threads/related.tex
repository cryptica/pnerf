\section{Related work}\label{sec-related}

The main inspiration for our work draws from the rely-guarantee
reasoning method \cite{JonesTOPLAS83, RelyGuarantee} and automatic
abstraction refinement approach to verification \cite{ClarkeCEGAR}.

The seminal work on rely-guarantee reasoning \cite{JonesTOPLAS83,
  RelyGuarantee} initially offered an approach to reason about
multi-threaded programs by making explicit the interference between
threads.
Subsequently, rely-guarantee reasoning was used to tackle the problem
of state explosion in verification of multi-threaded programs.
Rely-guarantee reasoning was mechanized and firstly implemented in the 
Calvin model checker \cite{CalvinESOP} for Java shared-memory programs.
Calvin reduces the verification of the multi-threaded program to the
verification of several sequential programs with the help of a
programmer specified environment assumption.
In \cite{ThreadModularFlanaganQadeer}, thread-modular model checking
was proposed to infer automatically environment assumptions that
propagate only global variable changes to other threads. 
The algorithm has low complexity, polynomial in the number of threads,
but is incomplete and fails to discover environment assumptions that
refer to the local states of a thread.
Thread-modular verification is formalized by \cite{MalkisICTAC06} in
the framework of abstract interpretation as Cartesian product of sets
of states. 

% we - weakest environment and refinement by strengthening
% RR - strongest environment and refinement by weakening

The method of \cite{HenzingerPLDI04} uses a richer abstraction scheme
that computes contextual thread reachability, where the context in
which a thread executes includes information on both global and local
states of threads.
The context (or environment) is computed using bisimilarity quotients
in steps that are interleaved with abstract reachability computations.
The verification starts with the strongest possible environment
assumption and, by refinement, the environment is weakened until
it over-approximates the transitions of the other threads.
In contrast, our approach refines iteratively the environment based on
over-approximation, starting with the weakest environment and
strengthening it at every iteration. 
% More importantly, our algorithm integrates the computation of
% environment assumptions with the abstract state reachability.
For abstraction refinement, a counterexample from
\cite{HenzingerPLDI04} is reduced to a concrete sequential path by
replacing environment transitions with their corresponding local
transitions.

The approach of \cite{CohenFMSD09} presents another
solution to overcome the incompleteness of local reasoning.
Guided by counterexamples, it refines the abstraction by exposing 
a local variable of a thread as a global variable. 
This refinement recovers the completeness of reasoning, but is
applicable to finite-state systems and may compute an unnecessarily
precise abstraction.
In contrast, our refinement procedure relies on interpolation and
includes predicates on local variables as needed during verification.

Another approach to overcome the state explosion problem of monolithic
reasoning over multi-threaded programs is to translate the
multi-threaded program to a sequential program assuming a bound on the
number of context switches.
This scheme was initially proposed and implemented in KISS
\cite{QadeerPLDI04}, a multi-threaded checker for C programs, and later
evolved to handle and reproduce even difficult to find Heisenbugs
\cite{MusuvathiOSDI08}.
Monolithic reasoning can be greatly facilitated by using techniques
evolved from partial-order reduction \cite{GodefroidPhd94}, like
dynamic partial-order reduction \cite{FlanaganPOPL05} or peephole
partial order reduction \cite{WangTACAS08}.
Yet another technique to fight state explosion is to factor out
redundancy due to thread replication as proposed in counter
abstraction \cite{PnueliCAV02} and implemented in the 
model checker Boom \cite{BaslerCAV09,BaslerTACAS10}.
% shape analysis for concurrent programs: 
% GotsmanPLDI07, RGSep-Action-Inference VMCAI10, HongseokYang-RCU-unpublished
We view these techniques as paramount in obtaining practical
multi-threaded verifiers, but orthogonal to our proposal for automatic
environment inference.

% Future work:
% - distribute the model checking algorithm: as in DArmc-VMCAI'11. Even more natural for
% compositional model checking algorithms like ours (as also suggested in Cohen-EC'09).
% - prove that multi-threaded programs eventually do something good Cook-POPL'07


%%% Local Variables: 
%%% mode: latex
%%% TeX-master: "main"
%%% End: 
