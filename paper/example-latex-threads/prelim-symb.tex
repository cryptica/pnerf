\section{Preliminaries}\label{sec-prelim}

In this section we briefly describe multi-threaded programs, their
computations and correctness. 
We also introduce auxiliary definitions that we apply for reasoning
about programs.

\paragraph{Programs}

% = \program_1 \parcomp \dots \parcomp \program_\Num 

We consider a \emph{multi-threaded} program $\program$ that consists
of $\Num\geq 1$ concurrent threads.
Let $\ThreadIds$ be the set~$\set{1, \dots, \Num}\ $.
We assume that the \emph{program} variables $\Vars = (\GlobVars,
\LocVarsOf{1}, \dots, \LocVarsOf{\Num})$ are partitioned into
\emph{global} variables $\GlobVars$ that are shared by all threads,
and \emph{local} variables $\LocVarsOf{1}, \dots, \LocVarsOf{\Num}$
that are only accessible by the threads $1, \dots, \Num\ $,
respectively.

The set of \emph{global states} $\Glob$ consists of the valuations of
global variables, and the sets of \emph{local states} $\Loc_1, \dots,
\Loc_\Num$ consist of the valuations of the local variables of
respective threads.
By taking the product of the global and local state spaces, we obtain
the set of \emph{program states} $\States = \Glob \times \Loc_1 \times
\dots \times \Loc_\Num\ $.
We represent sets of program states using assertions over program
variables. 
Binary relations between sets of program states are represented using
assertions over unprimed and primed variables.
Let $\models$ denote the satisfaction relation between (pairs) of
states and assertions.

The set of \emph{initial} program states is denoted by $\SymbInit\ $,
and the set of \emph{error} states is denoted by~$\SymbError\ $.
For each thread $i\in\ThreadIds$ we have a finite set of transition
relations~$\Trans_i\ $, which are abbreviated as \emph{transitions}.
Each transition $\rel \in \Trans_i$ can change the values of the
global variables and the local variables of the thread~$i\ $. 
Let $\relLocEqOf{i}$ be a constraint requiring that the local
variables of the thread $i$ do not change, i.e., $\relLocEqOf{i} =
(\LocVarsOf{i} = \LocVarsOf{i}')\ $.
Then, $\rel \in\Trans_i$ has the form
%
\begin{equation*}
  \rel^{\text{update}}(\GlobVars, \LocVarsOf{i}, \GlobVars', \LocVarsOf{i}') \land
  \Land_{j\in\ThreadIds\setminus{\set{i}}} \relLocEqOf{j}\ ,
\end{equation*}
%
where the first conjunct represents the update of the variables in the
scope of the thread $i$ and the remaining conjuncts ensure that the
local variables of other threads do not change.
%\todo{add notation for update and eq part?}
We write $\relTIdOf{i}$ for the union of the transitions of the thread
$i\ $, i.e., $\relTIdOf{i} = \Lor \Trans_i\ $.
The transition relation of the program is $\relProg =
\relTIdOf{1} \lor \dots \lor \relTIdOf{\Num}\ $.




\paragraph{Computations}

A \emph{computation} of $\program$ is a sequence of program states
$s_1, s_2, \dots$ such that $s_1$ is an initial state, i.e., $s_1
\models \SymbInit\ $, and each pair of consecutive states $s_i$ and
$s_{i+1}$ in the sequence is connected by some transition $\rel$ from
a program thread, i.e., $(s_i, s_{i+1})\models \rel\ $.
A \emph{path} is a sequence of transitions.
We write $\emptypath$ for the \emph{empty} sequence.

Let $\substXtoY{\FreshVars}{\FreeVars}$ be a substitution function
such that $\varphi\substXtoY{\FreshVars}{\FreeVars}$ replaces
$\FreeVars$ by $\FreshVars$ in~$\varphi\ $.
Let $\comp$ be the \emph{relational composition} function for binary
relations given by assertions over unprimed and primed variables such
that for assertions $\varphi$ and $\psi$ we have
%
$
\varphi \comp \psi \defeq
\exists \Vars'' : \varphi\substXtoY{\Vars''}{\Vars'} \land \psi\substXtoY{\Vars''}{\Vars} \ .
$
%
Then, a \emph{path relation} is a relational composition of transition
relations along the path, i.e., for $\path = \rho_1 \cdots \rho_n$ we
have $\rho_\path = \rho_1 \comp \dots \comp \rho_n\ $. 
A path $\path$ is \emph{feasible} if its path relation is not empty,
i.e.,~$\exists \Vars, \Vars' : \rho_\path\ $. 

A program state is \emph{reachable} if it appears in some computation.
Let $\SymbReach$ denote the set of reachable states.
The program is \emph{safe} if \emph{none} of its error states is
reachable, i.e., $\SymbReach \land \SymbError \limplies \lfalse\ $.


\paragraph{Auxiliary definitions}

We define a \emph{successor} function $\post$ such that for a
binary relation over states $\rel$ and a set of states $\varphi$ we
have
%
\begin{equation*}
  \post(\rel, \varphi) \defeq 
  \exists \Vars'': \varphi\substXtoY{\Vars''}{\Vars} \land \rel\substXtoY{\Vars''}{\Vars}\substXtoY{\Vars}{\Vars'}\ .
\end{equation*}
%
We also extend the logical implication to tuples of equal length, i.e.,
%
\begin{equation*}
  (\varphi_1, \dots, \varphi_n) \limplies (\psi_1, \dots, \psi_n) \defeq
  \varphi_1 \limplies \phi_1 \land \dots \land
  \varphi_n \limplies \phi_n\ ,
\end{equation*}
%
where each implication is implicitly universally quantified over the
free variables occurring in it.
From now on, we assume that tuples of assertions are partially ordered
by the above extension of~$\limplies\ $.

A \emph{Horn clause} 
%
$%\begin{equation*}
  \HornBody_1(\FreeVars_1)\land \dots\land \HornBody_n(\FreeVars_n)
  \rightarrow \HornBody(\FreeVars)\ 
$%\end{equation*} 
% 
consists of relation symbols $\HornBody_1$,\dots, $\HornBody_n$,
$\HornBody$, and vectors of variables $\FreeVars_1$,\dots,
$\FreeVars_n$,~$\FreeVars$.
\iffalse
The left-hand side of the implication is called the \emph{body} of the
clause and right-hand side is called the \emph{head}.
\fi 
For the algorithm \algSolveLinearHornClauses in
Section~\ref{sec-horn-solving} we only consider Horn clauses over linear
arithmetic.
We say that $\HornBody$ \emph{depends} on the relation symbols
$\set{\HornBody_i \mid i\in\betweenOf{1}{n}\land \HornBody_i \neq
(\leq)}\ $.
A set of Horn clauses is \emph{recursion-free} if the transitive
closure of the corresponding dependency relation is well founded.


\iffalse
We reason about program safety by computing approximations of
reachable states and relations between reachable states. 
The following auxiliary notions provide fundamental building blocks
for the development of algorithms for program
verification~\cite{Floyd,lattice77,CousotAPCT84}.

The set of reachable program states is the least fixpoint of $\post$
applied on the program transition relation that is above the initial
program states, i.e.,
%
\begin{equation*}
  \SymbReach \lequivalent \lfp(\lambda
  \varphi.\post(
  \relProg, \varphi),
  \SymbInit)\ .
\end{equation*}
%

\todo{Move/drop this?}
Unfortunately, the precise least fixpoint computation is not practical
for programs with very large or infinite state spaces.
To obtain a practical solution, we can apply abstraction to achieve
approximate, yet sound results.
We formalize abstraction as over-approximation following the framework
of abstract interpretation~\cite{lattice77,CousotAPCT84}.
For this purpose, we will use abstraction functions $\abst$ and
$\abstTrans$ that over-approximate sets and binary relations over
programs states, respectively.

\fi


%%% Local Variables: 
%%% mode: latex
%%% TeX-master: "main"
%%% End: 
