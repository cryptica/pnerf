\section{Constructing Invariants from Constraints}
\label{sec_method_invariant}

\subsection{A first approach}

We make a simple observation which, combined with the flexibility of
SMT-solvers, leads to an interesting improvement on the Safety
Method. We propose method \invariant\ with the same power as the
method \safety\ over the rational numbers, but dual to it: the
constraints of method \safety\ are unfeasible over the rational
numbers if{}f the constraints of method \invariant\ are
feasible. The advantage is that the solution delivered by the
method \invariant\ is an inductive linear invariant satisfied by all
reachable markings, but violated by all markings that do not satisfy
the property. So this invariant provides an {\em explanation} of why
the property holds.

We now derive the constraints given by method \invariant.
Consider the matrix form of the negation of property $P$ below.
$$ \neg P = ( A M \ge b ) $$
for some matrix $A$ and vector $b$.
Consider the following primal system of linear inequations $C$.
\begin{align*}
  \mathcal{C}&(S, T, F, M_0)  && \text{constraints given by \safety} \\
  A M & \ge b                 && \text{negation of property $P$}
\end{align*}

We have the following theorem:

\begin{theorem}
\label{thm:dual}
The system $C'$ of constraints given by
\begin{align*}
  \lambda C    & \le 0     && \text{inductivity constraint} \\
  \lambda M_0  & <   Y_1 b && \text{safety constraint} \\
  \lambda      & \ge Y_1 A && \text{property constraint} \\
  Y_1          & \ge 0     && \text{non-negativity constraint} 
\end{align*}
is feasible over the rational numbers if{}f the system $C$ is unfeasible
over teh rational numbers.
\end{theorem}
\begin{proof}
See the Appendix.
\end{proof}

We can now easily interpret the meaning of a solution $\lambda$ of $C'$. Recall that
for every reachable marking $M$ we find a solution of 
$M = M_0 + CX$. Multiplying by $\lambda$, and since $\lambda$ is a solution
of $C'$, we get 
$$\lambda M = \lambda M_0 + \lambda CX \leq \lambda M_0$$
\noindent On the other hand, for every marking $M$ violating the property
we have $ AM \geq b$, and so
$$\lambda M \geq Y_1 AM \geq Y_1 b > \lambda M_0$$

Therefore, the linear constraint
$$I(M) = \lambda M \leq \lambda M_0$$
is an invariant satisfied by all reachable markings $M$, 
but by no marking violating the property.

We can now prove that $I(M)$ is an inductive invariant, that is, if
$I(M)$ holds for some marking $M$ (reachable or not), and $M \xrightarrow{t} M'$
for some transition $t$, then $I(M')$ holds as well.
Indeed, in this case we have $M' = M + C e_t$ where $e_t = (0,\ldots,0,\underbrace{1}_{t},0,\ldots,0)^T$ is the vector whose components are all $0$ but the one
corresponding to $t$, which has value $1$. So we get
$$\lambda M' = \lambda(M + Ce_t) = \lambda M + \lambda Ce_t
\le \lambda M \le \lambda M_0$$
\noindent and so $I(M')$ holds.

\subsection{Example}

Again, we consider Lamport's algorithm and the mutual
exclusion property.  As in this case, we can not prove safety 
without using trap constraints, we can also
not obtain an invariant by simply applying method \invariant.

However, let us assume we wanted to prove that there is no marking
violating the mutual exclusion property and also marking each trap
found by method \safetyref. We can do this by adding
the trap constraints to the negation of the property, which yields
$$(p_3 \ge 1) \land (q_5 \ge 1) \land
(p_2 + q_2 + q_3 + notbit_1 + notbit_2 \ge 1)$$

The constraints constructed by the method \invariant\ then are

\noindent \emph{Inductivity constraints}
$$
\begin{array}[t]{clclclclclcl}
    -  & p_1 &  +  & p_2 &     &     &  +  & bit_1 &  -  & notbit_1 &  \le  & 0 \\
       &     &  -  & p_2 &  +  & p_3 &     &       &     &          &  \le  & 0 \\
       & p_1 &     &     &  -  & p_3 &  -  & bit_1 &  +  & notbit_1 &  \le  & 0
\end{array}\quad
\begin{array}[t]{clclclclclclclclclcl}
    -  & q_1 &  +  & q_2 &     &     &     &     &     &     &  -  & notbit_2 &  \le  & 0  \\
       &     &  -  & q_2 &  +  & q_3 &     &     &     &     &     &          &  \le  & 0  \\
       &     &     &     &  -  & q_3 &  +  & q_4 &     &     &  +  & notbit_2 &  \le  & 0  \\
       & q_1 &     &     &     &     &  -  & q_4 &     &     &     &          &  \le  & 0  \\
       &     &  -  & q_2 &     &     &     &     &  +  & q_5 &     &          &  \le  & 0  \\
       & q_1 &     &     &     &     &     &     &  -  & q_5 &  +  & notbit_2 &  \le  & 0 
\end{array}
$$
\emph{Safety constraint}
$$    
p_1 + q_1 + \neg bit_1 + \neg bit_2 < target_1 + target_2 + trap_1
$$
\emph{Property constraints}
$$
\begin{array}[t]{rcl}
  p_1   & \ge & 0 \\
  p_2   & \ge & trap_1 \\
  p_3   & \ge & target_1 \\
  bit_1 & \ge & 0 \\
  notbit_1 & \ge 0
\end{array}
\qquad
\begin{array}[t]{rcl}
  bit_1 & \ge & 0 \\
  notbit_1 & \ge & trap_1
\end{array}
\qquad
\begin{array}[t]{rcl}
  notbit_2 & \ge & trap_1
\end{array}
\qquad
\begin{array}[t]{rcl}
  q_1   & \ge & 0 \\
  q_2   & \ge & trap_1 \\
  q_3   & \ge & trap_1 \\
  q_4   & \ge & 0 \\
  q_5   & \ge & target_2 \\
\end{array}
$$
\emph{Non-negativity constraints}
$$
\begin{array}[t]{rcl}
  target_1 & \ge & 0 \\
  target_2 & \ge & 0 \\
  trap_1   & \ge & 0
\end{array}
$$

\noindent where $\lambda$ corresponds to the places and $Y_1 = (target_1\ target_2\ trap_1)$
corresponds to the two parts of the property and the added trap.

For the construction of an invariant, we consider the
following solution
$$
\begin{array}[t]{rcl}
  p_1      & = & 1 \\
  p_2      & = & 2 \\
  p_3      & = & 2 \\
  bit_1    & = & 0 \\
  notbit_1 & = & 0
\end{array}
\qquad
\begin{array}[t]{rcl}
  bit_1    & = & 0 \\
  notbit_1 & = & 1
\end{array}
\qquad
\begin{array}[t]{rcl}
  notbit_2 & = & 1
\end{array}
\qquad
\begin{array}[t]{rcl}
  q_1   & = & 0 \\
  q_2   & = & 1 \\
  q_3   & = & 1 \\
  q_4   & = & 0 \\
  q_5   & = & 1 \\
\end{array}
\begin{array}[t]{rcl}
  target_1 & = & 2 \\
  target_2 & = & 1 \\
  trap_1   & = & 1
\end{array}
$$

Given the solution, we construct the following
invariant \invI.
$$
I(M) = (p_1 + 2 p_2 + 2 p_3 + notbit_1 + notbit_2 + q_2 + q_3 + q_5 \le 3)
$$

%%% Local Variables: 
%%% mode: latex
%%% TeX-master: "main"
%%% End: 
