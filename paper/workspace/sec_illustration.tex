\section{An Example: Lamport's 1-bit Algorithm}

We illustrate three constraint-based functions for
invariant construction on Lamport's 1-bit algorithm for mutual
exclusion\cite{LamportJACM86}.
Figure~\ref{fig_lamport_src} shows the instance of the algorithm 
for two processes. Figure~\ref{fig_lamport_net} 
shows the corresponding Petri net.

\begin{figure}[t]
  \hrule
  \centering
  \begin{minipage}[t]{.04\columnwidth}
   \mbox{} \\ \\ \\ \\ \\ $p_1$: \\ $p_2$: \\ $p_3$:
  \end{minipage}
  \begin{minipage}[t]{.44\columnwidth}
    \mbox{}\\
    \algProcedure \textsc{Process 1} \\
    \algBegin \\
    \tabT $bit_1$ \algAssgn \mathfalse \\
    \tabT \algWhile \mathtrue\ \algDo \\
    \tabTT $bit_1$ \algAssgn \mathtrue \\
    \tabTT \algWhile $bit_2$ \algDo \algSkip \algOd \\
    \tabTT \algComment{critical section} \\
    \tabTT $bit_1$ \algAssgn \mathfalse \\
    \tabT \algOd \\
    \algEnd
  \end{minipage}
  \begin{minipage}[t]{.04\columnwidth}
   \mbox{} \\ \\ \\ \\ \\ $q_1$: \\ $q_2$: \\ $q_3$: \\
   $q_4$: \\ \\ \\ $q_5$:
  \end{minipage}
  \begin{minipage}[t]{.44\columnwidth}
    \mbox{}\\
    \algProcedure \textsc{Process 2} \\
    \algBegin \\
    \tabT $bit_2$ \algAssgn \mathfalse \\
    \tabT \algWhile \mathtrue\ \algDo \\
    \tabTT $bit_2$ \algAssgn \mathtrue \\
    \tabTT \algIf $bit_1$ \algThen \\
    \tabTTT $bit_2$ \algAssgn \mathfalse \\
    \tabTTT \algWhile $bit_1$ \algDo \algSkip \algOd \\
    \tabTTT \algGoto $q_1$ \\
    \tabTT \algFi \\
    \tabTT \algComment{critical section} \\
    \tabTT $bit_2$ \algAssgn \mathfalse \\
    \tabT \algOd \\
    \algEnd
  \end{minipage}
  \vspace{1.5ex}
  \hrule
  \caption{Lamport's 1-bit algorithm for mutual exclusion
    \cite{LamportJACM86}.}
  \label{fig_lamport_src}
\end{figure}

%%% Local Variables: 
%%% mode: latex
%%% TeX-master: "main"
%%% End: 


\subsection{An Invariant for Process 1}

We consider the property that the program counter of process 1 is
always either at location $p_1$, $p_2$, or $p_3$, i.e. $p_1 + p_2 +
p_3 \leq 1$. For the given network and property we construct an
invariant in the following way.

We obtain the following set of state constraints \setCp\ corresponding
to the initial marking, flow relation, and the property.
The set \setCp\ is such that a solution to the set corresponds to an
invariant.
\\ \mbox{ } \\
\begin{minipage}{\columnwidth}
  \emph{Inductivity constraints} \\[-2em]
  \begin{minipage}[t]{.49\columnwidth}
    \centering
    \begin{alignat*}{25}
      &{}-{}& p_1 &{}+{}& p_2 &     &     &{}+{}& bit_1 &{}-{}& notbit_1 &{}\le{}& 0 \\[-0.4em]
      &     &     &{}-{}& p_2 &{}+{}& p_3 &     &       &     &          &{}\le{}& 0 \\[-0.4em]
      &     & p_1 &     &     &{}-{}& p_3 &{}-{}& bit_1 &{}+{}& notbit_1 &{}\le{}& 0
    \end{alignat*}
  \end{minipage}
  \begin{minipage}[t]{.49\columnwidth}
    \centering
    \begin{alignat*}{25}
      &{}-{}& q_1 &{}+{}& q_2 &     &     &     &     &     &     &{}-{}& notbit_2 &{}\le{}& 0  \\[-0.4em]
      &     &     &{}-{}& q_2 &{}+{}& q_3 &     &     &     &     &     &          &{}\le{}& 0  \\[-0.4em]
      &     &     &     &     &{}-{}& q_3 &{}+{}& q_4 &     &     &{}+{}& notbit_2 &{}\le{}& 0  \\[-0.4em]
      &     & q_1 &     &     &     &     &{}-{}& q_4 &     &     &     &          &{}\le{}& 0  \\[-0.4em]
      &     &     &{}-{}& q_2 &     &     &     &     &{}+{}& q_5 &     &          &{}\le{}& 0  \\[-0.4em]
      &     & q_1 &     &     &     &     &     &     &{}-{}& q_5 &{}+{}& notbit_2 &{}\le{}& 0 
    \end{alignat*}
  \end{minipage} \\
  \emph{Safety constraint}\\[-2em]
  \begin{align*}
    p_1 + q_1 + \neg bit_1 + \neg bit_2 & < 2 * target_1
  \end{align*}
\end{minipage}
\begin{minipage}{\columnwidth}
  \emph{Property constraints} \\[-2em]
  \begin{minipage}[t]{.49\columnwidth}
    \centering
    \begin{align*}
      p_1   & \ge target_1 \\[-0.4em]
      p_2   & \ge target_1 \\[-0.4em]
      p_3   & \ge target_1 \\[-0.4em]
      bit_1 & \ge 0        \\[-0.4em]
      notbit_1 & \ge 0
    \end{align*}
  \end{minipage}
  \begin{minipage}[t]{.49\columnwidth}
    \centering
    \begin{align*}
      q_1   & \ge 0 \\[-0.4em]
      q_2   & \ge 0 \\[-0.4em]
      q_3   & \ge 0 \\[-0.4em]
      q_4   & \ge 0 \\[-0.4em]
      q_5   & \ge 0 \\[-0.4em]
      notbit_2 & \ge 0
    \end{align*}
  \end{minipage}
\end{minipage}
\begin{minipage}{\columnwidth}
  \emph{Non-negativity constraint}\\[-2em]
  \begin{align*}
    target_1 & \ge 0
  \end{align*}
\end{minipage}
\\ \mbox{ } \\
The \emph{inductivity constraints} correspond to the flow relation of the
Petri~net.
The \emph{safety constraint} corresponds to the initial marking and
the property.
The \emph{property constraint} corresponds to the property and the set
of places.
The \emph{non-negativity constraints} corresponds to the property.

For the construction of an invariant, we consider the
following solution \solA\ to the set \setCp.
\\[-2em]
\begin{minipage}[t]{.49\columnwidth}
  \begin{align*}
    p_1      & = 3 \\[-0.4em]
    p_2      & = 2 \\[-0.4em]
    p_3      & = 2 \\[-0.4em]
    bit_1    & = 1 \\[-0.4em]
    notbit_1 & = 0 \\[-0.4em]
  \end{align*}
\end{minipage}
\begin{minipage}[t]{.49\columnwidth}
  \begin{align*}
    q_1      & = 0 \\[-0.4em]
    q_2      & = 0 \\[-0.4em]
    q_3      & = 0 \\[-0.4em]
    q_4      & = 0 \\[-0.4em]
    q_5      & = 0 \\[-0.4em]
    notbit_2 & = 0 \\[-0.4em]
    target_1 & = 2
  \end{align*}
\end{minipage}

Given the solution, we construct the following
invariant \invI.
\begin{align*}
  3 p_1 + 2 p_2 + 2 p_3 + bit_1 & \le 3
\end{align*}
The invariant corresponds to the property that the program
counter of process 1 is always either at location $p_1$, $p_2$, or
$p_3$ and when the counter is at $p_1$ the flag $bit_1$ is not set.
The left-hand-side is a linear combination of the places of the
Petri~net where the coefficients are given by the solution.
The right hand side is the value of the combination at the initial
marking.

\begin{figure}
\begin{center}
  \begin{tikzpicture}
  \draw[box] (-2,8.5) rectangle (1,0);
  \draw[box] (3,8.5) rectangle (7.5,0);
  
  \node at (-0.5, 0.5) {First Process};
  \node at (5, 0.5) {Second Process};

  \node[place, label=left:p3] (p3) at (0,7) {};
  \node[transition, label=left:s3] (s3) at (0,6) {};
  \node[place, tokens=1, label=left:p1] (p1) at (0,5) {};
  \node[transition, label=left:s1] (s1) at (0,4) {};
  \node[place, label=left:p2] (p2) at (0,3) {};
  \node[transition, label=left:s2] (s2) at (0,2) {};

  \draw (p3) edge[flow] (s3);
  \draw (s3) edge[flow] (p1);
  \draw (p1) edge[flow] (s1);
  \draw (s1) edge[flow] (p2);
  \draw (p2) edge[flow] (s2);
  \draw[rounded corners=4mm] (s2) |- (-1.5,1) -- (-1.5,8) [flow] -| (p3);
 
  \node[transition, label=above:t2] (t2) at (4,7) {};
  \node[place, label=right:q3] (q3) at (4,6) {};
  \node[transition, label=right:t3] (t3) at (4,5) {};
  \node[place, label=right:q4] (q4) at (4,4) {};
  \node[transition, label=left:t4] (t4) at (4,3) {};

  \node[place, label=right:q2] (q2) at (5.5,7) {};
  \node[transition, label=right:t5] (t5) at (5.5,6) {};
  \node[place, label=right:q5] (q5) at (5.5,5) {};
  \node[transition, label=right:t6] (t6) at (5.5,4) {};
  \node[place, tokens=1, label=right:q1] (q1) at (5.5,3) {};
  \node[transition, label=right:t1] (t1) at (5.5,2) {};
  
  \draw (q2) edge[flow] (t5);
  \draw (t5) edge[flow] (q5);
  \draw (q5) edge[flow] (t6);
  \draw (t6) edge[flow] (q1);
  \draw (q1) edge[flow] (t1);
  \draw[rounded corners=4mm] (t1) |- (7,1) -- (7,8) [flow] -| (q2);
  
  \draw (q2) edge[flow] (t2);
  \draw (t2) edge[flow] (q3);
  \draw (q3) edge[flow] (t3);
  \draw (t3) edge[flow] (q4);
  \draw (q4) edge[flow] (t4);
  \draw (t4) edge[flow] (q1);
  
  \node[place, label=above:bit1] (bit1) at (2,6) {};
  \node[place, tokens=1, label=below:$\neg$bit1] (nbit1) at (2,4) {};
  \node[place, tokens=1, label=below:$\neg$bit2] (nbit2) at (2,2) {};
  
  \draw (bit1) edge[flow] (s3);
  \draw (s3) edge[flow] (nbit1);
  \draw (nbit1) edge[flow] (s1);
  \draw (s1) edge[flow] (bit1);
  \draw (t2) edge[flow, <->] (bit1);
  \draw (nbit1) edge[flow, <->, bend left=18] (t5);
  \draw (nbit1) edge[flow, <->] (t4);
  \draw (nbit2) edge[flow, <->] (s2);
  \draw (t3) edge[flow] (nbit2);
  \draw (t6) edge[flow, bend left=30] ([xshift=-0.2mm, yshift=1mm] nbit2.east);
  \draw (nbit2) edge[flow] (t1.west);
\end{tikzpicture}
\end{center}
\caption{Petri net for Lamport's 1-bit algorithm}
\label{fig:lamport-petri-net}
\end{figure}


\subsection{Example: minimal invariant for process 1}


\subsection{Example: invariant by refinement for process 1}



%%% Local Variables: 
%%% mode: latex
%%% TeX-master: "main"
%%% End: 
