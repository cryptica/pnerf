\subsection{Minimizing Invariants}
\label{sec_invariant_w_minimization}

There may be multiple solutions to the system $\mathcal{S}'$
leading to different invariants. 
We now propose a method \invariantwmin\ to obtain a minimal
invariant, where we define the size of an invariant as the number of 
place variables appearing in it.

Given an assignment $\lambda$ to the dual state constraints,
we add further constraints whose solution yields a
smaller invariant, if one exists. 
We iterate this process to obtain the minimal invariant.

To minimize a given invariant, we define a binary vector $b_\lambda$ which projects the
non-zero components of $\lambda$ to 1 and the other components to 0.
The sum of this vector is then equal to the number of places appearing
in the invariant.
We add constraints of the following form for each place $p$:
\begin{align*}
p &> 0 \implies b_{p} = 1 && \text{$p$ appears in the invariant} \\
p &= 0 \implies b_{p} = 0 && \text{$p$ does not appear in the invariant}
\end{align*}
Finally, we add the constraint $\sum_{p\in P} b_p < K$, where $K$ is the number
of places appearing in the current invariant.
For the invariant~\eqref{eq:invariant} for mutual exclusion, this constraint is
$$
    b_{p_1} + b_{p_2} + b_{p_3} + b_{bit_1} + b_{notbit_1} + b_{notbit_2} +
    b_{q_1} + b_{q_2} + b_{q_3} + b_{q_4} + b_{q_5} < 8\,.
$$
With the addded constraints, we obtain a new solution for our 
example, setting $p_2$, $p_3$, $notbit_1$, $notbit_2$, $q_2$, $q_3$,
$target_1$, $target_2$, and $trap_1$ to $1$, and all other variables
to $0$. The invariant for this solution is
$$
I(M)\ ::\  p_2 + p_3 + notbit_1 + notbit_2 + q_2 + q_3 + q_5 \le 2\,.
$$
which is the minimal invariant. Together with the trap constraint we 
finally get:
\begin{align*}
I'(M)\ ::&\ 
(p_2 + q_2 + notbit_1 + notbit_2 + q_3 \ge 1 ) \; \land \\
&\ (p_2 + p_3 + notbit_1 + notbit_2 + q_2 + q_3 + q_5 \le 2)\,.
\end{align*}


%%%%%%%%%%%%%%%%%%%%%%%%%%%%%%%%%%%%%%%%%%%%
%%%%%%%%I think this remark can be removed
%%%%%%%%%%%%%%%%%%%%%%%%%%%%%%%%%%%%%%%%%%%%
%Note that with the added trap constraints, the invariant might not be
%minimal as we are not guaranteed to find traps of minimal size
%or a minimal number of traps. However, we might still be able to
%reduce the size of the invariant.

% \begin{figure}
  \begin{algorithmic}[1]
    \State $C' := \mathcal C'(N, \neg P)$
    \If {SAT$(C')$}
    \State $A' := $Model$(C')$
    \State $C_M := $MinConstraints$(N, A')$
    \While {SAT$(C' \cup C_M)$}
    \State $A' := $Model$(C' \cup C_M)$
    \State $C_M := $MinConstraints$(N, A')$
    \EndWhile
    \State $I := $Inv$(N, A')$
    \State \Return 'Invariant I for the petri net: $I$'
    \Else
    \State \Return 'Failed at finding an invariant'
    \EndIf
  \end{algorithmic}
  \caption{Pseudocode for Method Invariant with Minimization}
  \label{fig:method-invariant-with-minimization-pseudocode}
\end{figure}



\iffalse

\begin{verbatim}
* Property of minimization constraints C_M generated from A': 
    If A'' satisfies C' u C_M, then Inv(N, A'') uses less places than Inv(N, A')
\end{verbatim}

\newpage

\begin{verbatim}
* Subprocedures \mathcal{C'} and Inv are the same as in Method Invariant

* Subprocedure \text{MinConstraints}:

  Input:
    N = (S, T, E, M0)  : Petri net
    A'                 : Satisfying assignment for C'
  Output:
    C_M                : Minimization Constraints

  Pseudocode:

\end{verbatim}

\begin{align*}
  \text{MinConstraints}(N, A') =& \left( \bigwedge_{s \in S} \left(
      (s > 0 \Rightarrow b_s = 1) \land (s = 0 \Rightarrow b_s = 0)
    \right) \right) \land
    \left( \sum_{s \in S} b_s < \sum_{s \in S : A'(s) > 0} 1 \right)
\end{align*}

\iffalse
\begin{verbatim}
* Example:

  - Dual state constraints C' and satisfying assignment A' for C' as in example for method Invariant with property 2

  - Minimization constraints C_M:
\end{verbatim}

\begin{verbatim}
    p1     > 0 => b_p1   = 1
    |            -----------
    |               |
    |            place p1 appears in invariant
    |
    place p1 coefficient in the invariant

    p1     = 0 => b_p1   = 0
    |           ------------
    |               |
    |            place p1 does not appear in invariant
    |
    place p1 coefficient in the invariant
\end{verbatim}

\begin{align*}
(     p_1    > 0 & \implies b_{p_1}   = 1 ) & &\land &
(     p_1    = 0 & \implies b_{p_1}    = 0 ) \\
(     p_2    > 0 & \implies b_{p_2}    = 1 ) & &\land &
(     p_2    = 0 & \implies b_{p_2}    = 0 ) \\
(     p_3    > 0 & \implies b_{p_3}    = 1 ) & &\land &
(     p_3    = 0 & \implies b_{p_3}    = 0 ) \\
(     bit_1  > 0 & \implies b_{bit_1}  = 1 ) & &\land &
(     bit_1  = 0 & \implies b_{bit_1}  = 0 ) \\
( notbit_1 > 0 & \implies b_{notbit_1} = 1 ) & &\land &
( notbit_1 = 0 & \implies b_{notbit_1} = 0 ) \\
(     q_1    > 0 & \implies b_{q_1}    = 1 ) & &\land &
(     q_1    = 0 & \implies b_{q_1}    = 0 ) \\
(     q_2    > 0 & \implies b_{q_2}    = 1 ) & &\land &
(     q_2    = 0 & \implies b_{q_2}    = 0 ) \\
(     q_3    > 0 & \implies b_{q_3}    = 1 ) & &\land &
(     q_3    = 0 & \implies b_{q_3}    = 0 ) \\
(     q_4    > 0 & \implies b_{q_4}    = 1 ) & &\land &
(     q_4    = 0 & \implies b_{q_4}    = 0 ) \\
(     q_5    > 0 & \implies b_{q_5}    = 1 ) & &\land &
(     q_5    = 0 & \implies b_{q_5}    = 0 ) \\
( notbit_2 > 0 & \implies b_{notbit_2} = 1 ) & &\land &
( notbit_2 = 0 & \implies b_{notbit_2} = 0 )
\end{align*}

\begin{align*}
    b_{p_1} + b_{p_2} + b_{p_3} + b_{bit_1} + b_{notbit_1} +
    b_{q_1} + b_{q_2} + b_{q_3} + b_{q_4} + b_{q_5} + b_{notbit_2} & < 4
\end{align*}

\iffalse
\begin{verbatim}
    b_p1 + b_p2 + b_p3 + b_bit1 + b_nbit1 + b_q1 + b_q2 + b_q3 + b_q4 + b_q5
         + b_nbit2 < 4
    --------------   |
                |    |
                |    number of places appearing in current invariant for A'
                |    = #{s | A'(s) > 0} = #{p1, p2, p3, bit1}
                |
  number of places appearing in new invariant
\end{verbatim}
\fi
  
\begin{verbatim}
  - Model A' for C' \cup C_M:
\end{verbatim}

\begin{align*}
     p_1      & = 1 &    b_{p_1}     & = 1 \\
     p_2      & = 1 &    b_{p_2}     & = 1 \\
     p_3      & = 1 &    b_{p_3}     & = 1 \\
     bit_1    & = 0 &    b_{bit_1}   & = 0 \\
notbit_1    & = 0 &    b_{notbit_1}  & = 0 \\
     q_1      & = 0 &    b_{q_1}     & = 0 \\
     q_2      & = 0 &    b_{q_2}     & = 0 \\
     q_3      & = 0 &    b_{q_3}     & = 0 \\
     q_4      & = 0 &    b_{q_4}     & = 0 \\
     q_5      & = 0 &    b_{q_5}     & = 0 \\
notbit_2    & = 0 &    b_{notbit_2}  & = 0 \\
     target_1 & = 1 \\
\end{align*}
    
\begin{verbatim}
  - Minimized Invariant:
\end{verbatim}

\begin{align*}
    p_1 + p_2 + p_3 & \le 1
\end{align*}
\fi

\fi

%%% Local Variables: 
%%% mode: latex
%%% TeX-master: "main"
%%% End: 
