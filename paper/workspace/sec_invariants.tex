\section{Method Invariant}

The method Invariant constructs an invariant \verb=I= for given Petri
net \verb=N= and \verb=P= when \verb=N= never violates \verb=P=.

\begin{verbatim}
* Code and Petri net: same as in section Method Safety.
\end{verbatim}

\begin{verbatim}
* Method Invariant

  Subprocedure \mathcal{C'} constructs dual state constraints C' corresponding to N and P.
  Subprocedure Model returns assignment A such that A satisfies C'.
  Subprocedure Inv constructs invariant I corresponding to N and A'.
\end{verbatim}

\ifthenelse{\equal{\isDraft}{true}}{\begin{figure}
  \begin{center}
    \begin{tikzpicture}[
      every path/.style={draw, ->, >=stealth, shorten >=2pt, shorten <=2pt}
      ]
      \node[state] (begin) {BEGIN};
      \node[action, below=of begin] (c) {$C':=\mathcal C'(N, \neg P)$};
      \node[decision, below=of c] (satc) {$\text{SAT}(C')$};
      \node[print, right=of satc] (noinv) {No Invariant};
      \node[action, below=of satc] (inv) {$A':=\text{Model}(C')$\\
        $I := \text{Inv}(N, A')$};
      \node[print, below=of inv] (printinv) {Invariant: $I$};
      \node[state, right=of noinv] (end1) {END};
      \node[state, below=of printinv] (end2) {END};
      
      \draw (begin) edge (c);
      \draw (c) edge (satc);
      \draw (satc) edge node[above]{NO} (noinv);
      \draw (satc) edge node[right]{YES} (inv);
      \draw (noinv) edge (end1);
      \draw (inv) edge (printinv);
      \draw (printinv) edge (end2);
    \end{tikzpicture}
  \end{center}
  \caption{Diagram for function Invariant}
  
\label{fig_function_invariant_diagram}
\end{figure}

%%% Local Variables: 
%%% mode: latex
%%% TeX-master: "main"
%%% End: 
}{}

\begin{figure}
  \hrule
  \centering
  \begin{minipage}[t]{.04\columnwidth}
   \mbox{} \\ \\ \\ \\ \\ \\ \\ \\ \\ \\
   \raggedleft 1 \\ 2 \\ 3 \\ 4 \\ 5 \\ 6 \\ 7 \\ 8 %\\ 9 \\ 10 \\
%   11 \\ 12 \\ 13 \\ 14 \\ 15 \\ 16 \\ 17 \\ 18 \\ 19 % \\ 20 \\
%   21 \\ 22 \\ 23 \\ 24 \\ 25 \\ 26 \\ 27 \\ 28 \\ 29 % \\ 30 \\ 
%   31 \\ 32 \\ 33 \\ 34 \\ 35 \\ 36
  \end{minipage}
  \begin{minipage}[t]{.94\columnwidth}
    \mbox{}\\
    \algMethod \textsc{Invariant} \\
    \algInput\\
    \tabT $N$ - Petri net \\
    \tabT $P$ - property \\
    \algVars\\
    \tabT $C'$ - dual state constraints \\
    \tabT $A'$ - satisfying assignment for the dual state constraints \\
    \tabT $I$ - invariant \\
    \algBegin\\
    \tabT $C'$ \algAssgn $\mathcal C'(N, \neg P)$ \\
    \tabT \algIf SAT$(C')$ \algThen \\
    \tabTT $A'$ \algAssgn Model$(C')$ \\
    \tabTT $I$ \algAssgn Inv$(N, A')$ \\
    \tabTT \algReturn ``Invariant I for the petri net: $I$'' \\
    \tabT \algElse \\
    \tabTT \algReturn ``Failed at finding an invariant'' \\
    \tabT \algFi \\
    \algEnd
  \end{minipage}
  \vspace{1.5ex}
  \hrule
  \caption{Method Invariant}
  \label{fig_method_invariant}
\end{figure}

%%% Local Variables: 
%%% mode: latex
%%% TeX-master: "main"
%%% End: 


\begin{verbatim}
* Property of dual state constraints C':  If C' is sat then N |= P.

* Property of invariant I:
  - I is reachable: For each reachable marking M, I(M) is valid
  - I is safe:      For markings that violate the property, I(M) is unsat
  - I is inductive: For each marking M, if I(M) is valid and M -> M1 then I(M1) is valid
\end{verbatim}

\newpage

\begin{verbatim}
* Subprocedure \mathcal{C'}:

  Input:
    N = (S, T, E, M0)  : Petri net
    \neg P = (p_1,1 + ... + p_1,m_1 >= b_1 /\ p_2,1 + ... + p_2,m_2 >= b_2 /\
              ... /\ p_n,1 + ... + p_n,m_n >= b_n ) : Negated property
  Output:
    C'            : Dual state constraints

Pseudocode:
\end{verbatim}

\begin{align*}
  C'(N, \neg P) =& \left( \bigwedge_{t \in T} \left( 0 \ge
                      \sum_{(t, s) \in E} s
                    - \sum_{(s, t) \in E} s \right) \right) \land
     \left( \sum_{s \in S} M_0(s) \cdot s <
       \sum_{i=1}^n b_i \cdot target_i \right) \land \\
     & \left( \bigwedge_{s \in S} \left ( s \ge 
       \sum_{i : s \in \{ p_{i,1}, \ldots, p_{i,m_i} \} } target_i \right) \right) \land
     \left( \bigwedge_{i=1}^n \left( target_i \ge 0 \right) \right)
\end{align*}

\begin{verbatim}
* Subprocedure Inv:

  Input:
    N = (S, T, E, M0) : Petri net
    A'                : Satisfying assignment for C'

  Output:
    I                 : Invariant

  Pseudocode:
\end{verbatim}

\begin{align*}
  I(N, A') =& \left( \sum_{s \in S} A'(s) \cdot s \le
                     \sum_{s \in S} A'(s) \cdot M_0(s) \right)
\end{align*}

\newpage

\begin{verbatim}
* Derivation of invariant I for Petri net N that satisfies property P:

  M                                    : marking          ~ places
  C                                    : incidence matrix ~ rows correspond to places, columns correspond to transitions, relates places to transitions
  X                                    : firing vector    ~ transitions

  The following constraints C1 are unsat.
  M = M0 + C*X                         : place equations
  M >= 0                               : non-negativity conditions for places
  X >= 0                               : non-negativity conditions for transitions
  AM >= b                              : property P negated

  Substitute M to obtain constraints C2.
  A(M0 + CX) >= b                      : property P negated
  M0 + CX >= 0                         : non-negativity conditions for places
  X >= 0                               : non-negativity conditions for transitions

  Rewrite each system to obtain constraints C3.
  (-A*C)      (A*M0-b)                 : property P negated
  (  -C)*X <= (  M0  )                 : non-negativity conditions for places
  (  -I)      (   0  )                 : non-negativity conditions for transitions

  Apply Farkas' Lemma to obtain constraints C4.
  yT*(-A*C)                            : 
     (  -C) = 0                        : 
     (  -I)                            : 

  yT*(A*M0-b)                          : 
     (  M0  ) < 0                      : 
     (   0  )                          : 

  y >= 0                               : 

  The constraints C4 are sat iff the following constraints C5 are sat.
  y1 * A * C + y2 * C + y3 = 0         : 
  y1 * (A*M - b) + y2 * M0 < 0         : 
  y1 >= 0                              : 
  y2 >= 0                              : 
  y3 >= 0                              : 

  The constraints C5 are sat iff the following constraints C6 are sat.
  (y1 * A + y2) * C  <= 0              : 
  (y1 * A + y2) * M0 < y1 * b          : 
  y1 >= 0                              : 
  y2 >= 0                              : 

  The constraints C6 are sat iff the following constraints C' are sat.
  \lambda * C  <= 0                    : inductivity constraint
  \lambda * M0 < y1 * b                : safety constraint
  \lambda >= y1 * A                    : property constraint
  y1 >= 0                              : non-negativity constraint

  For \lambda satisfying C' the invariant is the following.
  I(M) = (\lambda * M <= \lambda * M0) : invariant
\end{verbatim}

\newpage

\begin{verbatim}
* Example:
  
  - Using property 2

  - Dual state constraints C':

    0 >= - p1 + p2      + bit1 - nbit1
    0 >=      - p2 + p3
    0 >= + p1      - p3 - bit1 + nbit1

    0 >= - q1 + q2                - nbit2
    0 >=      - q2 + q3
    0 >=           - q3 + q4      + nbit2
    0 >= + q1           - q4
    0 >=      - q2           + q5
    0 >= + q1                - q5 + nbit2

    p1 + q1 + nbit1 + nbit2 < 2 * target_1

    p1    >= target_1
    p2    >= target_1
    p3    >= target_1
    bit1  >= 0
    nbit1 >= 0
    q1    >= 0
    q2    >= 0
    q3    >= 0
    q4    >= 0
    q5    >= 0
    nbit2 >= 0

    target_1 >= 0

  - Model A':

    p1       = 3
    p2       = 2
    p3       = 2
    bit1     = 1
    nbit1    = 0
    q1       = 0
    q2       = 0
    q3       = 0
    q4       = 0
    q5       = 0
    nbit2    = 0
    target_1 = 2
    
  - Invariant:

    3 * p1 + 2 * p2 + 2 * p3 + bit1 <= 3

\end{verbatim}

\newpage

\section{Method Invariant with Minimization}

The method Invariant with Minimization constructs an invariant \verb=I=
that uses a minimal number of places for given Petri net \verb=N= and
\verb=P= when \verb=N= never violates \verb=P=. 

\begin{verbatim}
* Code, property, and Petri net: same as in section Method Safety.

* Method Invariant with Minimization

  Subprocedure \text{MinConstraints} constructs minimization constraints C_M corresponding to N and A'
\end{verbatim}

\ifthenelse{\equal{\isDraft}{true}}{\begin{figure}
  \begin{center}
    \begin{tikzpicture}[
      every path/.style={draw, ->, >=stealth, shorten >=2pt, shorten <=2pt}
      ]
      \node[state] (begin) {BEGIN};
      \node[action, below=of begin] (c) {$C':=\mathcal C'(N, \neg P)$};
      \node[decision, below=of c] (satc) {$\text{SAT}(C')$};
      \node[print, right=of satc] (noinv) {No Invariant};
      \node[action, below=of satc] (model1) {$A':=\text{Model}(C')$};
      \node[action, below=of model1] (cm) {$C_M:=\text{MinConstraints}(N, A')$};
      \node[decision, below=of cm] (satcm) {$\text{SAT}(C' \cup C_M)$};
      \node[action, right=of satcm] (model2) {$A':=\text{Model}(C' \cup C_M)$};
      \node[action, below=of satcm] (inv) {$I := \text{Inv}(N, A')$};
      \node[print, below=of inv] (printinv) {Invariant: $I$};
      \node[state, right=of noinv] (end1) {END};
      \node[state, below=of printinv] (end2) {END};
      
      \draw (begin) edge (c);
      \draw (c) edge (satc);
      \draw (satc) edge node[above]{NO} (noinv);
      \draw (noinv) edge (end1);
      \draw (satc) edge node[right]{YES} (model1);
      \draw (model1) edge coordinate[pos=.5] (edgein) (cm);
      \draw (cm) edge (satcm);
      \draw (satcm) edge node[above]{YES} (model2);
      \draw (model2) |- (edgein);
      \draw (satcm) edge node[right]{NO} (inv);
      \draw (inv) edge (printinv);
      \draw (printinv) edge (end2);
    \end{tikzpicture}
  \end{center}
  \caption{Diagram for Method Invariant with Minimization}
  \label{fig:method-invariant-with-minimization-diagram}
\end{figure}
}{}

\begin{figure}
  \begin{algorithmic}[1]
    \State $C' := \mathcal C'(N, \neg P)$
    \If {SAT$(C')$}
    \State $A' := $Model$(C')$
    \State $C_M := $MinConstraints$(N, A')$
    \While {SAT$(C' \cup C_M)$}
    \State $A' := $Model$(C' \cup C_M)$
    \State $C_M := $MinConstraints$(N, A')$
    \EndWhile
    \State $I := $Inv$(N, A')$
    \State \Return 'Invariant I for the petri net: $I$'
    \Else
    \State \Return 'Failed at finding an invariant'
    \EndIf
  \end{algorithmic}
  \caption{Pseudocode for Method Invariant with Minimization}
  \label{fig:method-invariant-with-minimization-pseudocode}
\end{figure}



\begin{verbatim}
* Property of minimization constraints C_M generated from A': 
    If A'' satisfies C' u C_M, then Inv(N, A'') uses less places than Inv(N, A')
\end{verbatim}

\newpage

\begin{verbatim}
* Subprocedures \mathcal{C'} and Inv are the same as in Method Invariant

* Subprocedure \text{MinConstraints}:

  Input:
    N = (S, T, E, M0)  : Petri net
    A'                 : Satisfying assignment for C'
  Output:
    C_M                : Minimization Constraints

  Pseudocode:

\end{verbatim}

\begin{align*}
  C_M(N, A') =& \left( \bigwedge_{s \in S} \left(
      (s > 0 \Rightarrow b_s = 1) \land (s = 0 \Rightarrow b_s = 0)
    \right) \right) \land
    \left( \sum_{s \in S} b_s < \sum_{s \in S : A'(s) > 0} 1 \right)
\end{align*}

\newpage

\begin{verbatim}
* Example:

  - Dual state constraints C' and satisfying assignment A' for C' as in example for method Invariant with property 2

  - Minimization constraints C_M:

    p1     > 0 => b_p1   = 1
    |            -----------
    |               |
    |            place p1 appears in invariant
    |
    place p1 coefficient in the invariant

    p1     = 0 => b_p1   = 0
    |           ------------
    |               |
    |            place p1 does not appear in invariant
    |
    place p1 coefficient in the invariant

    p2    > 0 => b_p2    = 1
    p2    = 0 => b_p2    = 0
    p3    > 0 => b_p3    = 1
    p3    = 0 => b_p3    = 0
    bit1  > 0 => b_bit1  = 1
    bit1  = 0 => b_bit1  = 0
    nbit1 > 0 => b_nbit1 = 1
    nbit1 = 0 => b_nbit1 = 0

    q1    > 0 => b_q1    = 1
    q1    = 0 => b_q1    = 0
    q2    > 0 => b_q2    = 1
    q2    = 0 => b_q2    = 0
    q3    > 0 => b_q3    = 1
    q3    = 0 => b_q3    = 0
    q4    > 0 => b_q4    = 1
    q4    = 0 => b_q4    = 0
    q5    > 0 => b_q5    = 1
    q5    = 0 => b_q5    = 0
    nbit2 > 0 => b_nbit2 = 1
    nbit2 = 0 => b_nbit2 = 0

    b_p1 + b_p2 + b_p3 + b_bit1 + b_nbit1 + b_q1 + b_q2 + b_q3 + b_q4 + b_q5
         + b_nbit2 < 4
    --------------   |
                |    |
                |    number of places appearing in current invariant for A'
                |    = #{s | A'(s) > 0} = #{p1, p2, p3, bit1}
                |
  number of places appearing in new invariant
  
  - Model A' for C' \cup C_M:

    p1       = 1
    p2       = 1
    p3       = 1
    bit1     = 0
    nbit1    = 0
    q1       = 0
    q2       = 0
    q3       = 0
    q4       = 0
    q5       = 0
    nbit2    = 0
    target_1 = 1
    
    b_p1     = 1
    b_p2     = 1
    b_p3     = 1
    b_bit1   = 0
    b_nbit1  = 0
    b_q1     = 0
    b_q2     = 0
    b_q3     = 0
    b_q4     = 0
    b_q5     = 0
    b_nbit2  = 0
    
    
  - Minimized Invariant:

    p1 + p2 + p3 <= 1
\end{verbatim}
    
\newpage

\section{Method Invariant by Refinement}

\begin{verbatim}
* Code, property, and Petri net: same as in section Method Safety by Refinement.

* Method Invariant by Refinement
\end{verbatim}

\ifthenelse{\equal{\isDraft}{true}}{\begin{figure}
  \begin{tikzpicture}[
    every path/.style={draw, ->, >=stealth, shorten >=2pt, shorten <=2pt}
    ]
    \node[state] (begin) {BEGIN};
    \node[action, below=of begin] (c) {$C:=\mathcal C(N)$\\
      $D:=\{\}$};
    \node[decision, below=of c] (satc) {$\text{SAT}(C \cup \{\neg P\})$};
    \node[action, below=of satc] (modelc) {$A:=\text{Model}(C \cup \{\neg P\})$\\
      $C_{\theta}:=\text{TrapConditions}(N, A)$};
    \node[decision, below=of modelc] (satctheta) {$\text{SAT}(C_\theta)$};
    \node[action, below=of satctheta] (modelctheta) {$A_\theta:=\text{Model}(C_\theta)$\\
      $\delta:=\Delta(A_\theta)$\\
      $C:=C \cup \{\delta\}$\\
      $D:=D \cup \{\delta\}$};
    \node[print, right=of satctheta] (noinv2) {No Invariant};
    \node[state, below=of noinv2] (end3) {END};
    
    \node[action, right=of satc] (cprime) {$C':=\mathcal C'(N, \neg P \land D)$};
    \node[decision, right=of cprime] (satcprime) {$\text{SAT}(C')$};
    \node[action, below=of satcprime] (inv) {$A':=\text{Model}(C')$\\
      $I:=D \land \text{Inv}(N, A')$};
    \node[print, below=of inv] (printinv) {Invariant: $I$};
    \node[print, right=of satcprime] (noinv) {No Invariant};
    \node[state, below=of noinv] (end1) {END};
    \node[state, below=of printinv] (end2) {END};
    
    \draw (cprime) edge (satcprime);
    \draw (satcprime) edge node[above]{NO} (noinv);
    \draw (satcprime) edge node[right]{YES} (inv);
    \draw (inv) edge (printinv);
    \draw (noinv) edge (end1);
    \draw (printinv) edge (end2);
    
    \draw (begin) edge (c);
    \draw (c) edge coordinate[pos=.5] (edgein) (satc);
    \draw (satc) edge node[above]{NO} (cprime);
    \draw (satc) edge node[right]{YES} (modelc);
    \draw (modelc) edge (satctheta);
    \draw (satctheta) edge node[above]{NO} (noinv2);
    \draw (noinv2) edge (end3);
    \draw (satctheta) edge node[right]{YES} (modelctheta);
    \draw (modelctheta.south) -- ([yshift=-0.5cm] modelctheta.south)
    -| ([xshift=-2cm] modelctheta.west) |- (edgein);
  \end{tikzpicture}
  \caption{Diagram for Function Invariant by Refinement}
  \label{fig_function_invariant_by_refinement_diagram}
\end{figure}

%%% Local Variables: 
%%% mode: latex
%%% TeX-master: "main"
%%% End: 
}{}

\begin{figure}
  \hrule
  \centering
  \begin{minipage}[t]{.04\columnwidth}
   \mbox{} \\ \\ \\ \\ \\ \\ \\ \\ \\ \\ \\ \\ \\ \\ \\ \\
   \raggedleft 1 \\ 2 \\ 3 \\ 4 \\ 5 \\ 6 \\ 7 \\ 8 \\ 9 \\ 10 \\
   11 \\ 12 \\ 13 \\ 14 \\ 15 \\ 16 \\ 17 \\ 18 \\ 19 \\ 20 \\
   21 \\ 22 %\\ 23 \\ 24 \\ 25 \\ 26 \\ 27 \\ 28 \\ 29 % \\ 30 \\ 
%   31 \\ 32 \\ 33 \\ 34 \\ 35 \\ 36
  \end{minipage}
  \begin{minipage}[t]{.94\columnwidth}
    \mbox{}\\
    \algFunction \invariantref \\
    \algInput\\
    \tabT $N$ - Petri net \\
    \tabT $P$ - property \\
    \algVars\\
    \tabT $C$ - state constraints \\
    \tabT $A$ - satisfying assignment for the state constraints \\
    \tabT $C_\theta$ - trap conditions \\
    \tabT $A_\theta$ - satisfying assignment for the trap conditions \\
    \tabT $\delta$ - trap refinement constraint \\
    \tabT $C_T$ - accumulated trap refinement constraints \\
    \tabT $C'$ - dual state constraints \\
    \tabT $A'$ - satisfying assignment for the dual state constraints \\
    \tabT $I$ - invariant \\
    \algBegin\\
    \tabT $C$ \algAssgn $\mathcal C(N)$ \\
    \tabT $C_T$ \algAssgn $\{\}$ \\
    \tabT \algWhile SAT$(C \cup \{\neg P\} \cup C_T)$ \algDo \\
    \tabTT $A$ \algAssgn Model$(C \cup \{\neg P\} \cup C_T)$ \\
    \tabTT $C_\theta$ \algAssgn TrapConditions$(N, A)$ \\
    \tabTT \algIf SAT$(C_\theta)$ \algThen \\
    \tabTTT $A_\theta$ \algAssgn Model$(C_\theta)$ \\
    \tabTTT $\delta$ \algAssgn $\Delta(A_\theta)$ \\
    \tabTTT $C_T$ \algAssgn $C_T \cup \{\delta\}$ \\
    \tabTT \algElse \\
    \tabTTT \algReturn ``Failed at finding an invariant'' \\
    \tabTT \algFi \\
    \tabT \algOd \\
    \tabT $C'$ \algAssgn $\mathcal C'(N, \neg P, C_T)$ \\
    \tabT \algIf SAT$(C')$ \algThen \\
    \tabTT $A'$ \algAssgn Model$(C')$ \\
    \tabTT $I$ \algAssgn $C_T \land \text{Inv}(N, A')$ \\
    \tabTT \algReturn ``Invariant I for the petri net: $I$'' \\
    \tabT \algElse \\
    \tabTT \algReturn ``Failed at finding an invariant'' \\
    \tabT \algFi \\
    \algEnd
  \end{minipage}
  \vspace{1.5ex}
  \hrule
  \vspace{1.5ex}
  (a) Function Invariant by Refinement
  \vspace{2.5ex}
  \begin{minipage}{\columnwidth}
    \centering
    \begin{align*}
      \text{TrapConditions}(N, A) =& \left(
        \bigwedge_{s \in S} \left( s \Rightarrow
          \bigwedge_{t \in \post{s}} \bigvee_{p \in \post{t}} p
        \right) \right) \land
      \left( \bigvee_{s \in S: M_0(s) > 0} s \right) \land
      \left( \bigwedge_{s \in S: A(s) > 0} \neg s \right)
    \end{align*}
    \vspace{1.5ex}
    (b) Function TrapConditions
  \end{minipage}
  \vspace{2.5ex}
  \begin{minipage}{.4\columnwidth}
    \centering
    \begin{align*}
      & \Delta(A_\theta) = \left( \sum_{A_\theta(s)} s \ge 0 \right)
    \end{align*}
    \vspace{1.5ex}
    (c) Function $\Delta$
  \end{minipage}
  \begin{minipage}{.4\columnwidth}
    \centering
    \begin{align*}
      \text{Inv}(N, A') =& \left( \sum_{s \in S} A'(s) \cdot s \le
        \sum_{s \in S} A'(s) \cdot M_0(s) \right)
    \end{align*}
    \vspace{1.5ex}
    (d) Function Inv
  \end{minipage}
  \caption{Function \invariantref\ and auxiliary functions}
  \label{fig_function_invariant_by_refinement_pseudocode}
\end{figure}

%%% Local Variables: 
%%% mode: latex
%%% TeX-master: "main"
%%% End: 


\begin{verbatim}
* Subprocedures TrapConditions and \Delta are the same as
  in Method Safety by Refinement

* Subprocedures \mathcal{C'} and Inv are the same as
  in Method Invariant
\end{verbatim}

\newpage

\begin{verbatim}
* Example:

  - Using property 1

  - Trap constraints as in example from section Method Safety by Refinement.
  
  - Dual state constraints C':

    0 >= - p1 + p2      + bit1 - nbit2
    0 >=      - p2 + p3
    0 >= + p1      - p3 - bit1 + nbit2

    0 >= - q1 + q2                - nbit2
    0 >=      - q2 + q3
    0 >=           - q3 + q4      + nbit2
    0 >= + q1           - q4
    0 >=      - q2           + q5
    0 >= + q1                - q5 + nbit2

    p1 + q1 + nbit1 + nbit2 < 2 * target_1 + trap_1

    p1    >= 0
    p2    >= trap_1
    p3    >= target_1
    bit1  >= 0
    nbit1 >= trap_1
    q1    >= 0
    q2    >= trap_1
    q3    >= trap_1
    q4    >= 0
    q5    >= target_1
    nbit2 >= trap_1

    target_1 >= 0
    trap_1   >= 0

  - Model A':

    p1       = 0
    p2       = 1
    p3       = 1
    bit1     = 0
    nbit1    = 1
    q1       = 0
    q2       = 1
    q3       = 1
    q4       = 0
    q5       = 1
    nbit2    = 1
    target_1 = 1
    trap_1   = 1
    
  - Invariant:

    p2        q2 + q3      + nbit1 + nbit2 >= 1
    p2 + p3 + q2 + q3 + q5 + nbit1 + nbit2 <= 2

\end{verbatim}
