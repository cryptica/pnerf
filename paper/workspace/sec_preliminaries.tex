\section{Preliminaries}

A \emph{Petri net} is a tuple $(P, T, F, M_0)$, where $P$ is the set of
\emph{places}, $T$ is the set of \emph{transitions},
$F \subseteq P\times T \cup T\times P$ is the \emph{flow relation}
and $M_0$ is the initial marking.
We identify $F$ with its characteristic function
$P\times T \cup T\times P \to \{0, 1\}$.

For $x\in P\cup T$, the \emph{pre-set} is
$\pre{x}=\{y\in P\cup T\mid (y,x) \in F\}$
and the \emph{post-set} is $\post{x}=\{y\in P\cup T\mid (x,y) \in F\}$.
The pre- and post-set of a subset of $P \cup T$ are the union of
the pre and post-sets of its elements.

A \emph{marking} of a Petri net is a function $M\colon P \to \mathbb{N}$,
which describes the number of tokens $M(p)$ that the marking puts in
each place $p$.

Petri nets are represented graphically as follows: places and transitions
are represented as circles and boxes, respectively. An element $(x,y)$
of the flow relation is represented by an arc leading from $x$ to $y$.
A token on a place $p$ is represented by a black dot in the circle
corresponding to $p$.

A transition $t \in T$ is \emph{enabled at $M$} iff
$\forall p \in \pre{t} \colon M(p) \ge F(p, t)$.
If $t$ is enabled at $M$, then $t$ may \emph{fire} or \emph{occur},
yielding a new marking $M'$ (denoted as $M \xrightarrow{t} M'$),
where $M'(p) = M(p) + F(t,p) - F(p,t)$.

A sequence of transitions, $\sigma = t_1 t_2 \ldots t_r$ is an
$\emph{occurence sequence}$ of $N$ iff there exist markings
$M_1, \ldots, M_r$ such that $M_0 \xrightarrow{t_1} M_1
\xrightarrow{t_2} M_2 \ldots \xrightarrow{t_r} M_r$. The marking
$M_r$ is said to be \emph{reachable} from $M_0$ by the occurence
of $\sigma$ (denoted $M \xrightarrow{\sigma} M_r$).

A property $P$ is a safety property expressed over linear arithmetic
formulas. The property $P$ holds on a marking $M$ iff $M \models P$.
Examples of properties are $s_1 \le 2$, $s_1 + s_2 \ge 1$ and
$((s_1 \le 1) \land (s_2 \ge 1)) \lor (s_3 \le 1)$.

A Petri net $N$ satisfies a property $P$ (denoted by $N \models P$)
iff for all reachable markings $M_0 \xrightarrow{\sigma} M$
$M \models P$ holds.

An invariant $I$ of a Petri net $N$ is a property such that $N$ satisfies $I$.
The invariant $I$ is inductive iff for all markings
$M$, if $M \models I$ and $M \xrightarrow{t} M'$ for some $t$, then
$M' \models I$.

A trap is a set of places $S \subseteq P$ such that $\post{S} \subseteq \pre{S}$.
If a trap $S$ is marked in $M_0$, i.e. $\sum_{p \in S} M_0(p) > 0$, then it is also marked in all reachable markings.

