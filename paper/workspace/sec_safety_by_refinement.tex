\section{Refining Marking Equations with Traps}
\label{sec_method_safety_by_refinement}

The bottom line of the work of Esparza and Melzer \cite{EsparzaM00} was to
strengthen \safety{} with additional
\emph{trap constraints}. A {\em trap} of a Petri net $N=(P, T, F, m_0)$ is a subset of places
$Q \subseteq P$ satisfying the following condition for every
transition $t \in T$: if $t$ is an output transition of at least one
place of $Q$, then it is also an input transition of at least one
place of $Q$. Equivalently, $Q$ is a trap if its set of output
transitions is included in its set of input transitions, i.e., if
$\post{Q} \subseteq \pre{Q}$. Here we present a variant of Esparza's and
Melzer's method that encodes traps using Boolean constraints. We
call the new method \safetyref.

The method \safetyref\ is based on the following observation about traps.
If $Q$ is a trap and a marking $m$ marks $Q$, i.e., $m(p) > 0$ for some $p\in Q$,
then for each occurrence sequence $\sigma$ and marking $m'$ such that $m \xrightarrow{\sigma} m'$,
we also have $m'(p') > 0$ for some $p'\in Q$.  
Indeed, by the trap property any transition removing tokens from places of
$Q$ also adds at least one token to some place of $Q$. 
So, while $m'(Q)$ can be smaller than $m(Q)$, it can never become $0$. 
In particular, if a trap $Q$ satisfies $m_0(Q) > 0$, then every
reachable marking $m$ satisfies $m(Q) > 0$ as well.

Since the above property must hold for any trap, we can restrict the constrains
from method \safety{} as follows.
We additionally add a Boolean variable for each place $p\in P$.
These variables will be used as indicators for traps.
First, we write a constraint to specify that the Boolean variables encode a trap:
\[
\mathit{trap}(B) ::= \bigwedge_{t\in T}\left[ \bigvee_{p \in \pre{t}} B(p) \Rightarrow \bigvee_{p\in \post{t}} B(p) \right]\,.
\]
Next, we define a predicate $\mathit{mark}(m, B)$ that specifies that a marking marks a trap:
\[
\mathit{mark}(m, B) ::= \bigvee_{p\in P} B(p)\wedge (m(p) > 0)\,.
\]
Finally, we conjoin the following constraint to the constraints~\eqref{eq:safety}: 
\begin{equation}\label{eq:trap}
\forall B: \mathit{trap}(B) \wedge \mathit{mark}(m_0, B) \Rightarrow \mathit{mark}(M, B)\,.
\end{equation}
%
This constraint conceptually enumerates over all subsets of places, and ensures that if
the subset forms a trap, and this trap is marked by the initial marking, then
it is also marked in the current marking. 
Thus, markings violating the trap constraint are eliminated.

While the above constraint provides a refinement of the \safety\ method, it requires the
SMT solver to reason with universally quantified variables.
Instead of directly using universal quantifiers, we use a counterexample-guided
heuristic \cite{EsparzaM00,CEGIS} of adding trap constraints one-at-a-time in
the following way.

If the set of constraints constructed so far (for instance, the set
given by the method \safety) is feasible, the SMT solver delivers a
model that assigns values to each place, corresponding to a potentially reachable marking $m$.  
We search for a trap $P_m$ that violates the trap condition \eqref{eq:trap} for this specific model $m$.
If we find such a trap, then we know that $m$ is unreachable, and we can
add the constraint $\sum_{p \in P_m} M(p) \geq 1$ to exclude all markings that violate
this specific trap condition.

The search for $P_m$ is a pure Boolean satisfiability question.
We ask for an assignment to 
\begin{equation}\label{eq:booltrap}
\mathit{trap}(B) \wedge \mathit{mark}(m_0,B) \wedge \lnot \mathit{mark}(m, B)
\end{equation}
(and notice that for a fixed marking $m$, the predicate $\mathit{mark}(m,B)$ can
be evaluated to a Boolean predicate).
Given a satisfying assignment $b$ for this formula, we add the constraint
\begin{equation}\label{eq:trapcons}
\sum_{p \in P \mid b(p) = 1} M(p) \geq 1
\end{equation} 
to the constraints to rule out solutions that
do not satisfy this trap constraint.
We iteratively add such constraints until either the constraints are unsatisfiable
or the Boolean constraints~\eqref{eq:booltrap} are unsatisfiable (i.e., no traps are found
to invalidate the current solution). 
% How do we find $P_M$? The integration of boolean and arithmetic
% constraints allows to use the SMT-solver. The constraint problem
% delivering a trap that is marked at $M_0$ but not marked at a given
% marking $M$ is constructed as follows. As in the method \safety, we
% introduce a variable $p$ for every place $P$, but this time of {\em
%   boolean type}, with the following intended meaning: if $p$ has the
% value {\it true}, then $p$ belongs to the trap.
% 
% For each transition $t$ we add the constraint
% $$\bigvee_{p \in \pre{t}} p  \Rightarrow \bigvee_{p' \in \post{t}} p'$$
% \noindent which formalizes the trap property. Moreover we add
% %
% $$\bigvee_{p \in P, M_0(p) > 0} p \wedge \bigwedge_{p' \in P, M(p') > 0} \neg p'$$
% 
% \noindent modelling that the trap is marked at $M_0$ but not at $M$.
 
This yields the method \safetyref.  The method is still not
complete \cite{EsparzaM00}: one can find nets and unreachable markings that 
mark all traps of the net.

Let us apply the algorithm \safetyref{} to Lamport's algorithm and the
mutual exclusion property. 
Recall that the markings violating the property are those satisfying $ p_3 \geq 1$ and  $q_5 \geq 1$. \safety{} yields a
satisfying assignment with $p_3 = bit_1 = q_5 = 1$,
and $p=0$ for all other places $p$, 
which corresponds to a potentially reachable marking $m$. 
We search for a trap marked at $m_0$ but not at $m$. The constraints 
derived from the trap property are:
$$\begin{array}{l}
\begin{array}[t]{lcl}
  p_1 \lor notbit_1 &\implies& p_2 \lor    bit_1 \\
  p_2 \lor notbit_2 &\implies& p_3 \lor notbit_2 \\
  p_3 \lor    bit_1 &\implies& p_1 \lor notbit_1
\end{array} \qquad
\begin{array}[t]{lcl}
  q_1 \lor notbit_2 &\implies& q_2               \\
  q_2 \lor    bit_1 &\implies& q_3 \lor    bit_1 \\
  q_3               &\implies& q_4 \lor notbit_2 \\
  q_4 \lor notbit_1 &\implies& q_1 \lor notbit_1 \\
  q_2 \lor notbit_1 &\implies& q_5 \lor notbit_1 \\
  q_5               &\implies& q_1 \lor notbit_2
\end{array}
\end{array}
$$
and the following constraints model that at least one of the places 
initially marked belongs to the trap, but none of the places marked at 
the satisfying assigment do:
$$p_1 \lor q_1 \lor notbit_1 \lor notbit_2  \qquad \neg p_3 \land \neg q_5 \land
\neg bit_1$$ For this set of constraints we find the satisfying assignment 
that sets $p_2$, $notbit_1$, $notbit_2$, $q_2$, $q_3$ to $\mathit{true}$ and all
other variables to $\mathit{false}$. 
So this set of places is an initially marked trap, and 
so every reachable marking should put at least one token in it. 
Hence we can add the refinement constraint to marking
constraints~\eqref{eq:safety}:
\begin{align*}
  p_2 + q_2 + q_3 + notbit_1 + notbit_2 & \ge 1\,.
\end{align*}
%
On running the SMT solver again, we find the constraints are unsatisfiable, 
proving that the mutual exclusion property holds.


\iffalse

\begin{verbatim}
* Method Safety by Refinement:

  Subprocedure TrapConditions constructs trap conditions C_\theta corresponding to N and A.
  Subprocedure \Delta constructs refinement constraint \delta corresponding to A_\theta.
\end{verbatim}

\ifthenelse{\equal{\isDraft}{true}}{\begin{figure}
  \begin{center}
    \begin{tikzpicture}[
      every path/.style={draw, ->, >=stealth, shorten >=2pt, shorten <=2pt}
      ]
      \node[state] (begin) {BEGIN};
      \node[action, below=of begin] (c) {$C:=\mathcal C(N)$};
      \node[decision, below=of c] (satc) {$\text{SAT}(C \cup \{\neg P\})$};
      \node[action, below=of satc] (modelc) {$A:=\text{Model}(C \cup \{\neg P\})$\\
        $C_{\theta}:=\text{TrapConditions}(N, A)$};
      \node[decision, below=of modelc] (satctheta) {$\text{SAT}(C_\theta)$};
      \node[action, below=of satctheta] (modelctheta) {$A_\theta:=\text{Model}(C_\theta)$\\
        $\delta:=\Delta(A_\theta)$\\
        $C:=C \cup \{\delta\}$};
      \node[print, right=of satc] (yes) {N satisfies P};
      \node[print, right=of satctheta] (dontknow) {Don't know};
      \node[state, right=of yes] (end1) {END};
      \node[state, right=of dontknow] (end2) {END};

      \draw (begin) edge (c);
      \draw (c) edge coordinate[pos=.5] (edgein) (satc);
      \draw (satc) edge node[above]{NO} (yes);
      \draw (yes) edge (end1);
      \draw (satc) edge node[right]{YES} (modelc);
      \draw (modelc) edge (satctheta);
      \draw (satctheta) edge node[above]{NO} (dontknow);
      \draw (dontknow) edge (end2);
      \draw (satctheta) edge node[right]{YES} (modelctheta);
      \draw (modelctheta.south) -- ([yshift=-0.5cm] modelctheta.south)
      -| ([xshift=-2cm] modelctheta.west) |- (edgein);
    \end{tikzpicture}
  \end{center}
  \caption{Diagram for Method Safety by Refinement}
  \label{fig:method-safety-by-refinement-diagram}
\end{figure}}{}

\begin{figure}
  \hrule
  \centering
  \begin{minipage}[t]{.04\columnwidth}
   \mbox{} \\ \\ \\ \\ \\ \\ \\ \\ \\ \\ \\ \\ \\
   \raggedleft 1 \\ 2 \\ 3 \\ 4 \\ 5 \\ 6 \\ 7 \\ 8 \\ 9 \\ 10 \\
   11 \\ 12 \\ 13 \\ 14 %\\ 15 \\ 16 \\ 17 \\ 18 \\ 19 % \\ 20 \\
%   21 \\ 22 \\ 23 \\ 24 \\ 25 \\ 26 \\ 27 \\ 28 \\ 29 % \\ 30 \\ 
%   31 \\ 32 \\ 33 \\ 34 \\ 35 \\ 36
  \end{minipage}
  \begin{minipage}[t]{.94\columnwidth}
    \mbox{}\\
    \algMethod \safetyref \\
    \algInput\\
    \tabT $N$ - Petri net \\
    \tabT $P$ - property \\
    \algVars\\
    \tabT $C$ - state constraints \\
    \tabT $A$ - satisfying assignment for the state constraints \\
    \tabT $C_\theta$ - trap conditions \\
    \tabT $A_\theta$ - satisfying assignment for the trap conditions \\
    \tabT $\delta$ - trap refinement constraint \\
    \tabT $C_T$ - accumulated trap refinement constraints \\
    \algBegin\\
    \tabT $C$ \algAssgn $\mathcal C(N)$ \\
    \tabT $C_T$ \algAssgn $\{\}$ \\
    \tabT \algWhile SAT$(C \cup \{\neg P\} \cup C_T)$ \algDo \\
    \tabTT $A$ \algAssgn Model$(C \cup \{\neg P\} \cup C_T)$ \\
    \tabTT $C_\theta$ \algAssgn TrapConditions$(N, A)$ \\
    \tabTT \algIf SAT$(C_\theta)$ \algThen \\
    \tabTTT $A_\theta$ \algAssgn Model$(C_\theta)$ \\
    \tabTTT $\delta$ \algAssgn $\Delta(A_\theta)$ \\
    \tabTTT $C_T$ \algAssgn $C_T \cup \{\delta\}$ \\
    \tabTT \algElse \\
    \tabTTT \algReturn ``The petri net may not satisfy the property'' \\
    \tabTT \algFi \\
    \tabT \algOd \\
    \tabT \algReturn ``The petri net satisfies the property'' \\
    \algEnd
  \end{minipage}
  \vspace{1.5ex}
  \hrule
  \caption{Method Safety by Refinement}
  \label{fig_method_safety_by_refinement}
\end{figure}

%%% Local Variables: 
%%% mode: latex
%%% TeX-master: "main"
%%% End: 


\begin{verbatim}
* Property of trap conditions C_\theta: If C_\theta is sat then there is a set S such that
  1. S is a trap in the net N
  2. S is marked in the initial marking M0
  3. S is unmarked in the assignment A

* Property of A_\theta: for each place s, A_\theta(s) iff s \in S

* Property of refinement constraint \delta: Constraint \delta refines the abstraction, i.e.
  1. A ^ \delta is unsat (\delta excludes A)
  2. N |= \delta is sat  (\delta is a property of N)
\end{verbatim}

\newpage

\begin{verbatim}
* Subprocedure TrapConditions:

  Input:
    N = (S, T, F, M0) : Petri net
    A             : Satisfying assignment for C \cup { ~P }
  Output:
    C_\theta      : Trap conditions

  Pseudocode:
  
\end{verbatim}
\begin{align*}
  \text{TrapConditions}(N, A) =& \left(
      \bigwedge_{s \in S} \left( s \Rightarrow
      \bigwedge_{t \in \post{s}} \bigvee_{p \in \post{t}} p
    \right) \right) \land
    \left( \bigvee_{s \in S: M_0(s) > 0} s \right) \land
    \left( \bigwedge_{s \in S: A(s) > 0} \neg s \right)
\end{align*}
\begin{verbatim}

* Subprocedure \Delta:

  Input:
    A_\theta      : Satisfying assignment for C_\theta
  Output:
    \delta        : Refinement constraint \delta

  Pseudocode:

\end{verbatim}
\begin{align*}
  & \Delta(A_\theta) = \left( \sum_{A_\theta(s)} s \ge 0 \right)
\end{align*}

\newpage

\begin{verbatim}
* Example

  - State constraints C same as in the example for method Safety.
  
  - Using property 1
  
  - Negated property 1 \neg P:
\end{verbatim}


{\Large OK, but how does it continue? Is this enough to prove the property,
or do we need another trap?}

\iffalse
\begin{verbatim}
    p2 + q2 + q3 + nbit1 + nbit2 >= 1
    ^    ^    ^    ^       ^
    |    |    |    |       |
    ------------------------
      |
      S = {p2, q2, q3, nbit1, nbit2}, therefore \delta excludes
      assignment A in the next iteration
\end{verbatim}
\fi

\newpage

\begin{verbatim}

* Trap implication:

  place s \in S =>    /\       \    /
                     /  \       \  /     place p \in S
                    /    \       \/
                   t \in s*   p \in t*   


  "if s is in trap S then for each output transition t at least one successor p is in trap S"


* Refinement constraint \delta:

  \Sigma s >= 1  
  A(s)

  "At least one place in S is always marked"
\end{verbatim}

\fi


%%% Local Variables: 
%%% mode: latex
%%% TeX-master: "main"
%%% End: 
