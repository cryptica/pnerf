\section{Method Safety}

The method Safety checks that a given Petri net \verb=N= never violates a property \verb=P=.
We present the method Safety by example on Lamport's 1-bit algorithm [Esparza1997].

\begin{verbatim}
* Method Safety:

  Subprocedure \mathcal{C} constructs state constraints C corresponding to N.
\end{verbatim}

\ifthenelse{\equal{\isDraft}{true}}{\begin{figure}
  \begin{center}
    \begin{tikzpicture}[
      every path/.style={draw, ->, >=stealth, shorten >=2pt, shorten <=2pt}
      ]
      \node[state] (begin) {BEGIN};
      \node[action, below=of begin] (c) {$C:=\mathcal C(N)$};
      \node[decision, below=of c] (satc) {$\text{SAT}(C \cup \{\neg P\})$};
      \node[print, right=of satc] (yes) {N satisfies P};
      \node[print, below=of satc] (dontknow) {Don't know};
      \node[state, right=of yes] (end1) {END};
      \node[state, below=of dontknow] (end2) {END};
      
      \draw (begin) edge (c);
      \draw (c) edge (satc);
      \draw (satc) edge node[above]{NO} (yes);
      \draw (yes) edge (end1);
      \draw (satc) edge node[right]{YES} (dontknow);
      \draw (dontknow) edge (end2);
    \end{tikzpicture}
  \end{center}
  \caption{Diagram for Method Safety}
  \label{fig:method-safety-pseudocode}
\end{figure}}{}

\begin{figure}
  \begin{algorithmic}[1]
    \State $C := \mathcal C(N)$
    \If {SAT$(C \cup \{\neg P\})$}
    \State \Return 'The petri net may not satisfy the property'
    \Else
    \State \Return 'The petri net satisfies the property!'
    \EndIf
  \end{algorithmic}
  \caption{Pseudocode for Method Safety}
  \label{fig:method-safety-pseudocode}
\end{figure}


\begin{verbatim}
* Property of state constraints C: If C U {\neg P} is unsat then N |= P.

* Place equation:
  
  For a given place s the place equation is

  # of tokens in s = initial number of tokens of place s
                     + # times each input transition of s fires
                     - # times each output transition of s fires

* Non-negativity conditions:

  # of tokens in place s           >= 0
  # of times transition t is fired >= 0

* Subprocedure \mathcal{C}:

  Input:
    (S, T, E, M0) : Petri net
  Output:
    C : State constraints

  Pseudocode:

\end{verbatim}

\begin{align*}
  C(S, T, E, M_0) :=& \left( \bigwedge_{s \in S} \left(
    s = M_0(s) + \sum_{(t, s) \in E} t - \sum_{(s, t) \in E} t
  \right) \right) \land
    \left( \bigwedge_{s \in S} s \ge 0 \right) \land
    \left( \bigwedge_{t \in T} t \ge 0 \right)
\end{align*}

\newpage

\begin{verbatim}
* Example

  - Code:

  Process 1:                     |      Process 2:
                                 |
      bit1 := false              |      bit2 := false
      while true do              |      while true do
  p1:   bit1 := true             |  q1:   bit2 := true
  p2:   while bit2 do skip od    |  q2:   if bit1 then
  p3:   <critical section>       |  q3:     bit2 := true
        bit1 := false            |  q4:     while bit1 do skip od
      od                         |          goto q1
                                 |        fi
                                 |  q5:   <critical section>
                                 |        bit2 := false
                                 |      od

  - Property 1: Process 1 and Process 2 are never in their respective critical section at the same time.

  - Property 1: p3 + q5 <= 1

  - Property 2: Process 1 is in at most one state at a time

  - Property 2: p1 + p2 + p3 <= 1
  
  - Property 3: Variable bit1 is either true or false

  - Property 3: bit1 + nbit1 = 1, alternatively written as
                bit1 + nbit1 <= 1 /\ bit1 + nbit1 >= 1

  - Petri net:

\end{verbatim}

\begin{figure}[t]
  \begin{center}
    \begin{tikzpicture}[
        scale=0.6,
        every node/.style={scale=0.6}
      ]
      \draw[box] (-2,8.5) rectangle (1,0);
      \draw[box] (3,8.5) rectangle (7.5,0);
      
      \node at (-0.5, 0.5) {First Process};
      \node at (5, 0.5) {Second Process};

      \node[place, label=left:$p_3$] (p3) at (0,7) {};
      \node[transition, label=left:$s_3$] (s3) at (0,6) {};
      \node[place, tokens=1, label=left:$p_1$] (p1) at (0,5) {};
      \node[transition, label=left:$s_1$] (s1) at (0,4) {};
      \node[place, label=left:$p_2$] (p2) at (0,3) {};
      \node[transition, label=left:$s_2$] (s2) at (0,2) {};

      \draw (p3) edge[flow] (s3);
      \draw (s3) edge[flow] (p1);
      \draw (p1) edge[flow] (s1);
      \draw (s1) edge[flow] (p2);
      \draw (p2) edge[flow] (s2);
      \draw[flow, rounded corners=4mm] (s2) |- (-1.5,1) -- (-1.5,8) -| (p3);
      
      \node[transition, label=above:$t_2$] (t2) at (4,7) {};
      \node[place, label=right:$q_3$] (q3) at (4,6) {};
      \node[transition, label=right:$t_3$] (t3) at (4,5) {};
      \node[place, label=right:$q_4$] (q4) at (4,4) {};
      \node[transition, label=left:$t_4$] (t4) at (4,3) {};

      \node[place, label=right:$q_2$] (q2) at (5.5,7) {};
      \node[transition, label=right:$t_5$] (t5) at (5.5,6) {};
      \node[place, label=right:$q_5$] (q5) at (5.5,5) {};
      \node[transition, label=right:$t_6$] (t6) at (5.5,4) {};
      \node[place, tokens=1, label=right:$q_1$] (q1) at (5.5,3) {};
      \node[transition, label=right:$t_1$] (t1) at (5.5,2) {};
      
      \draw (q2) edge[flow] (t5);
      \draw (t5) edge[flow] (q5);
      \draw (q5) edge[flow] (t6);
      \draw (t6) edge[flow] (q1);
      \draw (q1) edge[flow] (t1);
      \draw[rounded corners=4mm, flow] (t1) |- (7,1) -- (7,8) -| (q2);
      
      \draw (q2) edge[flow] (t2);
      \draw (t2) edge[flow] (q3);
      \draw (q3) edge[flow] (t3);
      \draw (t3) edge[flow] (q4);
      \draw (q4) edge[flow] (t4);
      \draw (t4) edge[flow] (q1);
      
      \node[place, label=above:$bit_1$] (bit1) at (2,6) {};
      \node[place, tokens=1, label=below:$notbit_1$] (notbit1) at (2,4) {};
      \node[place, tokens=1, label=below:$notbit_2$] (notbit2) at (2,2) {};
      
      \draw (bit1) edge[flow] (s3);
      \draw (s3) edge[flow] (notbit1);
      \draw (notbit1) edge[flow] (s1);
      \draw (s1) edge[flow] (bit1);
      \draw (t2) edge[flow, <->] (bit1);
      \draw (notbit1) edge[flow, <->, bend left=18] (t5);
      \draw (notbit1) edge[flow, <->] (t4);
      \draw (notbit2) edge[flow, <->] (s2);
      \draw (t3) edge[flow] (notbit2);
      \draw (t6) edge[flow, bend left=30] ([xshift=-0.2mm, yshift=1mm] notbit2.east);
      \draw (notbit2) edge[flow] (t1.west);
    \end{tikzpicture}
  \end{center}
  \caption{Petri net for Lamport's 1-bit algorithm.}
  \label{fig_lamport_net}
\end{figure}

%%% Local Variables: 
%%% mode: latex
%%% TeX-master: "main"
%%% End: 


\newpage

\begin{verbatim}
  - State constraints C:

    Place equations:
  


    p1    = 1 - s1      + s3
    ^       ^   ^         ^
    |       |   |         |
    |       |   |         |
    |       |   |         |
    |       |   |        # of tokens given to p1
    |       |   |    
    |       |  # of tokens taken from p1
    |       |
    |      initial number of tokens in p1
    |
    number of tokens in p1

    p2    = 0 + s1 - s2
    p3    = 0      + s2 - s3
    bit1  = 0 + s1      - s3
    nbit1 = 1 - s1      + s3
    q1    = 1 +              - t1           + t4      + t6
    q2    = 0 +              + t1 - t2           - t5
    q3    = 0 +                   + t2 - t3
    q4    = 0 +                        + t3 - t4
    q5    = 0 +                                  + t5 - t6
    nbit2 = 1 +              - t1      + t3           + t6
    
    Non-negativity conditions:
  
    p1    >= 0
    p2    >= 0
    p3    >= 0
    bit1  >= 0
    nbit1 >= 0
    q1    >= 0
    q2    >= 0
    q3    >= 0
    q4    >= 0
    q5    >= 0
    nbit2 >= 0

  - Using property 1, with negated property \neg P:

    p3 + q5 >= 2

    The constraints are satisfiable, therefore we can not
    prove safety with this method.

  - Using property 2, with negated property \neg P:

    p1 + p2 + p3 >= 2
    
    The constraints are unsatisfiable, therefore we
    prove safety with this method.
  
  - Using property 3, with negated property \neg P:

    nbit1 + nbit1 <= 0 \/ nbit1 + nbit1 >= 2
    
    The constraints are unsatisfiable, therefore we
    prove safety with this method.
\end{verbatim}

\newpage

\section{Method Safety by Refinement}

The method Safety by Refinement applies trap conditions to check that a given Petri net \verb=N= never violates a property \verb=P=.

\begin{verbatim}
* Method Safety by Refinement:

  Subprocedure TrapConditions constructs trap conditions C_\theta corresponding to N and A.
  Subprocedure \Delta constructs refinement constraint \delta corresponding to A_\theta.
\end{verbatim}

\ifthenelse{\equal{\isDraft}{true}}{\begin{figure}
  \begin{center}
    \begin{tikzpicture}[
      every path/.style={draw, ->, >=stealth, shorten >=2pt, shorten <=2pt}
      ]
      \node[state] (begin) {BEGIN};
      \node[action, below=of begin] (c) {$C:=\mathcal C(N)$};
      \node[decision, below=of c] (satc) {$\text{SAT}(C \cup \{\neg P\})$};
      \node[action, below=of satc] (modelc) {$A:=\text{Model}(C \cup \{\neg P\})$\\
        $C_{\theta}:=\text{TrapConditions}(N, A)$};
      \node[decision, below=of modelc] (satctheta) {$\text{SAT}(C_\theta)$};
      \node[action, below=of satctheta] (modelctheta) {$A_\theta:=\text{Model}(C_\theta)$\\
        $\delta:=\Delta(A_\theta)$\\
        $C:=C \cup \{\delta\}$};
      \node[print, right=of satc] (yes) {N satisfies P};
      \node[print, right=of satctheta] (dontknow) {Don't know};
      \node[state, right=of yes] (end1) {END};
      \node[state, right=of dontknow] (end2) {END};

      \draw (begin) edge (c);
      \draw (c) edge coordinate[pos=.5] (edgein) (satc);
      \draw (satc) edge node[above]{NO} (yes);
      \draw (yes) edge (end1);
      \draw (satc) edge node[right]{YES} (modelc);
      \draw (modelc) edge (satctheta);
      \draw (satctheta) edge node[above]{NO} (dontknow);
      \draw (dontknow) edge (end2);
      \draw (satctheta) edge node[right]{YES} (modelctheta);
      \draw (modelctheta.south) -- ([yshift=-0.5cm] modelctheta.south)
      -| ([xshift=-2cm] modelctheta.west) |- (edgein);
    \end{tikzpicture}
  \end{center}
  \caption{Diagram for Method Safety by Refinement}
  \label{fig:method-safety-by-refinement-diagram}
\end{figure}}{}

\begin{figure}
  \hrule
  \centering
  \begin{minipage}[t]{.04\columnwidth}
   \mbox{} \\ \\ \\ \\ \\ \\ \\ \\ \\ \\ \\ \\ \\
   \raggedleft 1 \\ 2 \\ 3 \\ 4 \\ 5 \\ 6 \\ 7 \\ 8 \\ 9 \\ 10 \\
   11 \\ 12 \\ 13 \\ 14 %\\ 15 \\ 16 \\ 17 \\ 18 \\ 19 % \\ 20 \\
%   21 \\ 22 \\ 23 \\ 24 \\ 25 \\ 26 \\ 27 \\ 28 \\ 29 % \\ 30 \\ 
%   31 \\ 32 \\ 33 \\ 34 \\ 35 \\ 36
  \end{minipage}
  \begin{minipage}[t]{.94\columnwidth}
    \mbox{}\\
    \algMethod \safetyref \\
    \algInput\\
    \tabT $N$ - Petri net \\
    \tabT $P$ - property \\
    \algVars\\
    \tabT $C$ - state constraints \\
    \tabT $A$ - satisfying assignment for the state constraints \\
    \tabT $C_\theta$ - trap conditions \\
    \tabT $A_\theta$ - satisfying assignment for the trap conditions \\
    \tabT $\delta$ - trap refinement constraint \\
    \tabT $C_T$ - accumulated trap refinement constraints \\
    \algBegin\\
    \tabT $C$ \algAssgn $\mathcal C(N)$ \\
    \tabT $C_T$ \algAssgn $\{\}$ \\
    \tabT \algWhile SAT$(C \cup \{\neg P\} \cup C_T)$ \algDo \\
    \tabTT $A$ \algAssgn Model$(C \cup \{\neg P\} \cup C_T)$ \\
    \tabTT $C_\theta$ \algAssgn TrapConditions$(N, A)$ \\
    \tabTT \algIf SAT$(C_\theta)$ \algThen \\
    \tabTTT $A_\theta$ \algAssgn Model$(C_\theta)$ \\
    \tabTTT $\delta$ \algAssgn $\Delta(A_\theta)$ \\
    \tabTTT $C_T$ \algAssgn $C_T \cup \{\delta\}$ \\
    \tabTT \algElse \\
    \tabTTT \algReturn ``The petri net may not satisfy the property'' \\
    \tabTT \algFi \\
    \tabT \algOd \\
    \tabT \algReturn ``The petri net satisfies the property'' \\
    \algEnd
  \end{minipage}
  \vspace{1.5ex}
  \hrule
  \caption{Method Safety by Refinement}
  \label{fig_method_safety_by_refinement}
\end{figure}

%%% Local Variables: 
%%% mode: latex
%%% TeX-master: "main"
%%% End: 


\begin{verbatim}
* Property of trap conditions C_\theta: If C_\theta is sat then there is a set S such that
  1. S is a trap in the net N
  2. S is marked in the initial marking M0
  3. S is unmarked in the assignment A

* Property of A_\theta: for each place s, A_\theta(s) iff s \in S

* Property of refinement constraint \delta: Constraint \delta refines the abstraction, i.e.
  1. A ^ \delta is unsat (\delta excludes A)
  2. N |= \delta is sat  (\delta is a property of N)
\end{verbatim}

\newpage

\begin{verbatim}
* Subprocedure TrapConditions:

  Input:
    (S, T, E, M0) : Petri net
    A             : Satisfying assignment for C \cup { ~P }
  Output:
    C_\theta      : Trap conditions

  Pseudocode:
  
\end{verbatim}
\begin{align*}
  C_\theta(S, T, E, M_0) :=& \left( \bigwedge_{s \in S} \left( s \Rightarrow
      \bigwedge_{(s, t) \in E} \bigvee_{(t, p) \in E} p
    \right) \right) \land
    \left( \bigvee_{s \in S: M_0(s) > 0} s \right) \land
    \left( \bigwedge_{s \in S: A(s) > 0} \neg s \right)
\end{align*}
\begin{verbatim}

* Subprocedure \Delta:

  Input:
    A_\theta      : Satisfying assignment for C_\theta
  Output:
    \delta        : Refinement constraint \delta

  Pseudocode:

\end{verbatim}
\begin{align*}
  & \delta(A_\theta) := \left( \sum_{A_\theta(s)} s \ge 0 \right)
\end{align*}

\newpage

\begin{verbatim}
* Example

  - State constraints C same as in the example for method Safety.
  
  - Using property 1, with negated property 1 \neg P:

    p3 + q5 >= 2
 
  - Assignment A:
    p1    = 0
    p2    = 0
    p3    = 1
    bit1  = 1
    nbit1 = 0
    q1    = 0
    q2    = 0
    q3    = 0
    q4    = 0
    q5    = 1
    nbit2 = 0

  - Trap conditions C_\theta:

    - Trap implications:
                s1
             ----------
    p1    => p2 \/ bit1
    ^           ^      ^
    |           |      |
    |           |     bit1 \in S
    |           |
    |          p2 \in S
    |
    p1 \in S
    
    p2    => p3 \/ nbit2
    p3    => p1 \/ nbit1
    bit1  => (p1 \/ nbit1) /\ (q3 \/ bit1)
    nbit1 => (p2 \/ bit1) /\ (q1 \/ nbit1) /\ (q5 \/ nbit1)
    q1    => q2
    q2    => (q3 \/ bit1) /\ (q5 \/ nbit1)
    q3    => q4 \/ nbit2
    q4    => q1 \/ nbit1
    q5    => q1 \/ nbit2
    nbit2 => q2 /\ (p3 \/ nbit2)

    - At least one of the initially marked places belongs to S:
    p1 \/ q1 \/ nbit1 \/ nbit2

    - None of the marked places in A belongs to S:
    ~p3 /\ ~q5 /\ ~bit1
  
  - Assignment A_\theta:
    p1    = false
    p2    = true
    p3    = false
    bit1  = false
    nbit1 = true
    q1    = false
    q2    = true
    q3    = true
    q4    = false
    q5    = false
    nbit2 = true

  - Refinement constraint \delta:
    p2 + q2 + q3 + nbit1 + nbit2 >= 1
    ^    ^    ^    ^       ^
    |    |    |    |       |
    ------------------------
      |
      S = {p2, q2, q3, nbit1, nbit2}, therefore \delta excludes
      assignment A in the next iteration
\end{verbatim}

\newpage

\begin{verbatim}

* Trap implication:

  place s \in S =>    /\       \    /
                     /  \       \  /     place p \in S
                    /    \       \/
                   t \in s*   p \in t*   


  "if s is in trap S then for each output transition t at least one successor p is in trap S"


* Refinement constraint \delta:

  \Sigma s >= 1  
  A(s)

  "At least one place in S is always marked"
\end{verbatim}
