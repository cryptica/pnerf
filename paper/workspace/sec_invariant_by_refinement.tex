\section{Constructing Invariants from Constraints}
\label{sec_method_invariant_by_refinement}


We now show that one can compute inductive invariants from the method \safetyref.
That is, given a Petri net $N$ and a property $\varphi$, if \safetyref\ (over the rationals)
can prove $N$ satisfies $\varphi$, then in fact we can construct a linear inductive invariant
that contains $m_0$ and does not intersect $\lnot\varphi$. We call the
new method \invariantref.

The key observation is to use a constraint system dual to the constraint system for \safetyref.
%
% We make a simple observation which, combined with the flexibility of
% SMT-solvers, leads to an interesting improvement on the method \safetyref.
% We propose method \invariantref\ with the same power as the
% method \safetyref\ over the rational numbers, but dual to it: the
% constraints of method \safetyref\ are infeasible over the rational
% numbers if{}f the constraints of method \invariantref\ are
% feasible. 
% The advantage is that the solution delivered by the
% method \invariantref\ is an inductive linear invariant satisfied by all
% reachable markings, but violated by all markings that do not satisfy
% the property. 
% So this invariant provides an {\em explanation} of why
% the property holds.
We assume the negation of property $\varphi$ is represented as the constraints:
$$ \neg \varphi\  :: \ A M \ge b $$
for some matrix $A$ and vector $b$.
% consider the matrix form of the trap constraints $C_T$ below
% $$ C_T = ( D M \ge 1 ) $$
% for some matrix $D$.
Consider the following primal system $\mathcal{S}$:
\begin{align*}
  \mathcal{C}&(P, T, F, m_0)  && \text{marking constraints} \\
  A M & \ge b                 && \text{negation of property $\varphi$} \\
  D M & \ge 1                 && \text{trap constraints}
\end{align*}
%
Note that the trap constraints~\eqref{eq:trapcons} are collected in matrix form $D M \geq 1$.
We have the following theorem. (The proof is in Appendix A.)

\begin{theorem}
\label{thm:dual}
The system $\mathcal{S}'$ of constraints given by
\begin{align*}
  \lambda C    & \le 0            && \text{inductivity constraint} \\
  \lambda m_0  & <   Y_1 b + Y_2 1 && \text{safety constraint} \\
  \lambda      & \ge Y_1 A + Y_2 D && \text{property constraint} \\
  Y_1, Y_2     & \ge 0     && \text{non-negativity constraint} 
\end{align*}
is feasible over the rational numbers if{}f the system $\mathcal{S}$ is infeasible
over the rational numbers.
\end{theorem}


So $\mathcal{S}'$ is satisfiable if{}f $\mathcal{S}$ is unsatisfiable. We can
interpret the meaning of a solution $\lambda$. Recall that
for every reachable marking $m$ we find a solution of 
$m = m_0 + CX$. Multiplying by $\lambda$, and since $\lambda$ is a solution
of $\mathcal{S}'$, we get 
$$\lambda m = \lambda m_0 + \lambda CX \leq \lambda m_0\,.$$
Furthermore, any reachable marking will satisfy the trap constraints $DM\geq 1$.
On the other hand, for every marking $m$ violating the property
and satisfying the trap constraints, 
we have $ Am \geq b$ and $ Dm \geq 1$, and so
$$\lambda m \geq (Y_1 A + Y_2 D)m = Y_1 A m + Y_2 D m \geq
Y_1 b + Y_2 1 > \lambda m_0\,.$$
%
Therefore, the linear constraint
$$I(M)\  :: \  DM\geq 1 \land (\lambda M \leq \lambda m_0)$$
is an invariant satisfied by all reachable markings $m$, 
but by no marking violating the property.

We can now prove that $I(M)$ is an inductive invariant, that is, if
$I(m)$ holds for some marking $m$ (reachable or not), and $m \xrightarrow{t} m'$
for some transition $t$, then $I(m')$ holds as well.
Indeed, in this case we have $m' = m + C e_t$ where
$e_t$ is the unit vector which has a $1$ in the $t$th component and $0$ everywhere else.
% $e_t = (0,\ldots,0,\underbrace{1}_{t},0,\ldots,0)^T$ is the vector whose
% components are all $0$ but the one corresponding to $t$, which has value $1$.
So we get $$\lambda m' = \lambda(m + Ce_t) = \lambda m + \lambda Ce_t
\le \lambda m \le \lambda m_0\,,$$
and furthermore, as $m$ marks all traps in the trap constraints, 
$m'$ also marks all traps, so $I(m')$ holds.

% \begin{enumerate}
% \item The constraints given by \safetyref\ are unsat if{}f the
%   constraints by \invariantref\ are satisfiable.
% \item If the constraints by \invariantref\ are sat, then there is some
%   inductive invariant.
% \end{enumerate}

% \\ \mbox{ } \\
% \\ \mbox{ } \\
% \\ \mbox{ } \\
% \\ \mbox{ } \\

% Note we can express a set of trap constraints as a linear equation
% of the form $DM \ge 1$, where $D$ encodes the places in each trap.
% We can combine these with the negation of the property $AM \ge b$
% to obtain a new constraint $A'M \ge b'$, with
% $A' = (D A)^T$ and $b' = (1 b)^T$.
% The new constraint is then equivalent to $DM \ge 1 \land AM \ge b$.

% We can use this constraint instead of the negation of the property
% to obtain a linear constraint $I(M)$ with method Invariant.
% However, to obtain an invariant which only excludes violations of
% the property, we have to add back the trap constraints.
% By this, we obtain the invariant

% $$I'(M) := DM \ge 1 \land I(M)$$

% The invariant $I'(M)$ holds for all reachable markings, as the traps are marked
% initially and therefore also marked in all reachable markings and
% $I(M)$ holds for all reachable markings. 

% Let $M$ be a marking $M$ violating the property, i.e. $AM \ge b$.
% If $DM \ge 1$ does not hold, then $I'(M)$ does not hold.
% Otherwise, $DM \ge 1 \land AM \ge b$ holds and by the
% the properties of $I(M)$ also $I(M)$ and $I'(M)$ do not hold.

% The invariant $I'(M)$ is inductive, as traps are inductive and $I(M)$ is inductive.

\subsection{Example}

Consider again Lamport's algorithm and the mutual
exclusion property. Recall that the negation of the property for this example is $p_3 \ge 1 \land q_5 \ge 1$, and the trap constraint is
$$p_2 + q_2 + q_3 + notbit_1 + notbit_2 \ge 1\,.$$
%
Method \invariantref{} constructs the following constraints: 

\smallskip
\noindent \emph{Inductivity constraints}
$$
\begin{array}[t]{clclclclclcl}
    -  & p_1 &  +  & p_2 &     &     &  +  & bit_1 &  -  & notbit_1 &  \le  & 0 \\
       &     &  -  & p_2 &  +  & p_3 &     &       &     &          &  \le  & 0 \\
       & p_1 &     &     &  -  & p_3 &  -  & bit_1 &  +  & notbit_1 &  \le  & 0
\end{array}\quad
\begin{array}[t]{clclclclclclclclclcl}
    -  & q_1 &  +  & q_2 &     &     &     &     &     &     &  -  & notbit_2 &  \le  & 0  \\
       &     &  -  & q_2 &  +  & q_3 &     &     &     &     &     &          &  \le  & 0  \\
       &     &     &     &  -  & q_3 &  +  & q_4 &     &     &  +  & notbit_2 &  \le  & 0  \\
       & q_1 &     &     &     &     &  -  & q_4 &     &     &     &          &  \le  & 0  \\
       &     &  -  & q_2 &     &     &     &     &  +  & q_5 &     &          &  \le  & 0  \\
       & q_1 &     &     &     &     &     &     &  -  & q_5 &  +  & notbit_2 &  \le  & 0 
\end{array}
$$
\emph{Safety constraint}
$$    
p_1 + q_1 + \neg bit_1 + \neg bit_2 < target_1 + target_2 + trap_1
$$
\emph{Property constraints}
$$
\begin{array}[t]{rcl}
  p_1   & \ge & 0 \\
  p_2   & \ge & trap_1 \\
  p_3   & \ge & target_1 \\
  bit_1 & \ge & 0 \\
  notbit_1 & \ge & 0
\end{array}
\qquad
\begin{array}[t]{rcl}
  bit_1 & \ge & 0 \\
  notbit_1 & \ge & trap_1
\end{array}
\qquad
\begin{array}[t]{rcl}
  notbit_2 & \ge & trap_1
\end{array}
\qquad
\begin{array}[t]{rcl}
  q_1   & \ge & 0 \\
  q_2   & \ge & trap_1 \\
  q_3   & \ge & trap_1 \\
  q_4   & \ge & 0 \\
  q_5   & \ge & target_2 \\
\end{array}
$$
\emph{Non-negativity constraints}
$$
\begin{array}[t]{rcl}
  target_1, target_2, trap_1  & \ge & 0 
%  target_2 & \ge & 0 \\
% trap_1   & \ge & 0
\end{array}
$$
%
where $\lambda$ corresponds to the places,
$Y_1 = (target_1\ target_2)$ corresponds to the two parts of the property, and
$Y_2 = (trap_1)$ corresponds to the trap.

The SMT solver yields the following solution
$$
\begin{array}[t]{rcl}
  p_1      & = & 1 \\
  p_2      & = & 2 \\
  p_3      & = & 2 \\
  bit_1    & = & 0 \\
  notbit_1 & = & 0
\end{array}
\qquad
\begin{array}[t]{rcl}
  bit_1    & = & 0 \\
  notbit_1 & = & 1
\end{array}
\qquad
\begin{array}[t]{rcl}
  notbit_2 & = & 1
\end{array}
\qquad
\begin{array}[t]{rcl}
  q_1   & = & 0 \\
  q_2   & = & 1 \\
  q_3   & = & 1 \\
  q_4   & = & 0 \\
  q_5   & = & 1 \\
\end{array}
\begin{array}[t]{rcl}
  target_1 & = & 2 \\
  target_2 & = & 1 \\
  trap_1   & = & 1
\end{array}
$$
\noindent which leads to the invariant
\begin{align}\label{eq:invariant}
& (p_2 + q_2 + q_3 + notbit_1 + notbit_2 \ge 1 ) \; \land \nonumber \\
& (p_1 + 2 p_2 + 2 p_3 + notbit_1 + notbit_2 + q_2 + q_3 + q_5 \le 3)\,.
\end{align}
\noindent This ``explains'' why the mutual exclusion property holds: every
reachable marking satisfies this invariant, but no marking putting at least 
a token in $p_3$ and $q_5$ does. 
However, it is not the ``simplest'' explanation; we show next how to simplify invariants.
% The next section shows how to get a simpler one.

\ifthenelse{\equal{\isDraft}{true}}{\begin{figure}
  \begin{tikzpicture}[
    every path/.style={draw, ->, >=stealth, shorten >=2pt, shorten <=2pt}
    ]
    \node[state] (begin) {BEGIN};
    \node[action, below=of begin] (c) {$C:=\mathcal C(N)$\\
      $D:=\{\}$};
    \node[decision, below=of c] (satc) {$\text{SAT}(C \cup \{\neg P\})$};
    \node[action, below=of satc] (modelc) {$A:=\text{Model}(C \cup \{\neg P\})$\\
      $C_{\theta}:=\text{TrapConditions}(N, A)$};
    \node[decision, below=of modelc] (satctheta) {$\text{SAT}(C_\theta)$};
    \node[action, below=of satctheta] (modelctheta) {$A_\theta:=\text{Model}(C_\theta)$\\
      $\delta:=\Delta(A_\theta)$\\
      $C:=C \cup \{\delta\}$\\
      $D:=D \cup \{\delta\}$};
    \node[print, right=of satctheta] (noinv2) {No Invariant};
    \node[state, below=of noinv2] (end3) {END};
    
    \node[action, right=of satc] (cprime) {$C':=\mathcal C'(N, \neg P \land D)$};
    \node[decision, right=of cprime] (satcprime) {$\text{SAT}(C')$};
    \node[action, below=of satcprime] (inv) {$A':=\text{Model}(C')$\\
      $I:=D \land \text{Inv}(N, A')$};
    \node[print, below=of inv] (printinv) {Invariant: $I$};
    \node[print, right=of satcprime] (noinv) {No Invariant};
    \node[state, below=of noinv] (end1) {END};
    \node[state, below=of printinv] (end2) {END};
    
    \draw (cprime) edge (satcprime);
    \draw (satcprime) edge node[above]{NO} (noinv);
    \draw (satcprime) edge node[right]{YES} (inv);
    \draw (inv) edge (printinv);
    \draw (noinv) edge (end1);
    \draw (printinv) edge (end2);
    
    \draw (begin) edge (c);
    \draw (c) edge coordinate[pos=.5] (edgein) (satc);
    \draw (satc) edge node[above]{NO} (cprime);
    \draw (satc) edge node[right]{YES} (modelc);
    \draw (modelc) edge (satctheta);
    \draw (satctheta) edge node[above]{NO} (noinv2);
    \draw (noinv2) edge (end3);
    \draw (satctheta) edge node[right]{YES} (modelctheta);
    \draw (modelctheta.south) -- ([yshift=-0.5cm] modelctheta.south)
    -| ([xshift=-2cm] modelctheta.west) |- (edgein);
  \end{tikzpicture}
  \caption{Diagram for Function Invariant by Refinement}
  \label{fig_function_invariant_by_refinement_diagram}
\end{figure}

%%% Local Variables: 
%%% mode: latex
%%% TeX-master: "main"
%%% End: 
}{}

% \begin{figure}
  \begin{algorithmic}[1]
    \State $C := \mathcal C(N)$
    \State $D := $true
    \While {SAT$(C \cup \{\neg P\})$}
    \State $A := $Model$(C \cup \{\neg P\})$
    \State $C_\theta := $TrapConditions$(N, A)$
    \If {SAT$(C_\theta)$}
    \State $A_\theta := $Model$(C_\theta)$
    \State $\delta := \Delta(A_\theta)$
    \State $C := C \cup \{\delta\}$
    \State $D := D \land \delta$
    \Else
    \State \Return 'Failed at finding an invariant'
    \EndIf
    \EndWhile
    \State $C' := \mathcal C'(N, \neg P \land D)$
    \If {SAT$(C')$}
    \State $I := D \land $Inv$(N, A')$
    \State \Return 'Invariant I for the petri net: $I$'
    \Else
    \State \Return 'Failed at finding an invariant'
    \EndIf
  \end{algorithmic}
  \caption{Pseudocode for Method Invariant by Refinement}
  \label{fig:method-invariant-by-refinement-pseudocode}
\end{figure}



%%% Local Variables: 
%%% mode: latex
%%% TeX-master: "main"
%%% End: 
