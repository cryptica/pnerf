\section{Preliminaries}
\label{sec-preliminaries}

\begin{figure}[t]
  \begin{minipage}{.385\columnwidth}
   \begin{minipage}{\columnwidth}
  \footnotesize
  \centering
   \begin{tabular}{@{}r l l@{\;}l@{}}
     % \patts & $\ni \patt ::=$ & 
     % & pattern \\[\jot]
     % & & $|$ \ \var  & variable \\[\jot]
     % & & $|$ \ \ttt{\patt, \ldots, \patt}
      % & tuple pattern \\[4\jot]
      % \exprs & $\ni \expr ::=$ & 
      % & expression \\[\jot]
%%      \exprs & $\ni \expr ::=$ &\ \ \cst 
 &&      \exprs\  $\ni \expr ::=$ \ \ \cst 
      & \myexp{constant} \\[\jot]
     & & $|$ \var 
      & \myexp{variable} \\[\jot]
     % & & $|$ \ttt{\expr, \ldots, \expr}
     %  & tuple expression \\[\jot]
     & & $|$ \ttt{\cstr(\expr, \ldots, \expr)} 
      & \myexp{construction} \\[\jot]
     & & $|$ \expr\ \expr 
      & \myexp{application} \\[\jot]
     & & $|$ \ttt{fun \var\ -> \expr}
      & \myexp{abstraction} \\[\jot]
     & & $|$ \ttt{let \patt\ = \expr\ in \expr}
      & \myexp{let binding} \\[\jot]
     & & $|$ \ttt{let \exprRec \var\,=\,fun \var\,->\,\expr\,in \expr\
     }
      & \myexp{let-rec binding} \\[\jot]
     & & $|$
     \ttt{match \expr\ with}  & \myexp{match-with} \\
     & & \ttt{\tabT | \ttt{\cstr(\var, \ldots, \var)}\ -> \expr} & \\
     & & \ttt{\tabT \ldots} & \\
     & & \ttt{\tabT | \ttt{\cstr(\var, \ldots, \var)}\ -> \expr}
  \end{tabular}

  \centering
     (a) 
\end{minipage}
%%% Local Variables: 
%%% mode: latex
%%% TeX-master: "main"
%%% End: 
 
    \vspace*{1ex}

   \begin{minipage}{\columnwidth}
  \small
  \centering
  \begin{tabular}{@{}r@{}l@{}l@{\;}l@{}}
%    & $\type ::=$ &    & type \\[\jot]
    & & $\type ::=$\ \tybase  & \myexp{base type} \\[\jot]
    & & $|$ \ \tyvar  & \myexp{type variable} \\[\jot]
    & & $|$ \ \ttt{\type\ -> \type}  & \myexp{function type} \\[\jot]
%    & & $|$ \ \ttt{\type\ *\ \ldots\ *\ \type}  & tuple type \\[\jot]
    & & $|$ \ \ttt{(\type,\ \ldots,\ \type)\ \tycstr\ }  & \myexp{variant type}
  \end{tabular}

  \centering
  (b)
\end{minipage}

%%% Local Variables: 
%%% mode: latex
%%% TeX-master: "main"
%%% End: 

    \vspace*{1ex}

    \begin{minipage}[t]{\columnwidth}
  \small
  \centering
  \begin{tabular}{r l l}
    % \vals $\ni \val$ ::= &
    % & value \\[\jot]
%\vals $\ni \val$ ::=    & \ \ \ \cst
    & \vals\ $\ni \val$ ::=\ \cst \myexp{constant} \\[\jot]
    % & $|$ \ \ensuremath{\val,\ \ldots,\ \val}
    % & tuple value \\[\jot]
    & $|$ \ \ensuremath{\cstr(\val,\ \ldots,\ \val)}
    & \myexp{construction} \\[\jot]
    & $|$ \ \ensuremath{(\ttt{fun \var\ -> \expr},\ \ectx)}
    & \myexp{regular closure} \\[\jot]
    & $|$ \ \ensuremath{(\var, \ttt{fun \var\ -> \expr}, \ectx)}
    & \myexp{recursive closure}  %\\[\jot]
    % \evjs  $\ni \evjvar ::= $ & \evj{\ectx}{\expr}{\val}  & \myexp{judgement}
    % \\[\jot]
    % $2^{\evjs^+} \ni \evt$ \mbox{\quad\;\;}  & & \myexp{tree}
  \end{tabular}

  \centering
    (c) 
\end{minipage}
%%% Local Variables: 
%%% mode: latex
%%% TeX-master: "main"
%%% End: 

  \end{minipage}
  \begin{minipage}{.65\columnwidth}
%    \vspace{-27ex}
%    \vspace{-20ex}
    %  \tiny
  \scalebox{1}{
    \begin{mathpar}
      \ercst \and
      \ervar \and
      % \ertup \and
      \ercstr \and
      \erappcst \and
      \erapp \and
      \erapprec \and
      \erlet \and
      \erfun \and
      \erletrec \and
      \ermatch
    \end{mathpar}
  }
   \begin{center}
     (d)
   \end{center}
%%% Local Variables:
%%% mode: latex
%%% TeX-master: "main"
%%% End: 

  \end{minipage}
  % \end{tabular}
  \caption{\miniocaml\  syntax (a), types (b), and evaluation rules (c).
    Rule premises are ordered from left to right and from top to bottom.
    Patterns in \smatch\ are ordered from top to bottom.        
  }
  \label{fig-syn-and-sem}
\end{figure}
%%% Local Variables:
%%% mode: latex
%%% TeX-master: "main"
%%% End: 


In this section we describe \miniocaml, the programming language that
we use to represent programs. We also present the logger
monad~\cite{freetheorems} extended with an update operation.

\paragraph{\bf \miniocaml syntax}

Let \vars\ be a set of \emph{variables}, e.g., \ttt{x},
\ttt{List.map}, and~\ttt{myVar}.  Let \csts\ be a set of
\emph{constants}, e.g., \ttt{+}, \ttt{1}, \ttt{2.5}, and \ttt{"h"}.
Let \cstrs\ be a set of \emph{constructors}, e.g., \ttt{::},
\ttt{Some}, \ttt{[]}, and \ttt{true}.  We assume $\vst, \var \in
\vars$, $\cst\in \csts$, and $\cstr\in\cstrs$.
Figure~\ref{fig-syn-and-sem}(a) presents the syntax of \miniocaml
expressions. 
We encode \ttt{if-then-else} expressions using match expressions. 
Function applications are left associative. 
We assume that tuples of values are encoded using tuple constructors.
We assume that text in \ttt{type writer font} is \miniocaml code.

\paragraph{\bf \miniocaml types}

We use the \ocaml\ type system~\cite{LeroyPhD92} to type \miniocaml\
expressions. Let \basetypes\ be a set of base types, e.g., \ttt{int},
\ttt{string}, and \ttt{in\_channel}. Let \typevars\ be a set of type
variables, e.g., \ttt{'a} and \ttt{'b}. Let \typecstrs\ be a set of
type constructors, e.g., \ttt{list}, \ttt{option}, and \ttt{bool}.  We assume
$\tybase \in \basetypes$, $\tyvar \in \typevars$, and $\tycstr \in
\typecstrs$.  Figure~\ref{fig-syn-and-sem}(b) presents the set of \miniocaml\
types.  We write the typing proposition $\expr~:~\type$ if expression
\expr\ is of type \type\ under some typing context. We say that expression \expr\ is well-typed
if there exists a type \type\ such that $\expr~:~\type$. Examples of
valid propositions are $ \ttt{1}~:~\ttt{int} $ and $
\ttt{+}~:~\ttt{int~->~int~->~int} $.


\paragraph{\bf \miniocaml semantics}

Figure~\ref{fig-syn-and-sem}(c) presents  \emph{values} computed by \miniocaml
programs using judgements $\evjvar$ of the form
%
\begin{equation*}
  \evjs  \ni \evjvar ::=  \evj{\ectx}{\expr}{\val}
\end{equation*} 
%
Every judgement is derived by applying rules shown in
Figure~\ref{fig-syn-and-sem}(d).
A judgement \evjvar\ is \emph{valid} if there exists an evaluation
tree with \evjvar\ as the root.
Each evaluation tree is given by the set of its edges.
Each edge is a sequence of judgements $\evjvar[1], \dots, \evjvar[n],
\evjvar$, where $\evjvar[1], \dots, \evjvar[n]$ are the predecessor
nodes and \evjvar\ is the successor node.
If $n=0$ then the edge represents a leaf node. 

For example, we consider the evaluation tree \evt\ for \ttt{1 + 2} as
shown below.
\begin{center}
  \small
  \[
  \inference{ \inference{}{ \evj{\emptyectx}{\ttt{2}}{\ttt{2}} } &
    \inference{ \inference{}{ \evj{\emptyectx}{\ttt{1}}{\ttt{1}} } &
      \inference{}{ \evj{\emptyectx}{\ttt{+}}{\ttt{+}} } }{
      \evj{\emptyectx}{\ttt{(+) 1}}{\ttt{+}_1} } }{
    \evj{\emptyectx}{\ttt{1 + 2}}{\ttt{3}} }
  \]
\end{center}
Let
$\evjvar[1]~=~(\evj{\emptyectx}{\ttt{1+2}}{\ttt{3}})$, 
$\evjvar[2]~=~(\evj{\emptyectx}{\ttt{2}}{\ttt{2}})$, 
$\evjvar[3]~=~(\evj{\emptyectx}{\ttt{(+)~1}}{\ttt{+}_1})$, 
$\evjvar[4]~=~(\evj{\emptyectx}{\ttt{1}}{\ttt{1}})$, 
$\evjvar[5]~=~(\evj{\emptyectx}{\ttt{+}}{\ttt{+}})$.
The evaluation tree \evt\ is given by the set of five edges below.
%
\[
\evt~=~\{ \evjvar[2], \evjvar[4], \evjvar[5], (\evjvar[4],
\evjvar[5], \evjvar[3]), (\evjvar[2], \evjvar[3], \evjvar[1]) \}
\]
%
Let $\tsf{eval}\ \ectx\ \expr$ be the value of expression \expr\ in
the environment \ectx, i.e., $v = \tsf{eval}\ \ectx\ \expr$
if there is an evaluation tree for \evj{\ectx}{\expr}{\val}.

\paragraph{\bf Recursion relations and binary reachability}

We are interested in keeping track of (possibly partial) applications
of a function defined in a program.
Let $f$ be a function identifier of type $\type[1] \rightarrow \ldots
\rightarrow \type[m] \rightarrow \type$ that is bound using a let-rec
binding, and $\theArity$ be a number between $1$ and $m$.  
An \emph{$\theFunction/\theArity$-application judgement} describes evaluation of
an application of $f$ to $\theArity$-many actual parameters, i.e., it
is a judgement of the form~\evj{\ectx}{f\ \expr[1]\ \ldots\
\expr[n]}{\val}.
A \emph{$\theFunction/\theArity$-recursion relation} consists of pairs
of value tuples $(v_1, \dots, v_\theArity)$ and~$(u_1, \dots,
u_\theArity)$ that satisfy the following condition.
For each $(v_1, \dots, v_\theArity)$ and~$(u_1, \dots, u_\theArity)$
in the relation we require existence of a pair of valid
$\theFunction/\theArity$-application judgements $\evjvar[1] =
\evj{\ectx[1]}{f\ \expr[1]\ \ldots\ \expr[\theArity]}{\val[]}$ and
$\evjvar[2] = \evj{\ectx[2]}{f\ \expr[1]\ \ldots\
\expr[\theArity]}{u}$ such that $\evjvar[2]$ appears in the evaluation
tree of $\evjvar[1]$ and for each $i \in 1..\theArity$ we have
$\tsf{eval}\ \ectx[1]\ \expr[i] = \val[i]$ and $\tsf{eval}\ \ectx[2]\
\expr[i] = u_i$.

The goal of \emph{binary reachability analysis} is to check if the
$\theFunction/\theArity$-recursion relation of the program is
contained in a given binary relation.

We fix $\theFunction$, $\theArity$, and a binary relation~$\theTI$ for
the rest of this paper.
Furthermore, we assume that $\theTI$ is represented as an assertion
over tuples of variables $(\mathtt{a}_1, \dots, \mathtt{a}_\theArity)$
and $(\mathtt{m}_1, \dots, \mathtt{m}_\theArity)$.
This assumption will be used in Figure~\ref{fig-tsel}.




%%% Local Variables: 
%%% mode: latex
%%% TeX-master: "main"
%%% End: 
