\paragraph{\bf Logger monad}
A \emph{monad} consists of a type constructor \ttt{m} of arity~1 and two
operations
%
\begin{small}
\begin{align*}
  \ttt{unit~}&\ttt{:~'a~->~'a~m} \\
  \ttt{(~>>=~)~}&\ttt{:~'a~m~->~('a~->~'b~m)~->~'b~m}
\end{align*}
\end{small} 
%
These operations need to satisfy three conditions called \emph{left
unit}, \emph{right unit}, and \emph{associative}~\cite{Wadler95}.  
We assume that \ttt{state} is a given type.  
Figure~\ref{fig-state-monad} presents a variant of the \emph{logger
monad}~\cite{freetheorems}. The state
update operator \ttt{update} is of type
\ttt{(state~->~state)~->~unit~m}. A monadic expression (resp. value)
is an expression (resp. value) of the logger monad type. 

For example, the monadic expression \ttt{unit 1} evaluates to a
function that takes a state \st\ and returns a pair (\ttt{1}, \st). As
another example, the following monadic expression evaluates to a
function that takes a state \st\ and returns a pair (\ttt{1}, \st\ $+$
\ttt{1}).
\begin{center}
  \ttt{update (fun s -> s + 1) >>= fun () -> unit \ttt{1}}
\end{center}

\begin{figure}[t]
  \linespread{1}
  \begin{minipage}[t]{\columnwidth}
    \small
   \begin{tabular}{l@{\ }l}
     1 & \ttt{(* logger monad type *)} \\[\jot]
     2 & \ttt{type 'a m = state -> 'a * state } \\[\jot]
     3 & \ttt{(* unit operator *)} \\[\jot]
     4 & \ttt{let unit a = fun s -> (a, s) } \\[\jot]
     5 & \ttt{(* bind operator *)} \\[\jot]
     6 & \ttt{let ( >>= ) m k = fun s0 -> } \\[\jot]
     7 & \ttt{\tabTTTTTTTT let v1, s1 = m s0 in } \\[\jot]
     8 & \ttt{\tabTTTTTTTT k v1 s1 } \\[\jot]
     9 & \ttt{(* state transform operator *)} \\[\jot]
     10 & \ttt{let update f = fun s -> ( (), f s ) }
    \end{tabular}
  \end{minipage}
  \caption{Logger monad with state transform operator
    \ttt{update}. The unit operator takes a value and constructs a
    monadic value. The bind operator takes a monadic value and a
    function returning a monadic value, and constructs a new monadic
    value. The state transform operator creates a new monadic value by
    applying the state transformer \ttt{f}.}
  \label{fig-state-monad}
\end{figure}

\iffalse
We clarify the product expressions above by expressing them in
\haskell's \emph{do notation}~\cite{DoNotation}. For
the product of expression \ttt{1} and \msplus, the corresponding
\haskell\ program is the following.
\begin{center}
  \begin{minipage}{.4\columnwidth}
   \small
    \ttt{do \\
      \tabT update (fun s -> s) \\
      \tabT update (fun s -> s) \\
      \tabT unit \ttt{1}}
 \end{minipage}
\end{center}
For the product of expression \ttt{+} and \msplus, the corresponding
\haskell\ program is the following.
\begin{center}
  \begin{minipage}{.9\columnwidth}
    \small
    \ttt{do \\
      \tabT update (fun s -> s) \\
      \tabT update (fun s -> s + 1) \\
      \tabT unit (fun x -> unit (fun y -> unit ((+)\ x\ y)))} 
 \end{minipage}
\end{center}
\fi
%%% Local Variables:
%%% mode: latex
%%% TeX-master: "main"
%%% End: 
