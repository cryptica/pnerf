\section{Illustration}

In this section we illustrate what our transformation adds to the
program in order to keep track of pairs of argument valuations for
checking a transition invariant candidate.

We consider the following curried function \texttt{f} that has a type
\texttt{x:int -> y:int -> ret:int}.
Here, we annotated the parameter and return value types with
identifiers to improve readability.
%
\begin{center}
\begin{minipage}[h]{.8\linewidth}
\begin{small}
\begin{verbatim}
let rec f x = if x > 0 then f (x-1) else fun y -> f x y
\end{verbatim}
\end{small}
\end{minipage}
\end{center}
%
This function shows that -- in contrast to proving termination of
recursive procedures in imperative programs -- it is important to
differentiate between partial and complete applications when dealing
with curried functions.
First, we observe that any partial application of \texttt{f}
terminates.
For example, \texttt{f 10} stops after ten recursive calls and returns
a function \texttt{fun y -> f 10 y} where \texttt{f} is bound to a
closure.
That is, there is no infinite sequence of \texttt{f} applications that
are passed only one argument.
In contrast, any complete application of \texttt{f} does not terminate.
For example, \texttt{f 1 1} will lead to an infinite sequence of
\texttt{f} applications such that each of them is given two arguments.

Our binary reachability analysis takes as input a specification that
determines which kind of applications we want to keep track of. 
The specification consists of a function identifier, e.g., \texttt{f},
a number of parameters, e.g., one, and a transition invariant candidate. 
Then, such a specification requires that applications of \texttt{f} to
one argument satisfy the transition invariant candidate.
Alternatively, we may focus on applications of \texttt{f} to two
arguments.

Once the specification is given, we transform \texttt{f} into a
function \texttt{f\_m} that keeps track of arguments on which \texttt{f} was
applied using additional parameters \texttt{old\_x} and~\texttt{old\_y}.
As a result, \texttt{f\_m} fulfills two requirements. 
First, it computes a result value \texttt{res} such that $\mathtt{res}
= \mathtt{f\ x\ y}$. 
Second, it computes new values of additional parameters.
If \texttt{f\_m} was an imperative program, we would obtain the type
%
\begin{small}
\begin{verbatim}
        x:int * y:int * copied:bool * old_x:int * old_y:int ->
          ret:int * new_copied:bool * new_x:int * new_y:int
\end{verbatim}
\end{small}
% 
where \texttt{new\_x} and \texttt{new\_y} are computed as follows. 
If \texttt{old\_x} already stores a value that was given to \texttt{x} in the past, i.e.,
if $\mathtt{copied} = \mathtt{true}$, then $\mathtt{new\_x} = \mathtt{old\_x}$. 
Otherwise, \texttt{f\_m} can nondeterministically either store
\texttt{x} in \texttt{new\_x} and set $\mathtt{new\_copied} =
\mathtt{true}$, or leave $\mathtt{new\_x} = \mathtt{old\_x}$
and~$\mathtt{new\_copied} = \mathtt{false}$.
The computation of \texttt{new\_y} is similar.
Given a transformed program, checking binary reachability amounts to
checking that at each application site the pair of tuples
$(\mathtt{old\_x}, \mathtt{old\_y})$ and $(\mathtt{x}, \mathtt{y})$
satisfies the transition invariant whenever \texttt{copied}
is~\texttt{true}.

Due to partial applications we cannot expect that values of \texttt{x}
and \texttt{y} are provided simultaneously, which complicates both
computation of \texttt{new\_x} and \texttt{new\_y} and checking if
$(\mathtt{old\_x}, \mathtt{old\_y})$ together with $(\mathtt{x},
\mathtt{y})$ satisfy the transition invariant. 
Hence, we need to keep track of arguments as they are provided, which
requires ``waiting'' for missing arguments.
We implement this waiting process by introducing additional parameters
\texttt{old\_state\_x} and \texttt{old\_state\_y} for each partial
application, together with their updated versions
\texttt{new\_state\_x} and~\texttt{new\_state\_y}.
Each additional parameter accumulates arguments in its first
component, and it keeps a tuple of previously provided arguments in
its second component. 
We obtain the following type for \texttt{f\_m}.
%
\begin{small}
\begin{verbatim}
x:int 
-> old_state_x:((int * int) *                (* accumulate x and y     *)
                (bool * int * int))          (* store copied, x, and y *)
-> (y:int 
    -> old_state_y:((int * int) *            (* accumulate x and y     *)    
                    (bool * int * int))      (* store copied, x, and y *)
    -> ret:int * new_state_y:((int * int) * 
                              (bool * int * int))) *
   new_state_x:((int * int) * 
                (bool * int * int))
\end{verbatim}
\end{small}
% 
We refer to \texttt{(int * int) * (bool * int * int)} as~\texttt{state}. 
Then \texttt{f\_m} has the type:
%
\begin{small}
\begin{verbatim}
      x:int -> old_state_x:state -> 
        (y:int -> old_state_y:state -> ret:int * new_state_y:state) *
        new_state_x:state
\end{verbatim}
\end{small}
% 
We formalize the above transformation in
Section~\ref{sec-transformation}. 
Figure~\ref{fig-ex-prod} presents a detailed execution protocol of
applying our transformation on the above program.

\iffalse
When dealing with unary applications, the binary reachability
analysis can show that every pair of unary applications is included
in a transition invariant using the following check.
\fi

Note that if complete applications of a function terminate, then every
partial application of the function terminates.
For example, consider the following function~\texttt{g}.
%
\begin{center}
\begin{verbatim}
let rec g x = if x > 0 then g (x-1) else fun y -> x+y
\end{verbatim}
\end{center}
%
This function does not have any infinite application sequences neither
for complete nor for partial applications.


%%% Local Variables: 
%%% mode: latex
%%% TeX-master: "main"
%%% End: 
