\section{Experimental evaluation}
\label{sec-experiments}

In this section we describe our implementation and the corresponding
experimental evaluation.

\paragraph{Implementation.}
We implemented \product\ as an extension to the \camlp\
parser~\cite{Camlp4-dist}. Our implementation takes as input a user
program and a specification consisting of a function name, an arity,
and candidate transition invariant. Our implementation produces a
transformed program following the procedure depicted in
Figure~\ref{fig-ex-prod}.

\paragraph{Experiments.}
\begin{figure}[t]
  \scalebox{.8}{
    \begin{minipage}{\columnwidth}
      \small
      % \begin{tabular}{| c | c | c | c | c | c | c | c | c | c | c | c |}
      % \begin{tabular}{| c | c | c | c | c | c | c | c | c |}
      \begin{tabular}{ c | c | l }
        \# &
        Name &
        Description % &
        % Class %  &
        % Monitor Specification &
        % Harness &
        % % Exit Code &
        % Wall Time &
        % % System Time &
        % % User Time &
        % LOC &
        % Source
        \\ \hline

        % 1
%  & \verb,rev_aux, %% Name
%  & List reversal helper function. %% Desc 
%  & \sflinear %% SpecDesc 
% %%  & conses\_count\_st\_man.cmo %% MonSpec 
% %%  & \verb?pos = pre + mlength l? %% Harnesses 
% % &
% %% 0:08.11 %% RunInfo 
% %%  & 6
% %%  & \cite{Hofmann00} %% Source 
%  \\
% 2
%  & \verb,reverse, %% Name
%  & List reversal function. %% Desc 
%  & \sflinear %% SpecDesc 
% %%  & conses\_count\_st\_man.cmo %% MonSpec 
% %%  & \verb?pos = pre + mlength l? %% Harnesses 
% % &
% %% 0:07.54 %% RunInfo 
% %%  & 8
% %%  & \cite{Hofmann00} %% Source 
%  \\
% 3
%  & \verb,ins, %% Name
%  & Ordered insert function %% Desc 
%  & \sflinear %% SpecDesc 
% %%  & conses\_count\_st\_man.cmo %% MonSpec 
% %%  & \verb?pos <= pre + 1 + mlength l? %% Harnesses 
% % &
% %% 0:14.78 %% RunInfo 
% %%  & 7
% %%  & \cite{Hofmann00} %% Source 
%  \\
% 4
%  & \verb,clone, %% Name
%  & Create a pair of copies of a list. %% Desc 
%  & \sflinear %% SpecDesc 
% %%  & conses\_count\_st\_man.cmo %% MonSpec 
% %%  & \verb?pos = pre + mlength l + mlength l? %% Harnesses 
% % &
% %% 1:04.69 %% RunInfo 
% %%  & 9
% %%  & \cite{Hofmann03} %% Source 
%  \\
% 5
%  & \verb,tpo, %% Name
%  & Insert an element in every third position of a list. %% Desc 
%  & \sflinear %% SpecDesc 
% %%  & conses\_count\_st\_man.cmo %% MonSpec 
% %%  & \verb?pos <= pre + mlength l + mlength l? %% Harnesses 
% % &
% %% 0:20.50 %% RunInfo 
% %%  & 8
% %%  & \cite{Hofmann03} %% Source 
%  \\
% 6
%  & \verb,sec, %% Name
%  & Remove every third element from a list. %% Desc 
%  & \sflinear %% SpecDesc 
% %%  & conses\_count\_st\_man.cmo %% MonSpec 
% %%  & \verb?pos <= pre + mlength l? %% Harnesses 
% % &
% %% 0:31.61 %% RunInfo 
% %%  & 10
% %%  & \cite{Hofmann03} %% Source 
%  \\
% 7
%  & \verb,tos, %% Name
%  & Replace each third element of a list. %% Desc 
%  & \sflinear %% SpecDesc 
% %%  & conses\_count\_st\_man.cmo %% MonSpec 
% %%  & \verb?pos <= pre + mlength l + mlength l + mlength l? %% Harnesses 
% % &
% %% 1:18.71 %% RunInfo 
% %%  & 19
% %%  & \cite{Hofmann03} %% Source 
%  \\
% 8
%  & \verb,append, %% Name
%  & List append. %% Desc 
%  & \sflinear %% SpecDesc 
% %%  & conses\_count\_st\_man.cmo %% MonSpec 
% %%  & \verb?pos = pre + mlength l1? %% Harnesses 
% % &
% %% 0:06.99 %% RunInfo 
% %%  & 4
% %%  & \cite{Hofmann00} %% Source 
%  \\
% 9
%  & \verb,split_by, %% Name
%  & Split list. %% Desc 
%  & \sflinear %% SpecDesc 
% %%  & conses\_count\_st\_man.cmo %% MonSpec 
% %%  & \verb?pos = pre + mlength l? %% Harnesses 
% % &
% %% 0:26.02 %% RunInfo 
% %%  & 8
% %%  & \cite{Hofmann00} %% Source 
%  \\
% 10
%  & \verb,append, %% Name
%  & List append. %% Desc 
%  & \sflinear %% SpecDesc 
% %%  & mh\_heap\_consumption\_st\_man.cmo %% MonSpec 
% %%  & \verb?pos = pre? %% Harnesses 
% % &
% %% 0:07.79 %% RunInfo 
% %%  & 4
% %%  & \cite{Hofmann00} %% Source 
%  \\
% 11
%  & \verb,breadth_aux,\benchmarkontrees %% Name
%  & Breadth first tree fold auxiliary function. %% Desc 
%  & \sflinear %% SpecDesc 
% %%  & mh\_heap\_consumption\_st\_man.cmo %% MonSpec 
% %%  & \verb?pos = pre? %% Harnesses 
% % &
% %% 2:12.97 %% RunInfo 
% %%  & 15
% %%  & \cite{Hofmann00} %% Source 
%  \\
% 12
%  & \verb,breadth,\benchmarkontrees %% Name
%  & Breadth first tree fold function. %% Desc 
%  & \sflinear %% SpecDesc 
% %%  & mh\_heap\_consumption\_st\_man.cmo %% MonSpec 
% %%  & \verb?pos = pre - 1? %% Harnesses 
% % &
% %% 2:15.13 %% RunInfo 
% %%  & 16
% %%  & \cite{Hofmann00} %% Source 
%  \\
% 13
%  & \verb,app_tail,\hobenchmark %% Name
%  & Apply function to tail of list. %% Desc 
%  & \sflinear %% SpecDesc 
% %%  & mh\_heap\_consumption\_st\_man.cmo %% MonSpec 
% %%  & \verb?pos = pre? %% Harnesses 
% % &
% %% 0:20.07 %% RunInfo 
% %%  & 7
% %%  & \cite{Hofmann02} %% Source 
%  \\
% 14
%  & \verb,compose_list,\hobenchmark %% Name
%  & Hofmann's composing (or folding) all functions in a list %% Desc 
%  & \sflinear %% SpecDesc 
% %%  & mh\_heap\_consumption\_st\_man.cmo %% MonSpec 
% %%  & \verb?pos = pre + mlength l? %% Harnesses 
% % &
% %% 0:10.91 %% RunInfo 
% %%  & 6
% %%  & \cite{Hofmann02} %% Source 
%  \\
% 15
%  & \verb,processfile1,\benchmarkonfiledescriptors %% Name
%  & Open, read, and close one file. %% Desc 
%  & \sfchan %% SpecDesc 
% %%  & kobayashi\_cfu\_in\_st\_man.cmo %% MonSpec 
% %%  & \verb?Myset.eq (Myset.mns pos pre) Myset.empty? %% Harnesses 
% % &
% %% 0:26.32 %% RunInfo 
% %%  & 12
% %%  & \cite{Kobayashi09} %% Source 
%  \\
% 16
%  & \verb,processfile2,\benchmarkonfiledescriptors %% Name
%  & Open, read, and close one file, version 2. %% Desc 
%  & \sfchan %% SpecDesc 
% %%  & kobayashi\_cfu\_multiple\_st\_man.cmo %% MonSpec 
% %%  & \verb?begin ... end? %% Harnesses 
% % &
% %% 3:39.51 %% RunInfo 
% %%  & 12
% %%  & \cite{Kobayashi09} %% Source 
%  \\
% 17
%  & \verb,processtwofiles1,\benchmarkonfiledescriptors %% Name
%  & Manipulate two files at the same time. %% Desc 
%  & \sfchan %% SpecDesc 
% %%  & kobayashi\_cfu\_multiple\_st\_man.cmo %% MonSpec 
% %%  & \verb?begin ... end? %% Harnesses 
% % &
% %% 12:17.07 %% RunInfo 
% %%  & 17
% %%  & \cite{Kobayashi09} %% Source 
%  \\
% 18
%  & \verb,processtwofiles2,\benchmarkonfiledescriptors %% Name
%  & Manipulate two files at the same time, version 2. %% Desc 
%  & \sfchan %% SpecDesc 
% %%  & kobayashi\_cfu\_multiple\_st\_man.cmo %% MonSpec 
% %%  & \verb?begin ... end? %% Harnesses 
% % &
% %% 6:19.56 %% RunInfo 
% %%  & 16
% %%  & \cite{Kobayashi09} %% Source 
%  \\
% 19
%  & \verb,mem, %% Name
%  & Function \verb,mem, from list.ml. %% Desc 
%  & \sflinear %% SpecDesc 
% %%  & f\_apps\_count\_st\_man.cmo %% MonSpec 
% %%  & \verb?pos <= pre + mlength l? %% Harnesses 
% % &
% %% 0:28.46 %% RunInfo 
% %%  & 5
% %%  & stdlib/list.ml %% Source 
%  \\
% 20
%  & \verb,exists,\hobenchmark %% Name
%  & Function \verb,exists, from list.ml. %% Desc 
%  & \sflinear %% SpecDesc 
% %%  & f\_apps\_count\_st\_man.cmo %% MonSpec 
% %%  & \verb?pos <= pre + mlength l? %% Harnesses 
% % &
% %% 0:13.61 %% RunInfo 
% %%  & 4
% %%  & stdlib/list.ml %% Source 
%  \\
% 21
%  & \verb,for_all,\hobenchmark %% Name
%  & Function \verb,for_all, from list.ml. %% Desc 
%  & \sflinear %% SpecDesc 
% %%  & f\_apps\_count\_st\_man.cmo %% MonSpec 
% %%  & \verb?pos <= pre + mlength l? %% Harnesses 
% % &
% %% 0:13.55 %% RunInfo 
% %%  & 4
% %%  & stdlib/list.ml %% Source 
%  \\
% 22
%  & \verb,rev_map,\hobenchmark %% Name
%  & Function \verb,rev_map, from list.ml. %% Desc 
%  & \sflinear %% SpecDesc 
% %%  & f\_apps\_count\_st\_man.cmo %% MonSpec 
% %%  & \verb?pos = pre + mlength l? %% Harnesses 
% % &
% %% 0:15.52 %% RunInfo 
% %%  & 7
% %%  & stdlib/list.ml %% Source 
%  \\
% 23
%  & \verb,iter,\hobenchmark %% Name
%  & Function \verb,iter, from list.ml. %% Desc 
%  & \sflinear %% SpecDesc 
% %%  & f\_apps\_count\_st\_man.cmo %% MonSpec 
% %%  & \verb?pos = pre + mlength l? %% Harnesses 
% % &
% %% 0:06.99 %% RunInfo 
% %%  & 4
% %%  & stdlib/list.ml %% Source 
%  \\
% 24
%  & \verb,fold_right,\hobenchmark %% Name
%  & Function \verb,fold_right, from list.ml. %% Desc 
%  & \sflinear %% SpecDesc 
% %%  & f\_apps\_count\_st\_man.cmo %% MonSpec 
% %%  & \verb?pos = pre + mlength l? %% Harnesses 
% % &
% %% 0:07.90 %% RunInfo 
% %%  & 4
% %%  & stdlib/list.ml %% Source 
%  \\
% 25
%  & \verb,fold_left,\hobenchmark %% Name
%  & Function \verb,fold_left, from list.ml. %% Desc 
%  & \sflinear %% SpecDesc 
% %%  & f\_apps\_count\_st\_man.cmo %% MonSpec 
% %%  & \verb?pos = pre + mlength l? %% Harnesses 
% % &
% %% 0:09.48 %% RunInfo 
% %%  & 4
% %%  & stdlib/list.ml %% Source 
%  \\
% 26
%  & \verb,map,\hobenchmark %% Name
%  & Function \verb,map, from list.ml. %% Desc 
%  & \sflinear %% SpecDesc 
% %%  & conses\_count\_st\_man.cmo %% MonSpec 
% %%  & \verb?pos = pre + mlength l? %% Harnesses 
% % &
% %% 0:09.14 %% RunInfo 
% %%  & 4
% %%  & stdlib/list.ml %% Source 
%  \\
% 27
%  & \verb,memq, %% Name
%  & Function \verb,memq, from list.ml. %% Desc 
%  & \sflinear %% SpecDesc 
% %%  & f\_apps\_count\_st\_man.cmo %% MonSpec 
% %%  & \verb?pos <= pre + mlength l? %% Harnesses 
% % &
% %% 0:15.10 %% RunInfo 
% %%  & 5
% %%  & stdlib/list.ml %% Source 
%  \\
% 28
%  & \verb,mem_assoc, %% Name
%  & Function \verb,mem_assoc, from list.ml. %% Desc 
%  & \sflinear %% SpecDesc 
% %%  & f\_apps\_count\_st\_man.cmo %% MonSpec 
% %%  & \verb?pos <= pre + mlength l? %% Harnesses 
% % &
% %% 0:35.60 %% RunInfo 
% %%  & 6
% %%  & stdlib/list.ml %% Source 
%  \\
% 29
%  & \verb,mem_assq, %% Name
%  & Function \verb,mem_assq, from list.ml. %% Desc 
%  & \sflinear %% SpecDesc 
% %%  & f\_apps\_count\_st\_man.cmo %% MonSpec 
% %%  & \verb?pos <= pre + mlength l? %% Harnesses 
% % &
% %% 0:18.04 %% RunInfo 
% %%  & 6
% %%  & stdlib/list.ml %% Source 
%  \\
% 30
%  & \verb,remove_assoc, %% Name
%  & Function \verb,remove_assoc, from list.ml. %% Desc 
%  & \sflinear %% SpecDesc 
% %%  & f\_apps\_count\_st\_man.cmo %% MonSpec 
% %%  & \verb?pos <= pre + mlength l? %% Harnesses 
% % &
% %% 0:31.70 %% RunInfo 
% %%  & 7
% %%  & stdlib/list.ml %% Source 
%  \\
% 31
%  & \verb,find_all,\hobenchmark %% Name
%  & Function \verb,find_all, from list.ml. %% Desc 
%  & \sflinear %% SpecDesc 
% %%  & conses\_count\_st\_man.cmo %% MonSpec 
% %%  & \verb?pos <= pre + mlength l + mlength l? %% Harnesses 
% % &
% %% 0:24.44 %% RunInfo 
% %%  & 14
% %%  & stdlib/list.ml %% Source 
%  \\
% 32
%  & \verb,partition,\hobenchmark %% Name
%  & Function \verb,partition, from list.ml. %% Desc 
%  & \sflinear %% SpecDesc 
% %%  & conses\_count\_st\_man.cmo %% MonSpec 
% %%  & \verb?pos = pre + mlength l + mlength l? %% Harnesses 
% % &
% %% 5:42.12 %% RunInfo 
% %%  & 14
% %%  & stdlib/list.ml %% Source 
%  \\
% 33
%  & \verb,split, %% Name
%  & Function \verb,split, from list.ml. %% Desc 
%  & \sflinear %% SpecDesc 
% %%  & conses\_count\_st\_man.cmo %% MonSpec 
% %%  & \verb?pos <= pre + mlength l + mlength l? %% Harnesses 
% % &
% %% 0:14.65 %% RunInfo 
% %%  & 6
% %%  & stdlib/list.ml %% Source 
%  \\
% 34
%  & \verb,remove_assq, %% Name
%  & Function \verb,remove_assq, from list.ml. %% Desc 
%  & \sflinear %% SpecDesc 
% %%  & f\_apps\_count\_st\_man.cmo %% MonSpec 
% %%  & \verb?pos <= pre + mlength l? %% Harnesses 
% % &
% %% 0:17.94 %% RunInfo 
% %%  & 7
% %%  & stdlib/list.ml %% Source 
%  \\
% 35
%  & \verb,list_of_tree_aux,\benchmarkontrees %% Name
%  & Auxiliary function for depth first tree fold to list. %% Desc 
%  & \sflinear %% SpecDesc 
% %%  & conses\_count\_st\_man.cmo %% MonSpec 
% %%  & \verb?pos = pre + mtreesize t? %% Harnesses 
% % &
% %% 0:14.24 %% RunInfo 
% %%  & 9
% %%  & Dsolve/postests/treeset.ml %% Source 
%  \\
% 36
%  & \verb,list_of_tree,\benchmarkontrees %% Name
%  & Function for depth first tree fold to list. %% Desc 
%  & \sflinear %% SpecDesc 
% %%  & conses\_count\_st\_man.cmo %% MonSpec 
% %%  & \verb?pos = pre + mtreesize t? %% Harnesses 
% % &
% %% 0:14.53 %% RunInfo 
% %%  & 10
% %%  & Dsolve/postests/treeset.ml %% Source 
%  \\
% 37
%  & \verb,list_of_tree2,\benchmarkontrees %% Name
%  & Function for depth first tree fold to list, version 2. %% Desc 
%  & \sflinear %% SpecDesc 
% %%  & conses\_count\_st\_man.cmo %% MonSpec 
% %%  & \verb?pos = pre + mtreecost t? %% Harnesses 
% % &
% %% 0:36.08 %% RunInfo 
% %%  & 9
% %%  & Dsolve/postests/treeset.ml %% Source 
%  \\
% 38
%  & \verb,tree_of_list,\benchmarkontrees %% Name
%  & Function for constructing a tree from a list. %% Desc 
%  & \sflinear %% SpecDesc 
% %%  & f\_apps\_count\_st\_man.cmo %% MonSpec 
% %%  & \verb?pos = pre + mlength l? %% Harnesses 
% % &
% %% 0:16.43 %% RunInfo 
% %%  & 6
% %%  & Dsolve/postests/treeset.ml %% Source 
%  \\
% 39
%  & \verb,tree_of_list2,\benchmarkontrees %% Name
%  & Function for folding a tree into a list, version 2. %% Desc 
%  & \sflinear %% SpecDesc 
% %%  & f\_apps\_count\_st\_man.cmo %% MonSpec 
% %%  & \verb?pos = pre + mlistcost l? %% Harnesses 
% % &
% %% 2:30.75 %% RunInfo 
% %%  & 9
% %%  & Dsolve/postests/treeset.ml %% Source 
%  \\
% 40
%  & \verb,tree_flip,\benchmarkontrees %% Name
%  & Function for folding a tree into a list. %% Desc 
%  & \sflinear %% SpecDesc 
% %%  & f\_apps\_count\_st\_man.cmo %% MonSpec 
% %%  & \verb?pos = pre + mtreecost t? %% Harnesses 
% % &
% %% 0:25.08 %% RunInfo 
% %%  & 6
% %%  & Dsolve/postests/treeset.ml %% Source 
%  \\
% 41
%  & \verb,qsort, %% Name
%  & Quicksort. %% Desc 
%  & \sflinear %% SpecDesc 
% %%  & conses\_count\_st\_man.cmo %% MonSpec 
% %%  & \verb?pos <= pre + msquaredlen l? %% Harnesses 
% % &
% %% 1:26.75 %% RunInfo 
% %%  & 17
% %%  & \cite{Hofmann03} %% Source 
%  \\
% 42
%  & \verb,ins_sort, %% Name
%  & Insertion sort. %% Desc 
%  & \sflinear %% SpecDesc 
% %%  & conses\_count\_st\_man.cmo %% MonSpec 
% %%  & \verb?pos <= pre + 1 + mlength l && pos <= pre + mcostlist l? %% Harnesses 
% % &
% %% 0:31.20 %% RunInfo 
% %%  & 10
% %%  & \cite{Hofmann00} %% Source 
%  \\
% 43
%  & \verb,processfile3,\benchmarkonfiledescriptors %% Name
%  &  Open, read, and close one file, version 3. %% Desc 
%  & \sfchan %% SpecDesc 
% %%  & kobayashi\_cfu\_in\_st\_man.cmo %% MonSpec 
% %%  & \verb?Myset.eq (Myset.mns pos pre) Myset.empty? %% Harnesses 
% % &
% %% 0:14.96 %% RunInfo 
% %%  & 11
% %%  & \cite{Kobayashi09} %% Source 
%  \\
% 44
%  & \verb,processfile4,\benchmarkonfiledescriptors %% Name
%  &  Open, read, and close one file, version 4. %% Desc 
%  & \sfchan %% SpecDesc 
% %%  & kobayashi\_cfu\_multiple\_st\_man.cmo %% MonSpec 
% %%  & \verb?begin ... end? %% Harnesses 
% % &
% %% 2:03.84 %% RunInfo 
% %%  & 11
% %%  & \cite{Kobayashi09} %% Source 
%  \\
% 45
%  & \verb,compose_list, ver.2 %% Name
%  & Variant of Hofmann's composing (or folding) all functions in a list %% Desc 
%  & \sflinear %% SpecDesc 
% %%  & mh\_compose\_list\_st\_man.cmo %% MonSpec 
% %%  & \verb?pos = pre? %% Harnesses 
% % &
% %% 0:07.61 %% RunInfo 
% %%  & 6
% %%  & \cite{Hofmann02} %% Source 
%  \\
% 46
%  & \verb,list_of_leaves,\benchmarkontrees %% Name
%  & Left to right list of leaves of a tree. %% Desc 
%  & \sflinear %% SpecDesc 
% %%  & conses\_count\_st\_man.cmo %% MonSpec 
% %%  & \verb?pos = pre + mtreecost t? %% Harnesses 
% % &
% %% 0:27.52 %% RunInfo 
% %%  & 9
% %%  & Original creation %% Source 
%  \\
% 47
%  & \verb,left_path,\benchmarkontrees %% Name
%  & Given a tree, the path from the leftmost leave to the root. %% Desc 
%  & \sflinear, %% SpecDesc 
% %%  & conses\_count\_st\_man.cmo %% MonSpec 
% %%  & \verb?pos = pre + mtreecost t && pos <= pre + mtreesize t? %% Harnesses 
% % &
% %% 0:56.68 %% RunInfo 
% %%  & 9
% %%  & Original creation %% Source 
%  \\
% 48
%  & \verb,twice, %% Name
%  & Duplicate every element of a list. %% Desc 
%  & \sflinear %% SpecDesc 
% %%  & conses\_count\_st\_man.cmo %% MonSpec 
% %%  & \verb?pos = pre + mlength l + mlength l? %% Harnesses 
% % &
% %% 0:07.81 %% RunInfo 
% %%  & 4
% %%  & \cite{Hofmann02} %% Source 
%  \\
% 49
%  & \verb,add_if_pos1, %% Name
%  & Add a number if it is a positive integer, version 1. %% Desc 
%  & \sflinear %% SpecDesc 
% %%  & f\_apps\_count\_st\_man.cmo %% MonSpec 
% %%  & \verb?pos <= pre + 1? %% Harnesses 
% % &
% %% 0:05.54 %% RunInfo 
% %%  & 2
% %%  & Original creation %% Source 
%  \\
% 50
%  & \verb,add_if_pos2, %% Name
%  & Add a number if it is a positive integer, version2. %% Desc 
%  & \sflinear %% SpecDesc 
% %%  & f\_apps\_count\_st\_man.cmo %% MonSpec 
% %%  & \verb?(u <= 0 || pos <= pre + 1) && (u > 0 || pos = pre)? %% Harnesses 
% % &
% %% 0:06.75 %% RunInfo 
% %%  & 2
% %%  & Original creation %% Source 
%  \\
% 51
%  & \verb,fold_tree,\hobenchmark\benchmarkontrees %% Name
%  & Pre-order list fold. %% Desc 
%  & \sflinear %% SpecDesc 
% %%  & f\_apps\_count\_st\_man.cmo %% MonSpec 
% %%  & \verb?pos = pre + mtreesize t? %% Harnesses 
% % &
% %% 0:30.12 %% RunInfo 
% %%  & 8
% %%  & Original creation %% Source 
%  \\
% 52
%  & \verb,write_byte,\benchmarkonfiledescriptors %% Name
%  & Write byte to file. %% Desc 
%  & \sfchan %% SpecDesc 
% %%  & kobayashi\_cfu\_out\_st\_man.cmo %% MonSpec
% %%  & \verb?Myset.eq (Myset.mns pos pre) Myset.empty? %% Harnesses
% % &
% %% 0:12.33 %% RunInfo 
% %%  & 4
% %%  & Original creation %% Source 
%  \\

1
& \ttt{ack}
& The Ackermann function
\\ %& \liter \\

2
& \ttt{chop}
& Chop the first $n$ elements of a list
\\ %& \liter \\

3
& \ttt{dictionary}
&  An algebraic data type recursive  manipulation
\\ %& \liter \\

4
& \ttt{fold2} \hobenchmark
& Fold a pair of lists
\\ %& \liter \\

5
& \ttt{mccarthy91}
& The McCarthy 91 function
\\ %& \liter \\

6
& \ttt{mult}
& Recursive definition of multiplication
\\ %& \liter \\

7
& \ttt{rev\_append}
& Append a list reversed
\\ %& \liter \\

8
& \ttt{rev\_merge}
& Merge two lists
\\ %& \liter \\

9
& \ttt{simple-rec}
& A simple recursive function
\\ %& \liter \\

10
& \ttt{sum}
& The sum of the first $n$ naturals
\\ %& \liter \\

% sereni's examples

11
& \ttt{sereni29} \hobenchmark
& \ttt{map} applied  to a function constructed with
\ttt{compose} and a list
\\ %& \literrank \\

12
& \ttt{sereni56} \hobenchmark
& Computation of the n-th Church numeral using \ttt{compose}
\\ %& \literrank \\

13
& \ttt{sereni81} \hobenchmark
& Fold left defined using fold right
\\ %& \literrank \\

14
& \ttt{sereni85} \hobenchmark
& A program with two call sites to map
\\ %& \literrank \\

15
& \ttt{sereni163} \hobenchmark
& A parameter function applied to a
non-terminating parameter function
\\ %& \literrank \\

%%% Local Variables: 
%%% mode: latex
%%% TeX-master: "main"
%%% End: 


      \end{tabular}
      \\ \\
      % \\*
      % \benchmarkontrees Benchmark on trees. \\*
      % \benchmarkonfiledescriptors Benchmark on file descriptors.
    \end{minipage}
  }
  \caption{Verified benchmarks (benchmarks with higher order functions are
    marked with \hobenchmark.)}
  \label{fig-experiments}
\end{figure}

%%% Local Variables: 
%%% mode: latex
%%% TeX-master: "main"
%%% End: 

Our experiments consisted of two steps. First we applied our
transformation to the set of benchmarks summarized in
Figure~\ref{fig-experiments}. Then we analyzed the transformed
benchmarks using the reachability checker \dsolve~\cite{Dsolve, DsolveCAV}.
\ifthenelse{\equal{\isTechReport}{true}}{
The set of benchmarks is available at~\ttt{http://www7.in.tum.de/\~{}ruslan/binreach/}.
}{
The set of benchmarks is available at~\cite{FunVTechReport}.
}
Our benchmarks feature higher order functions and algebraic
data types (lists).  We summarize our verified benchmarks in
Figure~\ref{fig-experiments}.
Benchmarks 11-15 correspond to the higher-order programs
in~\cite{Sereni05terminationanalysis} that are strict and
\mbox{type-check}.
The experiments show that our transformation can be used together with a state of
the art static analyzer to prove termination of higher-order programs found
in the literature.

% \paragraph{Evaluation.}
% The experiments show that our transformation can be used together with a state of
% the art static analyzer to prove termination of higher-order programs found
% in the literature.

% For example, consider the higher-order benchmark \ttt{sereni29} shown below.
% \begin{verbatim}
% let rec length l =
%   match l with
%     | [] -> 0
%     | _ :: t -> 1 + length t
% let rec map f xs =
%   match xs with
%     | [] -> []
%     | y :: ys -> f y :: map f ys in
% let compose f g x = f (g x) in
% let add x y = x + y in
% map (compose (add 1) (add 2)) [0;1]
% \end{verbatim}
% We applied our program transformation on \ttt{sereni29} using the function
% identifier \ttt{map}, the number of parameters $2$, and the transition
% invariant candidate $\ttt{length}(a_2) \geq 0 \wedge
% \ttt{length}(m_2) > \ttt{length}(a_2)$. Then we obtained a
% transformed program that we verified correct using \dsolve.

%%% Local Variables: 
%%% mode: latex
%%% TeX-master: "main"
%%% End: 