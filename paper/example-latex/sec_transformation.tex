\section{Program transformation}
\label{sec-transformation}

In this section we present a program transformation that realizes the
function \monitor presented in Section~\ref{sec-semantics}.
To implement the state passing between judgemets we apply a so-called
logger monad.

\paragraph{\bf Logger monad}
A \emph{monad} consists of a type constructor \ttt{m} of arity~1 and two
operations
%
\begin{small}
\begin{align*}
  \ttt{unit~}&\ttt{:~'a~->~'a~m} \\
  \ttt{(~>>=~)~}&\ttt{:~'a~m~->~('a~->~'b~m)~->~'b~m}
\end{align*}
\end{small} 
%
These operations need to satisfy three conditions called \emph{left
unit}, \emph{right unit}, and \emph{associative}~\cite{Wadler95}.  
We assume that \ttt{state} is a given type.  
Figure~\ref{fig-state-monad} presents a variant of the \emph{logger
monad}~\cite{freetheorems}. The state
update operator \ttt{update} is of type
\ttt{(state~->~state)~->~unit~m}. A monadic expression (resp. value)
is an expression (resp. value) of the logger monad type. 

For example, the monadic expression \ttt{unit 1} evaluates to a
function that takes a state \st\ and returns a pair (\ttt{1}, \st). As
another example, the following monadic expression evaluates to a
function that takes a state \st\ and returns a pair (\ttt{1}, \st\ $+$
\ttt{1}).
\begin{center}
  \ttt{update (fun s -> s + 1) >>= fun () -> unit \ttt{1}}
\end{center}

\begin{figure}[t]
  \linespread{1}
  \begin{minipage}[t]{\columnwidth}
    \small
   \begin{tabular}{l@{\ }l}
     1 & \ttt{(* logger monad type *)} \\[\jot]
     2 & \ttt{type 'a m = state -> 'a * state } \\[\jot]
     3 & \ttt{(* unit operator *)} \\[\jot]
     4 & \ttt{let unit a = fun s -> (a, s) } \\[\jot]
     5 & \ttt{(* bind operator *)} \\[\jot]
     6 & \ttt{let ( >>= ) m k = fun s0 -> } \\[\jot]
     7 & \ttt{\tabTTTTTTTT let v1, s1 = m s0 in } \\[\jot]
     8 & \ttt{\tabTTTTTTTT k v1 s1 } \\[\jot]
     9 & \ttt{(* state transform operator *)} \\[\jot]
     10 & \ttt{let update f = fun s -> ( (), f s ) }
    \end{tabular}
  \end{minipage}
  \caption{Logger monad with state transform operator
    \ttt{update}. The unit operator takes a value and constructs a
    monadic value. The bind operator takes a monadic value and a
    function returning a monadic value, and constructs a new monadic
    value. The state transform operator creates a new monadic value by
    applying the state transformer \ttt{f}.}
  \label{fig-state-monad}
\end{figure}

\iffalse
We clarify the product expressions above by expressing them in
\haskell's \emph{do notation}~\cite{DoNotation}. For
the product of expression \ttt{1} and \msplus, the corresponding
\haskell\ program is the following.
\begin{center}
  \begin{minipage}{.4\columnwidth}
   \small
    \ttt{do \\
      \tabT update (fun s -> s) \\
      \tabT update (fun s -> s) \\
      \tabT unit \ttt{1}}
 \end{minipage}
\end{center}
For the product of expression \ttt{+} and \msplus, the corresponding
\haskell\ program is the following.
\begin{center}
  \begin{minipage}{.9\columnwidth}
    \small
    \ttt{do \\
      \tabT update (fun s -> s) \\
      \tabT update (fun s -> s + 1) \\
      \tabT unit (fun x -> unit (fun y -> unit ((+)\ x\ y)))} 
 \end{minipage}
\end{center}
\fi
%%% Local Variables:
%%% mode: latex
%%% TeX-master: "main"
%%% End: 


\paragraph{\bf Transformation of types}

We transform each program expression into a monadic expression that
keeps track of the state that results in the judgement augmentation.
Figure~\ref{fig-types-prods} presents the function \monadic\ that maps
types of expressions in the original program to types of the
transformed program.
Function \monadic\ indicates that a tranformed program is a \miniocaml\
function that takes an initial state and returns a \mbox{pre-monadic}
program value together with a final state.

\begin{figure}[t]
  \linespread{1}
  \begin{minipage}[t]{\columnwidth}
    \small
    \begin{tabular}{l@{\ }l}
      1 & \algLet \algRec \monadic\ \type\ = (\premonadic\ \type) \moncstr \\[\jot]
      & \\[\jot]
      2 & \algAnd \premonadic\ = \algFunction \\[\jot]
      3 & \tabT $|$ \ttt{\type[1] -> \type[2]}\
      \algArrow \ttt{$(\premonadic\ \type[1])$ -> $(\premonadic\
        \type[2])$ m} \\[\jot]
      % 6 & \tabT $|$ \ttt{\type[1]\ *\ \ldots\ *\ \type[n]}\
      % \algArrow \ttt{$(\premonadic\ \type[1])$\ *\ \ldots\ *\
      %   $(\premonadic\ \type[n])$} \\[\jot] 
      4 & \tabT $|$ \ttt{(\type[1],\ \ldots,\ \type[n])\ \tycstr}\ 
      \algArrow \ttt{($(\premonadic\ \type[1])$,\ \ldots,\ $(\premonadic\
      \type[n])$)\ \tycstr} \\[\jot]
      5 & \tabT $|$ \type\ \algArrow \type
    \end{tabular}
  \end{minipage}
  \caption{Type transformation function \monadic.}
  \label{fig-types-prods}
\end{figure}

For example, consider the following applications of \monadic.
%
\begin{align*}
  \monadic\ \texttt{(int -> (int -> int))} = \;
  \begin{array}[t]{@{}l@{}}
    \texttt{(int -> ((\premonadic\ int -> int) m)) m}\\[\jot]
    \texttt{(int -> ((int -> int m) m)) m}
    % (int -> ((prem int -> int) m)) m
    % (int -> ((int -> int m) m)) m
  \end{array}
\end{align*}
%
\begin{figure}[!t]
%  \hline
  \centering
  \begin{minipage}[t]{1.05\linewidth}
    \linespread{1.2}
    \begin{minipage}[t]{.03\linewidth}
      \small
      1 \\ 2 \\ 3 \\[\jot] \vspace{.3ex} \\ 4 \\ 5 \\ 6 \\7 \\8 \\ 9 \\
      10 \\ 11 \\ 12 \\ \\ 13 \\ 14 \\ 15 \\ 16
%      \gutternumbering{1}{29}
    \end{minipage}
    \begin{minipage}[t]{1.02\linewidth}
      \small
     % \algLet \tiCheck\ a\ = \\
     %  \tabT \quo{ \hspace{.5ex} \ttt{let m\_c, m\_1, ..., m\_\theArity\ = m in \\
     %    \tabTT if m\_c then assert(\theTI); \\
     %    \tabTT }a\ttt{, if not m\_c \&\& nondet () then true, }a\ttt{ else m }  } \\
     %  \\
      \algLet \tselEnter\ = \algFunction \\
      \tabT \algCase{\theFunction\ \expr[1]} \\ \tabTT
      \quo{ \ttt{fun v -> 
          fun ((\_\!\_, a$_2$, ..., a$_\theArity$), m) -> 
          (v, a$_2$, ..., a$_\theArity$), m} }
      \\
      \tabT \hspace{.5ex} \vdots\\
      \tabT \algCase{( \ldots\ (\theFunction\ \expr[1]) \dots\ \expr[\theArity-1])} \\ \tabTT
      \quo{ \ttt{fun v -> 
          fun ((a$_1$, ..., a$_{\theArity-2}$, \_\!\_, a$_\theArity$), m) -> 
%          \\ \hspace*{25ex}
          (a$_1$, ..., a$_{\theArity-2}$, v, a$_\theArity$), m} }
      \\
      \tabT \algCase{( \ldots\ (\theFunction\ \expr[1]) \dots\
        \expr[\theArity])} \\ 
%     \tabTT
%       \quo{ \ttt{fun v -> 
%           fun ((a$_1$, ..., a$_{\theArity-1}$, \_\!\_), m) -> 
% %          \\ \hspace*{25ex}
%           \aq{
%             \tiCheck\ 
%             \quo{(a$_1$, ..., a$_{\theArity-1}$, v)}} }}
%       \\
      \tabTT
      \quo{ \ttt{fun v -> 
          fun ((a$_1$, ..., a$_{\theArity-1}$, \_\!\_), m) -> \\ \hspace*{25ex}
          % let a$_{\theArity}$ =  v in           \\ \hspace*{25ex}
          let a = a$_1$, ..., a$_{\theArity-1}$, v in           \\ \hspace*{25ex}
          let m\_c, m\_1, ..., m\_\theArity\ = m in \\ \hspace*{25ex}
           if m\_c then assert\ \theTI ; \\ \hspace*{25ex}
          a, if not m\_c \&\& nondet () then true, a else m} } \\
      \tabT \algCase{\_\!\_}\ \quo{ \ttt{fun \_\!\_ -> id} }
      \\
      \\
      \algLet \tselExit\ = \algFunction \\
      \tabT \hspace{1ex}$|$ \theFunction\ \expr[1] $| \, \ldots \, |$ ( \ldots\ (\theFunction\ \expr[1]) \dots\ \expr[\theArity - 1]) $\rightarrow$
      \quo{ \ttt{fun \_\!\_ s -> (fun \_\!\_ -> s) } }
      \\
      \tabT \algCase{( \ldots\ (\theFunction\ \expr[1]) \dots\ \expr[\theArity])}
      \quo{ \ttt{fun s \_\!\_ -> (fun \_\!\_ -> s) } }
      \\
      \tabT \algCase{\_\!\_}\ \quo{ \ttt{fun \_\!\_ \_\!\_ -> id} }
    \end{minipage}
  \end{minipage}
  \caption{The transformer selector functions \tselEnter\ and \tselExit.  
    The operator \quo{$\;\cdot\;$} emits a \miniocaml\ expression after evaluating
    expressions that are embedded using~\aq{$\;\cdot\;$}.  }
  \label{fig-tsel}
\end{figure}
%%% Local Variables: 
%%% mode: latex
%%% TeX-master: "main"
%%% End: 


\begin{figure}[p]
%  \hline
  \centering
  \begin{minipage}[t]{\linewidth}
    \linespread{1.2}
    \begin{minipage}[t]{.02\linewidth}
      \small
%      \mbox{} \\
      \gutternumbering{1}{36}
    \end{minipage}
    \begin{minipage}[t]{.962\linewidth}
      \small
%      \mbox{} \\
      \algLet \product\ \expr\ = \\
      \tabT \algMatch \expr\ \algWith
      \\
      \tabT \algCase{\cst\ } \\
      \tabTT \algLet \var[1], \ldots, \var[\arity\ \cst]\ = \freshvar, \ldots, \freshvar\ \algIn \\
      \tabTTT\hspace{-2.3ex}
      \quo{
        \ttt{unit (fun \var[1]-> \\ \tabTTTT 
          \ldots\ \\ \tabTTTTT 
          unit (fun \var[\arity\ \cst]-> \\  \tabTTTTTT 
          unit (\cst\ \var[1] \ldots\ \var[\arity\ \cst])) \ldots\ )
        }}\\
      \iffalse
      \tabT \algCase{\cst\ } \\
      \tabTT \algLet \var[1], \ldots, \var[\arity\ \cst]\ = \freshvar, \ldots, \freshvar\ \algIn \\
      \tabTTT\hspace{-2.3ex}
      \quo{
        \ttt{unit (fun \var[1]-> \\ \tabTTTT 
          \ldots\ \\ \tabTTTTT 
          unit (fun \var[\arity\ \cst]-> \\  \tabTTTTTT 
          unit (\cst\ \var[1] \ldots\ \var[\arity\ \cst])) \ldots\ )
        }
      }\\
      \fi
      \tabT \algCase{\var\ } % \\      \tabTTT\hspace{-2.3ex}
      \quo{
        \ttt{unit \var}
      }
      \\
%      % \tabT \algCase{\ntupleexpr\ \algWhen $n > 1$} \\
      % \tabTT \algLet \var[1], \ldots, \var[n]\ = \freshvar, \ldots, \freshvar\ \algIn \\
      % \tabTTT\hspace{-2.3ex}
      % \quo{
      %   \ttt{\aq{\product\ \expr[n]} >>= fun \var[n]-> %\\ \tabTTT 
      %     \ldots %\\ \tabTTT 
      %     \aq{\product\ \expr[1]} >>= fun \var[1]-> %\\ \tabTTT 
      %     unit (\var[1], \ldots, \var[n])}~}
%      % \\
      \tabT \algCase{\ttt{\cstr(\ntupleexpr)}} \\
      \tabTT \algLet \var[1], \ldots, \var[n]\ = \freshvar, \ldots, \freshvar\ \algIn \\
      \tabTTT\hspace{-2.3ex}
      \quo{
        \ttt{
          % update \aq{\tsel\ \expr\ $\uparrow$} >>= fun () ->\\
          % \tabTTT
          \aq{\product\ \expr[n]} >>= fun \var[n] -> \\
          \tabTTT \ldots \\
          \tabTTT \aq{\product\ \expr[1]} >>= fun \var[1] -> \\
          % \tabTTT update \aq{\tsel\ \expr\ $\downarrow$} >>= fun () -> \\
          \tabTTT unit (\cstr (\var[1], \ldots, \var[n]))
        }
      }
      \\
      \tabT \algCase{\expr[f]\ \expr[p]\ } \\
      \tabTT \algLet \var[app], \var[p], \vst[\mathit{full}]\ \vst[\mathit{partial}] = \freshvar, \freshvar, \freshvar, \freshvar\ \algIn \\
      \tabTTT\hspace{-2.3ex}
      \quo{
        \ttt{
          fun \vst[\mathit{full}]\ -> \\
          \tabTTTT (\aq{\product\ \expr[p]} >>= fun \var[p] -> \\
          \tabTTTT \aq{\product\ \expr[f]} >>= fun \var[f] -> \\
          \tabTTTT update (\aq{\tselEnter\ \expr } \var[p]) >>= fun () \vst[\mathit{partial}]\  -> \\
          \tabTTTT (\var[f]\ \var[p] >>= fun \varapp\ -> \\
          \tabTTTT update (\aq{\tselExit\ \expr} \vst[\mathit{full}] \vst[\mathit{partial}]) >>= fun () -> \\
          \tabTTTT unit \varapp) \vst[\mathit{partial}]) \vst[\mathit{full}]
        }
      }
      \\
      \tabT
      \algCase{ \funexpr\ } %\\      \tabTT\hspace{-3ex}
      \quo{ \ttt{unit (fun \var\ -> \aq{\product\ \expr[b]})} }
      \\
      \tabT \algCase{ \ttt{let \patt\ = \expr[1]\ in \expr[2]}\ } 
%      \\ \tabTTTT\hspace{-2.3ex}
      \quo{
        \ttt{( \aq{\product\ \expr[1]} %\hspace{2.5ex}
          >>= fun \patt\ -> %\\  \tabTTTT 
          \aq{\product\ \expr[2]} )}
      }
      \\
      \tabT
      \algCase{ \ttt{let rec \var[f] = \funexpr\ in \expr[2]}\ } 
      \\ \tabTTTT\hspace{-2.3ex}
      \quo{
        \ttt{( let rec \var[f] = fun \var\ -> \aq{\product\ \expr[b]} 
          in \aq{\product\ \expr[2]})}
      }
      \\
      \tabT
      \algCase{ 
        \ttt{match \expr[m]\ with} %\\
        \ttt{ | \ensuremath{e^p_1}\ -> \expr[1]} % \\
        \ttt{\tabT \ldots} %\\
        \ttt{ |  \ensuremath{e^p_i}\ -> \expr[i]}
      } \\
      \tabTTT \algLet \var[m]
      \ = \freshvar
      \ \algIn \\
      \tabTTTT\hspace{-2.3ex}
      \quo{
        \ttt{
          % update \aq{\tsel\ \expr\ $\uparrow$} >>= fun () -> \\
          % \tabTTT
          \aq{\product\ \expr[m]} >>= fun \var[m]\ -> \\
          \tabTTTT  ( match \var[m] with \\ \tabTTTTT 
          | \ensuremath{e^p_1}\ -> \aq{\product\ \expr[1]}  \\
          \tabTTTTT \ldots  \\
          \tabTTTTT| \ensuremath{e^p_i}\ -> \aq{\product\ \expr[i]} \\ \tabTTTT 
          )
        }
      }
      \iffalse
      \\
      \tabT
      \algCase{ 
        \ttt{match \expr[m]\ with} \\
        \ttt{\tabTT | \ensuremath{e^p_1}\ -> \expr[1]} \\
        \ttt{\tabTT \ldots} \\
        \ttt{\tabTT |  \ensuremath{e^p_i}\ -> \expr[i]}
      } \\
      \tabTTT \algLet \var[m]
%      , \var[v]
      \ = \freshvar
%      , \freshvar
      \ \algIn \\
      \tabTTTT\hspace{-2.3ex}
      \quo{
        \ttt{
          % update \aq{\tsel\ \expr\ $\uparrow$} >>= fun () -> \\
          % \tabTTT
          \aq{\product\ \expr[m]} >>= fun \var[m]\ -> \\
          \tabTTTT  ( match \var[m] with \\
          \tabTTTTT | \ensuremath{e^p_1}\ -> \aq{\product\ \expr[1]} \\
          \tabTTTTT \ldots \\
          \tabTTTTT | \ensuremath{e^p_i}\ -> \aq{\product\ \expr[i]} \\
          \tabTTTT ) %>>= fun \var[v]\ -> \\
          % \tabTTT  update \aq{\tsel\ \expr\ $\downarrow$} >>= fun () -> \\
%          \tabTTT unit \var[v]
        }
      }
      \fi
    \end{minipage}
  \end{minipage}
  \caption{The transformation function \product.  
    The operator \quo{$\;\cdot\;$} emits a \miniocaml\ expression after evaluating
    expressions that are embedded using~\aq{$\;\cdot\;$}.  }
  \label{fig-prod-alg}
\end{figure}
\iffalse  

\begin{figure}[t]
  %
  \begin{minipage}[t]{\linewidth}
    \linespread{1.2}
    \begin{minipage}[t]{.04\linewidth}
      \small
      \mbox{} \\
      \gutternumbering{31}{48}
    \end{minipage}
    \begin{minipage}[t]{.94\linewidth}
      \small
      \mbox{} \\
      % \algCase{\expr[f]\ \expr[p]\ } \\
      % \tabT \algLet \var[f], \var[p]\ = \freshvar, \freshvar\ \algIn \\
      % \tabTTT\hspace{-2.3ex}
      % \quo{
      %   \ttt{
      %     \aq{\product\ \expr[p]} >>= fun \var[p] -> \\
      %     \tabTTT \aq{\product\ \expr[f]} >>= fun \var[f] -> \\
      %     \tabTTT \var[f]\ \var[p]
      %   }
      % }
      % \\
      \algCase{\funexpr\ } %\\      \tabTT\hspace{-3ex}
      \quo{
        \ttt{
          % update \aq{\tsel\ \expr\ $\uparrow$} >>= fun () ->\\
          % \tabTT update \aq{\tsel\ \expr\ $\downarrow$} >>= fun () -> \\
          % \tabTT
          unit (fun \var\ -> \aq{\product\ \expr[b]})
        }
      }
      \\
      \algCase{ \ttt{let \patt\ = \expr[1]\ in \expr[2]}\ } \\
%      \tabTT \algLet \var[v]\ = \freshvar\ \algIn \\
      \tabTTT\hspace{-2.3ex}
      \quo{
        \ttt{
          % update \aq{\tsel\ \expr\ $\uparrow$} >>= fun () -> \\
          % \tabTTT\hspace{-0.5ex} 
          ( \aq{\product\ \expr[1]} %\hspace{2.5ex}
          >>= fun \patt\ -> %\\  \tabTTTT 
          \aq{\product\ \expr[2]} ) % >>= fun \var[v] -> \\
          % \tabTTT update \aq{\tsel\ \expr\ $\downarrow$} >>= fun () -> \\
 %         \tabTTT unit \var[v]
        }
      }
      \\
      \algCase{ \ttt{let rec \var[f] = \funexpr\ in \expr[2]}\ } \\
%      \tabTT \algLet \var[v]\ = \freshvar\ \algIn \\
      \tabTTT\hspace{-2.3ex}
      \quo{
        \ttt{
          % update \aq{\tsel\ \expr\ $\uparrow$} >>= fun () ->\\
          % \tabTTT\hspace{-0.5ex}
          ( let rec \var[f] = fun \var\ -> \aq{\product\ \expr[b]} 
          \\ \tabTTTT 
          in \aq{\product\ \expr[2]} ) % >>= fun \var[v]\ -> \\
          % \tabTTT update \aq{\tsel\ \expr\ $\downarrow$} >>= fun () -> \\
%          \tabTTT unit \var[v]
        }
      }
      \\
      \algCase{ 
        \ttt{match \expr[m]\ with} \\
        \ttt{\tabT | \ncstrpatt{1}\ -> \expr[1]} \\
        \ttt{\tabT \ldots} \\
        \ttt{\tabT | \ncstrpatt{i}\ -> \expr[i]}
      } \\
      \tabTT \algLet \var[m]
%      , \var[v]
      \ = \freshvar
%      , \freshvar
      \ \algIn \\
      \tabTTT\hspace{-2.3ex}
      \quo{
        \ttt{
          % update \aq{\tsel\ \expr\ $\uparrow$} >>= fun () -> \\
          % \tabTTT
          \aq{\product\ \expr[m]} >>= fun \var[m]\ -> \\
          \tabTTT  ( match \var[m] with \\
          \tabTTTT | \ncstrpatt{1}\ -> \aq{\product\ \expr[1]} \\
          \tabTTTT \ldots \\
          \tabTTTT | \ncstrpatt{i}\ -> \aq{\product\ \expr[i]} \\
          \tabTTT ) %>>= fun \var[v]\ -> \\
          % \tabTTT  update \aq{\tsel\ \expr\ $\downarrow$} >>= fun () -> \\
%          \tabTTT unit \var[v]
        }
      }
      \\
    \end{minipage}
  \end{minipage}
  \caption{The product construction function \product.  It takes as
    input an expression \expr.
    The output is the product of \expr.  The operator
    \quo{$\;\cdot\;$} emits a \miniocaml\ expression after evaluating
    the embedded programs~\aq{$\;\cdot\;$}.  }
  \label{fig-prod-alg}
\end{figure}
\fi
%%% Local Variables: 
%%% mode: latex
%%% TeX-master: "main"
%%% End: 


\paragraph{\bf Transformation of expressions}


We present the transformation function \product\ in
Figure~\ref{fig-prod-alg}. 
\product uses two auxiliary functions $\tselEnter$ and $\tselExit$
shown in Figure~\ref{fig-tsel}.

For a expression \expr, \product\ traverses the abstract syntax tree
of \expr\ and gives a core monadic expression that evaluates the user
program together with two state transform operations.
\product generates \miniocaml expressions using the $\quo{\cdot}$
function.
For example, $\quo{\ttt{let x = 1 in 1}}$ emits the expression
\ttt{let x = 1 in 1}.
Within $\quo{\cdot}$ we can perform an evaluation by
applying~$\aq{\cdot}$.
For example, $\quo{\ttt{let x = \aq{\quo{1+2}} in 1}}$ emits $\ttt{let
x = 1+2 in 1}$.

The important case is the tranformation of
$\theFunction/\theArity$-applications.
Such applications are recognized in $\tselEnter$ and $\tselExit$.
The emitted code either saves the argument values into the state, or
propagates further the current state.
Furthermore, $\tselEnter$ performs a check if the snapshot stored in a
state together with the arguments of a
$\theFunction/\theArity$-application satisfy the transition invariant
candidate~$\theTI$.
This check is guarded by the condition that the snapshot must have
been stored previously.

We show an example application of \product in Figure~\ref{fig-ex-prod}
for analysing $\texttt{f}/1$-applications.
First, we present subexpressions of the program and then show the
result of the application of \product on them (we have partially
simplified the transformed expressions to improve readability).
%
\begin{figure}[!t]
  \centering
{
\small
\begin{verbatim}
e = let rec f x = if x > 0 then f (x - 1) else fun y -> f x y in f 1
e1 = if x > 0 then f (x - 1) else fun y -> f x y
e2 = x > 0
e3 = 0
e4 = (>) x
e5 = x
e6 = (>)
e7 = f (x - 1)
...

Transform e = let rec f = fun x -> 'Transform e1' in 'Transform e20'
Transform e1 = 'Transform e2' >>= fun x2 -> (if x2 then 'Transform e7'
                                             else 'Transform e14')
Transform e2 = fun s_full -> ('Transform e3' >>= fun x3 ->
                              'Transform e4' >>= fun x4 ->
                              update (fun s -> s) >>= fun () s_partial ->
                             (x4 x3 >>= fun xapp2 ->
                              update (fun s -> s) >>= fun () ->
                              unit xapp2) s_partial) s_full
Transform e3 = unit 0
Transform e4 = fun s_full -> ('Transform e5' >>= fun x5 ->
                              'Transform e6' >>= fun x6 ->
                              update (fun s -> s) >>= fun () s_partial ->
                             (x6 x5 >>= fun xapp4 ->
                              update (fun s -> s) >>= fun () ->
                              unit xapp4) s_partial) s_full
Transform e5 = unit x
Transform e6 = unit (fun z1 -> unit (fun z2 -> unit (z1 > z2)))
Transform e7 = fun s_full -> ('Transform e8' >>= fun x8 ->
                              'Transform e13' >>= fun x13 ->
                              update (fun ( _, m ) ->
                                let m_c, m_1 = m in
                                if m_c then assert ( x8 > 0 && m_1 > x8 );
                                x8, if not m_c && nondet () then true, x8 else m
                              ) >>= fun () s_partial ->
                             (x13 x8 >>= fun xapp7 ->
                              update (fun _ -> s_full) >>= fun () ->
                              unit xapp7) s_partial) s_full
...
\end{verbatim}
}
  \caption{Example application of \product\ with $N = 1$ and \mbox{$\theTI = (\
      a_1 > 0 \wedge m_1 > a_1 \ )$}. Given a stored snapshot, the
    transformed application \ttt{f (x - 1)} checks that the current
    actual satisfies $\theTI$.}
  \label{fig-ex-prod}
\end{figure}
%

We establish a relationship between augmented evaluation trees and
evaluation trees of the transformed program in the following theorem.
%
\begin{theorem}[\product implements \monitor]
\label{thm-product}
%
A pair $(v_1, \dots, v_\theArity)$ and~$(u_1, \dots, u_\theArity)$ is
obtained from the augmented evaluation tree as described in
Theorem~\ref{thm-monitor}
if and only if a judgement of the following form appears in the
evaluation tree of the
program obtained by applying \product:
%
\begin{center}
  \mevj{\evj{\ectx}{$\theFunction$\ \expr[1]\ \_\!\_\ \dots\
      \expr[\theArity]\ s}{\val}}{\stupa}{}
\end{center}
%
such that $\tsf{eval}\ \ectx\ s = (true, (v_1, \dots, v_\theArity))$
and for each $i \in 1..\theArity$ we have \mbox{$\tsf{eval}\ \ectx\ \expr[i]
= u_i$}.
%
\end{theorem}
%

The following corollary of Theorem~\ref{thm-product} allows one to
rely on the assertion validity in the transformed program to implement
the binary reachability analysis of the original program.
%
\begin{theorem}[Binary reachability analysis as assertion checking]
\label{thm-assert}
%
Each pair $(v_1, \dots, v_\theArity)$ and~$(u_1, \dots, u_\theArity)$
in the $\theFunction/\theArity$-recursion relation of the program 
satisfies $\theTI$ if and only if the assertion inserted by
$\tselEnter$ is valid in the transformed program.
%
\end{theorem}
%


%%% Local Variables:
%%% mode: latex
%%% TeX-master: "main"
%%% End: 
